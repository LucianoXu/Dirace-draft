\documentclass{article}

\usepackage[utf8]{inputenc}
\usepackage[margin=0.8in]{geometry}

\usepackage{color}
\usepackage{amsmath}
%\usepackage{amssymb}
\usepackage{graphicx}
\usepackage{amsthm}
\usepackage{stmaryrd}
\usepackage[all]{xy}
\usepackage{multirow}
\usepackage{paralist}
\usepackage{hhline}
\usepackage{bm}
\usepackage{longtable}
\usepackage{makecell}
\usepackage{mdframed}

\usepackage{wasysym}
\usepackage{extarrows}
\usepackage{tikz}

\newtheorem{theorem}{Theorem}[section]
\newtheorem{corollary}{Corollary}[section]
\newtheorem{lemma}{Lemma}[section]
\newtheorem{remark}{Remark}[section]
\newtheorem{definition}{Definition}[section]
\newtheorem{proposition}{Proposition}[section]
\newtheorem{example}{Example}[section]
\newtheorem{problem}{Problem}[section]
\newtheorem{principle}{Principle}
\newtheorem{conjecture}{Conjecture}
\newtheorem{postulate}{Postulate}
\newtheorem{fact}{Fact}
\newtheorem{claim}{Claim}


% add new commands for comments here
\newcommand{\yx}[1]{\textit{\color{blue}[YX] : #1}}

\newcommand{\modify}[1]{{\color{red}#1}}

\usepackage{amssymb}
\usepackage{amsmath}
\usepackage{stmaryrd}
\usepackage{hyperref}

\usepackage{braket}

% framed environment
\newmdenv[
  skipabove=5pt,
  skipbelow=5pt,
  leftmargin=0pt,
  rightmargin=0pt,
  innerleftmargin=5pt,
  innerrightmargin=5pt,
  innertopmargin=5pt,
  innerbottommargin=5pt,
  linecolor=black,
]{framedeq}


\title{\textbf{DiracDec C++/Coq Implementation\cite{ExampleCitation}}}

\begin{document}

\maketitle

%%%%%%%%%%%%%%%%

\newcommand{\reduce}{\triangleright}

\newcommand{\Sort}{\mathsf{Sort}}
\newcommand{\WF}{\mathcal{WF}}

\newcommand{\Type}{\mathsf{Type}}
\newcommand{\BaseS}{\mathsf{Base}}

\newcommand{\SType}{\mathcal{S}}
\newcommand{\KType}{\mathcal{K}}
\newcommand{\BType}{\mathcal{B}}
\newcommand{\OType}{\mathcal{O}}

\newcommand{\ZEROK}{\mathbf{0}_\mathcal{K}}
\newcommand{\ZEROB}{\mathbf{0}_\mathcal{B}}
\newcommand{\ZEROO}{\mathbf{0}_\mathcal{O}}

\newcommand{\PAIR}{\mathsf{PAIR}}

\newcommand{\ZERO}{\mathsf{0}}
\newcommand{\ONE}{\mathsf{1}}
\newcommand{\ADDS}{\mathsf{ADDS}}
\newcommand{\ADD}{\mathsf{ADD}}
\newcommand{\MULS}{\mathsf{MULS}}
\newcommand{\MUL}{\mathsf{MUL}}
\newcommand{\CONJ}{\mathsf{CONJ}}
\newcommand{\CJG}{\mathsf{CJG}}
\newcommand{\ADJ}{\mathsf{ADJ}}
\newcommand{\DELTA}{\mathsf{DELTA}}
\newcommand{\DOT}{\mathsf{DOT}}
\newcommand{\SCR}{\mathsf{SCR}}
\newcommand{\TSR}{\mathsf{TSR}}
\newcommand{\KET}{\mathsf{KET}}
\newcommand{\BRA}{\mathsf{BRA}}
\newcommand{\ONEO}{\mathbf{1}_\mathcal{O}}
\newcommand{\OUTER}{\mathsf{OUTER}}
\newcommand{\MULK}{\mathsf{MULK}}
\newcommand{\MULB}{\mathsf{MULB}}
\newcommand{\MULO}{\mathsf{MULO}}

%%%%%%%%%%%%%%%%%%%%%%%%%%%


\section{Language Syntax}
\begin{definition}[syntax]
    The syntax for Dirac notation types is defined as 
    \begin{align*}
        T ::=\ & x\ |\ \SType\ |\ \KType(\sigma)\ |\ \BType(\sigma)\ |\ \OType(\sigma_1, \sigma_2)\ |\ T_1 \to T_2, \\
        \sigma ::=\ & x\ |\ \sigma_1 \times \sigma_2.
    \end{align*}
    The syntax for Dirac notation terms is defined as
    \begin{align*}
        e ::=\ & x\ |\ (e_1, e_2) \\
        & |\ 0\ |\ 1\ |\ \ADDS(e_1 \cdots e_n) \ |\ e_1 \times \cdots \times e_n\ |\ e^* \ |\ \delta_{e_1, e_2}\ |\ \DOT(e_1\ e_2) \\
        & |\ \ZEROK(\sigma)\ |\ \ZEROB(\sigma)\ |\ \ZEROO(\sigma_1, \sigma_2)\ |\ \ONEO(\sigma) \\
        & |\ \ket{e}\ |\ \bra{t}\ |\ e^\dagger\ |\ e_1.e_2\ |\ \ADD(e_1 \cdots e_n)\ |\ e_1 \otimes e_2 \\
        & |\ \MULK(e_1\ e_2)\ |\ \MULB(e_1\ e_2)\ |\ \OUTER(e_1\ e_2)\ |\ \MULO(e_1\ e_2).
    \end{align*}
\end{definition}
 % after consideration I still think keep DOT as binary is the best choice. It reduces the unstability of complicated matching in sequences.

Compared to [?], this syntax for Dirac notations merges the symbols with overlapped properties, such as the addition and scaling symbols for ket, bra and operator.
Here $\ADDS$ and $\ADD$ are two different AC symbols representing the scalar addition and the linear algebra addition respectively. They will be denoted as $a_1 + \cdots + a_n$ and $X_1 + \cdots + X_n$.
There are five kinds of linear algebra multiplications among ket, bra and operator, whose properties are similar but still diverge to some extent. For example, the rules $(O_1 \cdot O_2) \cdot K\ \reduce\ O_1 \cdot (O_2 \cdot K)$ and $B \cdot (O_1 \cdot O_2) \ \reduce\ (B \cdot O_1) \cdot O_2$ indicate that the sorting of multiplication sequences depends on the subterm types. To avoid frequent but unnecessary type checkings, we encode the typing information by using five different symbols, namely $\DOT$, $\MULK$, $\MULB$, $\OUTER$ and $\MULO$. They are denoted as $B\cdot K$, $K_1 \cdot K_2$, $B_1 \cdot B_2$, $K \cdot B$ and $O_1 \cdot O_2$, respectively.
 
\section{Typing System}
The type checking of Dirac notations involves maintaining a well-formed context $\Gamma$, which specifies the definitions and typing assumtpions for variables.
Notice that the type checking $\Gamma \vdash t : T$ is decidable.
\begin{figure}[h]
    \begin{gather*}
        \textbf{W-Empty} \qquad
        \frac{}{\WF([])}
        \qquad \qquad
        \textbf{W-Assum-Type} \qquad
        \frac{\WF(\Gamma) \qquad T \notin \Gamma}{\WF(\Gamma; T:\Type)}\\
        \\
        \textbf{W-Assum-Term} \qquad
        \frac{\Gamma\vdash T:\Type \qquad x \notin \Gamma}{\WF(\Gamma; x:T)}
        \qquad \qquad
        \textbf{W-Def} \qquad
        \frac{\Gamma\vdash t:T \qquad x \notin \Gamma}{\WF(\Gamma; x:=t:T)} \\
        \\
        \textbf{Var} \qquad
        \frac{\WF(\Gamma) \qquad (x:T) \in \Gamma \text{ or } (x := t : T) \in \Gamma \text{ for some $t$}}{\Gamma \vdash x : T} \\
        \\
        \textbf{Arrow} \qquad
        \frac{\Gamma \vdash T:\Type \qquad \Gamma \vdash U:\Type}{\Gamma \vdash T \to U : \Type} \\
        \\
        \textbf{Lam} \qquad
        \frac{\Gamma;x:T \vdash t : U}{\Gamma \vdash (\lambda x:T . t) : T \to U}
        \qquad \qquad
        \textbf{App} \qquad
        \frac{\Gamma \vdash t:U \to T \qquad \Gamma \vdash u:U}{\Gamma \vdash (t\ u):T}
    \end{gather*}
    \caption{The typing rules for the typed lambda calculus with environment.}
\end{figure}

\begin{figure}[h]
    \begin{gather*}
        \textbf{Type-Base} \qquad
        \frac{\WF(\Gamma)}{\Gamma \vdash \BaseS : \Type} 
        \qquad \qquad
        \textbf{Base-Subtype} \qquad
        \frac{\Gamma \vdash T : \BaseS}{\Gamma \vdash T : \Type} \\
        \\
        \textbf{Type-Ket} \qquad
        \frac{\Gamma \vdash \sigma : \BaseS}{\Gamma \vdash \KType(\sigma) : \Type}
        \qquad \qquad
        \textbf{Type-Bra} \qquad
        \frac{\Gamma \vdash \sigma : \BaseS}{\Gamma \vdash \BType(\sigma) : \Type} \\
        \\
        \textbf{Type-Opt} \qquad
        \frac{\Gamma \vdash \sigma : \BaseS \qquad \Gamma \vdash \tau : \BaseS}{\Gamma \vdash \OType(\sigma, \tau) : \Type}
        \qquad \qquad
        \textbf{Type-Scalar} \qquad
        \frac{\WF(\Gamma)}{\Gamma \vdash \SType : \Type}
    \end{gather*}
    \caption{Typing rules for Dirac notation types.}
\end{figure}

\begin{figure}[h]
    \begin{gather*}
        \textbf{Sca-0} \qquad
        \frac{\WF(\Gamma)}{\Gamma \vdash 0 : \SType}
        \qquad \qquad
        \textbf{Sca-1} \qquad
        \frac{\WF(\Gamma)}{\Gamma \vdash 1 : \SType}
        \qquad \qquad
        \textbf{Sca-Delta} \qquad
        \frac{            
            \begin{aligned}
                & \Gamma\vdash \sigma : \BaseS \\
                & \Gamma\vdash s : \sigma \qquad \Gamma\vdash t : \sigma
            \end{aligned}} {\Gamma \vdash \delta_{s, t} : \SType} \\
        \\
        \textbf{Sca-Add} \qquad
        \frac{\Gamma\vdash a_i : \SType \text{ for all $i$}}{\Gamma\vdash a_1 + \cdots + a_n : \SType}
        \qquad \qquad
        \textbf{Sca-Mul} \qquad
        \frac{\Gamma\vdash a_i : \SType \text{ for all $i$}}{\Gamma\vdash a_1 \times \cdots \times a_n : \SType} \\
        \\
        \textbf{Sca-Conj} \qquad
        \frac{\Gamma \vdash a : \SType}{\Gamma \vdash a^*:\SType}
        \qquad \qquad
        \textbf{Sca-Dot} \qquad
        \frac{\Gamma\vdash B : \BType(\sigma) \qquad \Gamma\vdash K : \KType(\sigma)}{\Gamma \vdash B \cdot K : \SType}
    \end{gather*}
    \caption{Scalar typing rules.}
\end{figure}

\begin{figure}[h]
    \begin{gather*}
        % this two rules can be considered later
        % \textbf{Base-Def} \qquad
        % \frac{\WF(\Gamma) \qquad a_i \notin \Gamma \text{ for all $i$}}
        % {\Gamma\vdash \{a_1, a_2, \cdots, a_n\} : \BaseS}
        % \qquad \qquad
        % \textbf{Base-Element} \qquad
        % \frac{\WF(\Gamma) \qquad c \in \{a_1, a_2, \cdots a_n\}}{\Gamma\vdash c : \{a_1, a_2, \cdots, a_n\}} \\
        % \\
        \textbf{Base-Prod} \qquad
        \frac{\Gamma\vdash \sigma : \BaseS \qquad \Gamma\vdash \tau : \BaseS}{\Gamma\vdash \sigma \times \tau : \BaseS}
        \qquad \qquad
        \textbf{Pair-Base} \qquad
        \frac{
            \begin{aligned}
                & \Gamma\vdash \sigma : \BaseS \qquad \Gamma\vdash s : \sigma \\
                & \Gamma\vdash \tau : \BaseS \qquad \Gamma\vdash t : \tau
            \end{aligned}} {\Gamma\vdash (s, t) : \sigma \times \tau} 
        \textbf{}
    \end{gather*}
    \caption{Typing rules for bases. }
\end{figure}


\begin{figure}[h]
    \begin{gather*}
        \textbf{Ket-0} \qquad
        \frac{\Gamma \vdash \sigma : \BaseS}{\Gamma \vdash \ZEROK(\sigma) : \KType(\sigma)} 
        \qquad \qquad
        \textbf{Ket-Base} \qquad
        \frac{\Gamma \vdash \sigma : \BaseS \qquad \Gamma\vdash t : \sigma}{\Gamma \vdash \ket{t} : \KType(\sigma)} \\
        \\
        \textbf{Ket-Adj} \qquad
        \frac{\Gamma \vdash B : \BType(\sigma)}{\Gamma \vdash B^\dagger : \KType(\sigma)} 
        \qquad \qquad
        \textbf{Ket-Scr} \qquad
        \frac{\Gamma \vdash a : \SType \qquad \Gamma \vdash K : \KType(\sigma)}{\Gamma \vdash a.K : \KType(\sigma)} \\
        \\
        \textbf{Ket-Add} \qquad
        \frac{\Gamma \vdash K_i : \KType(\sigma) \text{ for all $i$}}{\Gamma \vdash K_1 + \cdots + K_n : \KType(\sigma)}
        \qquad \qquad
        \textbf{Ket-MulK} \qquad
        \frac{\Gamma \vdash O : \OType(\sigma, \tau) \qquad \Gamma \vdash K : \KType(\tau)}{\Gamma \vdash O \cdot K : \KType(\sigma)} \\
        \\
        \textbf{Ket-Tsr} \qquad
        \frac{\Gamma \vdash K_1 : \KType(\sigma) \qquad \Gamma \vdash K_2 : \KType(\tau)} {\Gamma \vdash K_1 \otimes K_2 : \KType(\sigma \times \tau)}
    \end{gather*}
    \caption{Ket typing rules.}
\end{figure}

\begin{figure}[h]
    \begin{gather*}
        \textbf{Bra-0} \qquad
        \frac{\Gamma \vdash \sigma : \BaseS}{\Gamma \vdash \ZEROB(\sigma) : \BType(\sigma)} 
        \qquad \qquad
        \textbf{Bra-Base} \qquad
        \frac{\Gamma \vdash \sigma : \BaseS \qquad \Gamma\vdash t : \sigma}{\Gamma \vdash \bra{t} : \BType(\sigma)} \\
        \\
        \textbf{Bra-Adj} \qquad
        \frac{\Gamma \vdash K : \KType(\sigma)}{\Gamma \vdash K^\dagger : \BType(\sigma)} 
        \qquad \qquad
        \textbf{Bra-Scr} \qquad
        \frac{\Gamma \vdash a : \SType \qquad \Gamma \vdash B : \BType(\sigma)}{\Gamma \vdash a.B : \BType(\sigma)} \\
        \\
        \textbf{Bra-Add} \qquad
        \frac{\Gamma \vdash B_i : \BType(\sigma) \text{ for all $i$}}{\Gamma \vdash B_1 + \cdots + B_n : \BType(\sigma)}
        \qquad \qquad
        \textbf{Bra-MulB} \qquad
        \frac{\Gamma \vdash B : \KType(\sigma) \qquad \Gamma \vdash O : \OType(\sigma, \tau)}{\Gamma \vdash B \cdot O : \BType(\tau)} \\
        \\
        \textbf{Bra-Tsr} \qquad
        \frac{\Gamma \vdash B_1 : \BType(\sigma) \qquad \Gamma \vdash B_2 : \BType(\tau)} {\Gamma \vdash B_1 \otimes B_2 : \BType(\sigma \times \tau)}
    \end{gather*}
    \caption{Bra typing rules.}
\end{figure}


\begin{figure}[h]
    \begin{gather*}
        \textbf{Opt-0} \qquad
        \frac{\Gamma \vdash \sigma : \BaseS \qquad \Gamma \vdash \tau : \BaseS}{\Gamma \vdash \ZEROO(\sigma, \tau) : \OType(\sigma, \tau)} 
        \qquad \qquad
        \textbf{Opt-1} \qquad
        \frac{\Gamma \vdash \sigma : \BaseS}{\Gamma \vdash \ONEO(\sigma) : \OType(\sigma, \sigma)} \\
        \\
        \textbf{Opt-Adj} \qquad
        \frac{\Gamma \vdash O : \OType(\sigma, \tau)}{\Gamma \vdash O^\dagger : \OType(\tau, \sigma)} 
        \qquad \qquad
        \textbf{Opt-Scr} \qquad
        \frac{\Gamma \vdash a : \SType \qquad \Gamma \vdash O : \OType(\sigma, \tau)}{\Gamma \vdash a.O : \OType(\sigma, \tau)} \\
        \\
        \textbf{Opt-Add} \qquad
        \frac{\Gamma \vdash O_i : \OType(\sigma, \tau) \text{ for all $i$}}{\Gamma \vdash O_1 + \cdots + O_n : \OType(\sigma, \tau)}
        \qquad \qquad
        \textbf{Opt-Outer} \qquad
        \frac{\Gamma\vdash K : \KType(\sigma) \qquad \Gamma\vdash B : \BType(\tau)}{\Gamma\vdash K \cdot B : \OType(\sigma, \tau)} \\
        \\
        \textbf{Opt-MulO} \qquad
        \frac{\Gamma \vdash O_1 : \OType(\sigma, \tau) \qquad \Gamma \vdash O_2 : \OType(\tau, \rho)}{\Gamma \vdash O_1 \cdot O_2 : \OType(\sigma, \rho)} \\
        \\
        \textbf{Opt-Tsr} \qquad
        \frac{\Gamma \vdash O_1 : \OType(\sigma_1, \tau_1) \qquad \Gamma \vdash O_2 : \OType(\sigma_2, \tau_2)} {\Gamma \vdash O_1 \otimes O_2 : \OType(\sigma_1 \times \sigma_2, \tau_1 \times \tau_2)}
    \end{gather*}
    \caption{Operator typing rules.}
\end{figure}

\clearpage
\section{Conversions and Reductions}
\begin{figure}[h]
    \begin{gather*}
        \textbf{Alpha} \qquad \text{(bound variable renaming)}
        \qquad \qquad
        \textbf{Beta} \qquad
        \frac{}{\Gamma \vdash ((\lambda x:T.t)\ u)\ \reduce_\beta\ t\{x/u\}} \\
        \\
        \textbf{Delta} \qquad
        \frac{\WF(\Gamma) \qquad (c:=t:T) \in \Gamma}{\Gamma \vdash c\ \reduce_\delta\ t}
        \qquad \qquad
        \textbf{Eta} \qquad
        \frac{\Gamma\vdash t : T \to U}{\Gamma \vdash \lambda x : T . (t\ x)\ \reduce_\eta\ t}
    \end{gather*}
    \caption{Conversions and reductions for the typed lambda calculus.}
\end{figure}

\yx{Alpha rule will be implemented by renaming, and we can leave it for the future.}

\newenvironment{ruletable}[1]
{
    \begin{longtable}{cl}
    \caption{#1}\\
    \hline
    \textbf{Rule} & \textbf{Description} \\
    \hline
    \endfirsthead

    \hline
    \textbf{Rule} & \textbf{Description} \\
    \hline
    \endhead

    \hline
    \multicolumn{2}{r}{\textit{Continued on the next page}} \\
    \hline
    \endfoot

    \hline
    \endlastfoot
}
{
    \end{longtable}
}

\renewcommand{\arraystretch}{1.2} % Increases row height by 50%

\begin{ruletable}{The special flattening rule.}
    %
    R-FLATTEN
    & $a_1 + \cdots + (b_1 + \cdots + b_m) + \cdots + a_n\ \reduce\ a_1 + \cdots + b_1 + \cdots + b_m + \cdots + a_n$ \\
    & $a_1 \times \cdots \times (b_1 \times \cdots \times b_m) \times \cdots \times a_n\ \reduce\ a_1 \times \cdots \times b_1 \times \cdots \times b_m \times \cdots \times a_n$ \\
    & $X_1 + \cdots + (X_1' + \cdots + X_m') + \cdots + X_n\ \reduce\ X_1 + \cdots + X_1' + \cdots + X_m' + \cdots + X_n$ \\
    & \textcolor{gray}{A special rule to flatten all AC symbols within one call.}    
\end{ruletable}

\begin{ruletable}{Rules for scalar addition and multiplication.}
    %
    R-ADDSID 
    & $+(a) \ \reduce\ a$ \\
    & \textcolor{gray}{Reduce the term if the AC symbol $+$ only has one argument.} \\
    %
    R-MULSID 
    & $\times(a) \ \reduce\ a$ \\
    & \textcolor{gray}{Similar to (R-ADDSID).} \\
    %
    R-ADDS0 
    & $a_1 + \cdots + 0 + \cdots + a_n \ \reduce\ a_1 + \cdots + a_n$ \\
    & \textcolor{gray}{This rule removes all $\ZERO$ occurances and keeps the order of remaining subterms.} \\
    %
    R-MULS0
    & $a_1 \times \cdots \times 0 \times \cdots \times a_n \ \reduce\ 0$ \\
    %
    R-MULS1
    & $a_1 \times \cdots \times \ONE \times \cdots \times a_n \ \reduce\ a_1 \times \cdots \times a_n$ \\
    & \textcolor{gray}{Similar to (R-ADDS0).} \\
    %
    R-MULS2
    & $b_1 \times \cdots \times (a_1 + \cdots + a_n) \times \cdots \times b_m$ \\
    & $\reduce\ (b_1 \times \cdots \times a_1 \times \cdots \times b_m) + \cdots + (b_1 \times \cdots \times a_n \times \cdots \times b_m)$ \\
    & \textcolor{gray}{This rule matches the first scalar addition subterm in the list.} \\
\end{ruletable}

\begin{ruletable}{Rules for other scalar symbols.}
    %
    R-CONJ0
    & $0^* \ \reduce\ 0$ \\
    %
    R-CONJ1
    & $1^* \ \reduce\ 1$ \\
    %
    R-CONJ2
    & $(a_1 + \cdots + a_n)^* \ \reduce\ a_1^* + \cdots + a_n^*$ \\
    %
    R-CONJ3
    & $(a_1 \times \cdots \times a_n)^* \ \reduce\ a_1^* \times \cdots \times a_n^*$ \\
    %
    R-CONJ4
    & $ (a^*)^* \ \reduce\ a$ \\
    % 
    R-CONJ5
    & $ \delta_{s, t}^* \ \reduce\ \delta_{s, t}$ \\
    %
    R-CONJ6
    & $ (B \cdot K)^* \ \reduce\ K^\dagger \cdot B^\dagger $ \\
    %
    R-DOT0
    & $ \ZEROB(\sigma) \cdot K \ \reduce\ 0 $ \\
    %
    R-DOT1
    & $ B \cdot \ZEROK(\sigma) \ \reduce\ 0 $ \\
    %
    R-DOT2
    & $ (a.B) \cdot K \ \reduce\ a \times (B \cdot K) $ \\
    %
    R-DOT3
    & $ B \cdot (a.K) \ \reduce\ a \times (B \cdot K) $ \\
    %
    R-DOT4
    & $ (B_1 + \cdots + B_n) \cdot K \ \reduce\ B_1 \cdot K + \cdots + B_n \cdot K $ \\
    %
    R-DOT5
    & $ B \cdot (K_1 + \cdots + K_n) \ \reduce\ B \cdot K_1 + \cdots + B \cdot K_n $ \\
    %
    R-DOT6
    & $ \bra{s} \cdot \ket{t} \ \reduce\ \delta_{s, t} $ \\
    %
    R-DOT7
    & $ (B_1 \otimes B_2) \cdot \ket{(s, t)} \ \reduce\ (B_1 \cdot \ket{s}) \times (B_2 \cdot \ket{t}) $ \\
    %
    R-DOT8
    & $ \bra{(s, t)} \cdot (K_1 \otimes K_2) \ \reduce\ (\bra{s} \cdot K_1) \times (\bra{t} \cdot K_2) $ \\
    %
    R-DOT9
    & $ (B_1 \otimes B_2) \cdot (K_1 \otimes K_2) \ \reduce\ (B_1 \cdot K_1) \times (B_2 \cdot K_2) $ \\
    %
    R-DOT10
    & $ (B \cdot O) \cdot K \ \reduce\ B \cdot (O \cdot K) $ \\
    %
    R-DOT11
    & $ \bra{(s, t)} \cdot ((O_1 \otimes O_2) \cdot K) \ \reduce\ ((\bra{s} \cdot O_1) \otimes (\bra{t} \cdot O_2)) \cdot K $ \\
    %
    R-DOT12
    & $ (B_1 \otimes B_2) \cdot ((O_1 \otimes O_2) \cdot K) \ \reduce\ ((B_1 \cdot O_1) \otimes (B_2 \cdot O_2)) \cdot K $ \\
    %
    R-DELTA0
    & $ \delta_{a, a} \ \reduce\ 1$ \\
    %
    R-DELTA1
    & $ \delta_{(a, b), (c, d)} \ \reduce\ \delta_{a, c} \times \delta_{b, d}$ \\
\end{ruletable}


\begin{ruletable}{Rules for linear algebra scaling.}
    %
    R-SCR0
    & $ 1.X \ \reduce\ X $ \\
    %
    R-SCR1
    & $ a.(b.X) \ \reduce\ (a \times b).X $ \\
    %
    R-SCR2
    & $ a.(X_1 + \cdots + X_n) \ \reduce\ a.X_1 + \cdots + a.X_n $ \\
    %
    R-SCRK0
    & $ K : \KType(\sigma) \Rightarrow 0.K \ \reduce\ \ZEROK(\sigma)$ \\
    %
    R-SCRK1
    & $ a.\ZEROK(\sigma) \ \reduce\ \ZEROK(\sigma) $ \\
    % 
    R-SCRB0
    & $ B : \BType(\sigma) \Rightarrow 0.B \ \reduce\ \ZEROB(\sigma)$ \\
    % 
    R-SCRB1
    & $ a.\ZEROB(\sigma) \ \reduce\ \ZEROB(\sigma) $ \\
    % 
    R-SCRO0
    & $ O : \OType(\sigma, \tau) \Rightarrow 0.O \ \reduce\ \ZEROO(\sigma, \tau)$ \\
    % 
    R-SCRO1
    & $ a.\ZEROO(\sigma, \tau) \ \reduce\ \ZEROO(\sigma, \tau) $ \\
\end{ruletable}

\begin{ruletable}{Rules for linear algebra addition. }
    %
    R-ADDID
    & $+(X) \ \reduce\ X$ \\
    & \textcolor{gray}{Reduce the term if the AC symbol $+$ only has one argument.} \\
    %
    R-ADD0
    & $Y_1 + \cdots + X + \cdots + X + \cdots + Y_n \ \reduce\ Y_1 + \cdots + Y_n + \cdots + (1+1).X$ \\
    & \textcolor{gray}{\makecell[l]{This rule matches the first $X$ in the list satisfying the pattern. \\ The result $(1+1).X$ will be placed at the end of the list.}} \\
    %
    R-ADD1
    & $Y_1 + \cdots + X + \cdots + a.X + \cdots + Y_n \ \reduce\ Y_1 + \cdots + Y_n + (1+a).X$ \\
    & \textcolor{gray}{Similar to (R-ADD0).} \\
    %
    R-ADD2
    & $Y_1 + \cdots + a.X + \cdots + X + \cdots + Y_n \ \reduce\ Y_1 + \cdots + Y_n + (a+1).X$ \\
    & \textcolor{gray}{Similar to (R-ADD0).} \\
    %
    R-ADD3
    & $Y_1 + \cdots + a.X + \cdots + b.X + \cdots + Y_n \ \reduce\ Y_1 + \cdots + Y_n + (a+b).X$ \\
    & \textcolor{gray}{Similar to (R-ADD0).} \\
    %
    R-ADDK0
    & $K_1 + \cdots + \ZEROK(\sigma) + \cdots + K_n\ \reduce K_1 + \cdots + K_n$ \\
    & \textcolor{gray}{Similar to (R-ADDS0).} \\
    %
    R-ADDB0
    & $B_1 + \cdots + \ZEROB(\sigma) + \cdots + B_n \ \reduce\ B_1 + \cdots + B_n$ \\
    & \textcolor{gray}{Similar to (R-ADDS0).} \\
    %
    R-ADDO0
    & $O_1 + \cdots + \ZEROO(\sigma, \tau) + \cdots + O_n \ \reduce\ O_1 + \cdots + O_n$ 
    \\
    & \textcolor{gray}{Similar to (R-ADDS0).} \\
\end{ruletable}

\begin{ruletable}{Rules for adjoint.}
    %
    R-ADJ0
    & $ (X^\dagger)^\dagger \ \reduce\ X $ \\
    %
    R-ADJ1
    & $ (a.X)^\dagger \ \reduce\ (a^*).(X^\dagger) $ \\
    %
    R-ADJ2
    & $ (X_1 + \cdots + X_n)^\dagger \ \reduce\ X_1^\dagger + \cdots + X_n^\dagger $ \\
    %
    R-ADJ3
    & $ (X \otimes Y)^\dagger \ \reduce\ X^\dagger \otimes Y^\dagger$ \\
    %
    R-ADJK0
    & $ \ZEROB(\sigma)^\dagger \ \reduce\ \ZEROK(\sigma) $ \\
    %
    R-ADJK1
    & $ \bra{t}^\dagger \ \reduce\ \ket{t} $ \\
    %
    R-ADJK2
    & $ (B \cdot O)^\dagger\ \reduce\ O^\dagger \cdot B^\dagger $ \\
    %
    R-ADJB0
    & $ \ZEROK(\sigma)^\dagger \ \reduce\ \ZEROB(\sigma) $ \\
    %
    R-ADJB1
    & $ \ket{t}^\dagger \ \reduce\ \bra{t} $ \\
    %
    R-ADJB2
    & $ (O \cdot K)^\dagger\ \reduce\ K^\dagger \cdot O^\dagger $ \\
    %
    R-ADJO0
    & $ \ZEROO(\sigma, \tau)^\dagger \ \reduce\ \ZEROO(\tau, \sigma) $ \\
    %
    R-ADJO1
    & $ \ONEO(\sigma)^\dagger \ \reduce\ \ONEO(\sigma)$ \\
    %
    R-ADJO2
    & $ (K \cdot B)^\dagger \ \reduce\ B^\dagger \cdot K^\dagger$ \\
    %
    R-ADJO3
    & $ (O_1 \cdot O_2)^\dagger \ \reduce\ O_2^\dagger \cdot O_1^\dagger $ \\
\end{ruletable}

\begin{ruletable}{Rules for tensor product.}
    R-TSR0
    & $ (a.X_1) \otimes X_2 \ \reduce\ a.(X_1 \otimes X_2) $ \\
    %
    R-TSR1
    & $ X_1 \otimes (a.X_2) \ \reduce\ a.(X_1 \otimes X_2) $ \\
    %
    R-TSR2
    & $ (X_1 + \cdots + X_n) \otimes X' \reduce X_1 \otimes X' + \cdots + X_n \otimes X' $ \\
    %
    R-TSR3
    & $ X' \otimes (X_1 + \cdots + X_n) \reduce X' \otimes X_1 + \cdots + X' \otimes X_n $ \\
    %
    R-TSRK0
    & $ K : \KType(\tau) \Rightarrow \ZEROK(\sigma) \otimes K\ \reduce\ \ZEROK(\sigma \times \tau) $ \\
    %
    R-TSRK1
    & $ K : \KType(\tau) \Rightarrow K \otimes \ZEROK(\sigma)\ \reduce\ \ZEROK(\tau \times \sigma) $ \\
    %
    R-TSRK2
    & $\ket{s} \otimes \ket{t} \ \reduce\ \ket{(s, t)}$ \\
    %
    R-TSRB0
    & $ B : \BType(\tau) \Rightarrow \ZEROB(\sigma) \otimes B\ \reduce\ \ZEROB(\sigma \times \tau) $ \\
    %
    R-TSRB1
    & $ B : \BType(\tau) \Rightarrow B \otimes \ZEROB(\sigma)\ \reduce\ \ZEROB(\tau \times \sigma) $ \\
    %
    R-TSRB2
    & $\bra{s} \otimes \bra{t} \ \reduce\ \bra{(s, t)}$ \\
    %
    R-TSRO0
    & $ O : \OType(\sigma, \tau) \Rightarrow O \otimes \ZEROO(\sigma', \tau') \ \reduce\ \ZEROO(\sigma \times \sigma', \tau \times \tau') $ \\
    % 
    R-TSRO1
    & $ O : \OType(\sigma, \tau) \Rightarrow \ZEROO(\sigma', \tau') \otimes O \ \reduce\ \ZEROO(\sigma' \times \sigma, \tau' \times \tau) $ \\
    %
    R-TSRO2
    & $\ONEO(\sigma) \otimes \ONEO(\tau)\ \reduce\ \ONEO(\sigma \times \tau)$ \\
    %
    R-TSRO3
    & $ (K_1 \cdot B_1) \otimes (K_2 \cdot B_2)\ \reduce\ (K_1 \otimes K_2) \cdot (B_1 \otimes B_2) $ \\
    %
\end{ruletable}

\begin{ruletable}{Rule for $O\cdot K$.}
    %
    R-MULK0
    & $ \ZEROO(\sigma, \tau) \cdot K \ \reduce\ \ZEROK(\sigma) $ \\
    %
    R-MULK1
    & $ O : \OType(\sigma, \tau) \Rightarrow O \cdot \ZEROK(\tau) \ \reduce\ \ZEROK(\sigma) $ \\
    %
    R-MULK2
    & $ \ONEO(\sigma) \cdot K \ \reduce K $ \\
    %
    R-MULK3
    & $ (a.O) \cdot K \ \reduce\ a.(O \cdot K) $ \\
    %
    R-MULK4
    & $ O \cdot (a.K) \ \reduce\ a.(O \cdot K) $ \\
    %
    R-MULK5
    & $ (O_1 + \cdots + O_n) \cdot K \ \reduce\ O_1 \cdot K + \cdots + O_n \cdot K $ \\
    %
    R-MULK6
    & $ O \cdot (K_1 + \cdots + K_n) \ \reduce\ O \cdot K_1 + \cdots + O \cdot K_n $ \\
    %
    R-MULK7
    & $ (K_1 \cdot B) \cdot K_2 \ \reduce\ (B \cdot K_2).K_1 $ \\
    %
    R-MULK8
    & $ (O_1 \cdot O_2) \cdot K \ \reduce\ O_1 \cdot (O_2 \cdot K) $ \\
    %
    R-MULK9
    & $ (O_1 \otimes O_2) \cdot ((O_1' \otimes O_2') \cdot K)\ \reduce\ ((O_1 \cdot O_1') \otimes (O_2 \cdot O_2')) \cdot K $ \\
    %
    R-MULK10
    & $ (O_1 \otimes O_2) \cdot \ket{(s, t)}\ \reduce\ (O_1 \cdot \ket{s}) \otimes (O_2 \cdot \ket{t}) $ \\
    %
    R-MULK11
    & $ (O_1 \otimes O_2) \cdot (K_1 \otimes K_2)\ \reduce\ (O_1 \cdot K_1) \otimes (O_2 \cdot K_2) $
\end{ruletable}


\begin{ruletable}{Rule for $B\cdot O$.}
    %
    R-MULB0
    & $ B \cdot \ZEROO(\sigma, \tau) \ \reduce\ \ZEROB(\tau) $ \\
    %
    R-MULB1
    & $ O : \OType(\sigma, \tau) \Rightarrow \ZEROB(\sigma) \cdot O\ \reduce\ \ZEROB(\tau) $ \\
    %
    R-MULB2
    & $ B \cdot \ONEO(\sigma) \ \reduce B $ \\
    %
    R-MULB3
    & $ (a.B) \cdot O \ \reduce\ a.(B \cdot O) $ \\
    %
    R-MULB4
    & $ B \cdot (a.O) \ \reduce\ a.(B \cdot O) $ \\
    %
    R-MULB5
    & $ (B_1 + \cdots + B_n) \cdot O \ \reduce\ B_1 \cdot O + \cdots + B_n \cdot O $ \\
    %
    R-MULB6
    & $ B \cdot (O_1 + \cdots + O_n) \ \reduce\ B \cdot O_1 + \cdots + B \cdot O_n $ \\
    %
    R-MULB7
    & $ B_1 \cdot (K \cdot B_2) \ \reduce\ (B_1 \cdot K).B_2 $ \\
    %
    R-MULB8
    & $ B \cdot (O_1 \cdot O_2) \ \reduce\ (B \cdot O_1) \cdot O_2 $ \\
    %
    R-MULB9
    & $ (B \cdot (O_1' \otimes O_2')) \cdot (O_1 \otimes O_2) \ \reduce\ B \cdot ((O_1' \otimes O_2') \cdot (O_1 \otimes O_2)) $ \\
    %
    R-MULB10
    & $ \bra{(s, t)} \cdot (O_1 \otimes O_2)\ \reduce\ (\bra{s} \cdot O_1) \otimes (\bra{t} \cdot O_2) $ \\
    %
    R-MULB11
    & $ (B_1 \otimes B_2) \cdot (O_1 \otimes O_2)\ \reduce\ (B_1 \cdot O_1) \otimes (B_2 \cdot O_2) $
\end{ruletable}

\begin{ruletable}{Rules for $K \cdot B$.}
    %
    R-OUTER0
    & $ B : \BType(\tau) \Rightarrow \mathbf{0}_\mathcal{K}(\sigma) \cdot B\ \reduce\ \mathbf{0}_\mathcal{O}(\sigma, \tau) $ \\
    %
    R-OUTER1
    & $ K : \KType(\sigma) \Rightarrow K \cdot \mathbf{0}_\mathcal{B}(\tau)\ \reduce\ \mathbf{0}_\mathcal{O}(\sigma, \tau) $ \\
    %
    R-OUTER2
    & $ (a.K) \cdot B\ \reduce\ a.(K \cdot B) $ \\
    %
    R-OUTER3
    & $ K \cdot (a.B)\ \reduce\ a.(K \cdot B) $ \\
    %
    R-OUTER4
    & $ (K_1 + \cdots + K_n) \cdot B\ \reduce\ K_1 \cdot B + \cdots + K_n \cdot B $ \\
    %
    R-OUTER5
    & $ K \cdot (B_1 + \cdots + B_n)\ \reduce\ K \cdot B_1 + \cdots + K \cdot B_n $ \\
\end{ruletable}

\begin{ruletable}{Rules for $O_1 \cdot O_2$.}
    %
    R-MULO0
    & $ O : \OType(\tau, \rho) \Rightarrow \ZEROO(\sigma, \tau) \cdot O\ \reduce\ \ZEROO(\sigma, \rho) $ \\
    %
    R-MULO1
    & $ O : \OType(\sigma, \tau) \Rightarrow O \cdot \ZEROO(\tau, \rho)\ \reduce\ \mathbf{0}_\mathcal{O}(\sigma, \rho) $ \\
    %
    R-MULO2
    & $ \ONEO(\sigma) \cdot O \ \reduce\ O $ \\
    %
    R-MULO3
    & $ O \cdot \ONEO(\sigma) \ \reduce\ O $ \\
    %
    R-MULO4
    & $ (K \cdot B) \cdot O \ \reduce\ K \cdot (B \cdot O) $ \\
    %
    R-MULO5
    & $ O \cdot (K \cdot B) \ \reduce\ (O \cdot K) \cdot B $ \\
    %
    R-MULO6
    & $ (a.O_1) \cdot O_2 \ \reduce\ a.(O_1 \cdot O_2) $ \\
    %
    R-MULO7
    & $ O_1 \cdot (a.O_2) \ \reduce\ a.(O_1 \cdot O_2) $ \\
    %
    R-MULO8
    & $ (O_1 + \cdots + O_n) \cdot O'\ \reduce\ O_1 \cdot O' + \cdots + O_n \cdot O' $ \\
    %
    R-MULO9
    & $ O' \cdot (O_1 + \cdots + O_n)\ \reduce\ O' \cdot O_1 + \cdots + O' \cdot O_n $ \\
    %
    R-MULO10
    & $ (O_1 \cdot O_2) \cdot O_3\ \reduce\ O_1 \cdot (O_2 \cdot O_3) $ \\
    %
    R-MULO11
    & $ (O_1 \otimes O_2) \cdot (O_1' \otimes O_2')\ \reduce\ (O_1 \cdot O_1') \otimes (O_2 \cdot O_2') $ \\
    %
    R-MULO12
    & $ (O_1 \otimes O_2) \cdot ((O_1' \otimes O_2') \cdot O_3)\ \reduce\ ((O_1 \cdot O_1') \otimes (O_2 \cdot O_2')) \cdot O_3 $ \\  
\end{ruletable}

% About big operators: maybe we should use the "sum operators"?


%%%%%%%%%%%%%%%%%%%%%%%%%%%%%%%%%%%%%%%%%%%%%%%%%%%%%%%%%%%%%%%%%%%%%%%%%%%%%%%%%%%%%%%%%%%%%%%%

\clearpage

\section{\texttt{Diracoq} language}

\begin{align*}
    cmd ::=&\ \texttt{Def}(\texttt{ID}\ term) \\
        &|\ \texttt{Def}(\texttt{ID}\ term\ type) \\
        &|\ \texttt{Var}(\texttt{ID}\ term) \\
        &|\ \texttt{Check}(term) \\
        &|\ \texttt{Show}(\texttt{ID}) \\
        &|\ \texttt{ShowAll} \\
        &|\ \texttt{Normalize}(term) \\
        &|\ \texttt{CheckEq}(term\ term) \\
    type ::=&\ \texttt{Type}\ |\ \texttt{Arrow}(type\ type) \\
            &|\ \texttt{Base} \\
    term ::=&\ \texttt{Type}\ |\ \texttt{fun}(\texttt{ID}\ type\ term)\ |\ \texttt{apply}(term\ term)\ |\ \texttt{ID} \\
\end{align*}


\section{Rewriting Control and Intermediate Language}
The associativity is already handled by the (R-FLATTEN) rule. In order to decide two terms $A$ and $B$ are equivalent under commutativity, we need to proof that $A$ can be transformed into $B$ with a structured permutation, which is described by the \textit{permutation tree}.

\begin{definition}[permutation tree]
    The syntax for permutation trees are inductively defined below:
    \[
    P ::= \mathsf{E}\ |\ [(i:P)^+].
    \]
    Here $i$ represents positive numbers.
\end{definition}
We always only consider \textit{well-formed} permutation trees. That is, if $P \equiv [i_1:P_1\ i_2:P_2\ \cdots\ i_n:P_n]$, then $\{i_1, ... i_n\}$ forms the set of integers from $0$ to $n-1$.

We can transform a term A with a suitable permutation tree. The transformation is defined as

\begin{align*}
    \texttt{apply}(A, P) :=\ & \texttt{match}\ P\ \texttt{with} \\
    & \quad |\ \textsf{E} \Rightarrow A \\
    & \quad |\ [i_1:P_1\ \cdots\ i_n:P_n] \Rightarrow \texttt{A.head}(\texttt{apply(A.args[$i_1$], $P_{i_1}$)}\ \cdots\ \texttt{apply(A.args[$i_n$], $P_{i_n}$)}) \\
    & \texttt{end}
\end{align*}




\clearpage

\bibliographystyle{plain}
\bibliography{ref}

\end{document}
