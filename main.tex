\documentclass{article}

\usepackage[utf8]{inputenc}
\usepackage[margin=0.8in]{geometry}

\usepackage{color}
\usepackage{amsmath}
%\usepackage{amssymb}
\usepackage{graphicx}
\usepackage{amsthm}
\usepackage{stmaryrd}
\usepackage[all]{xy}
\usepackage{multirow}
\usepackage{paralist}
\usepackage{hhline}
\usepackage{bm}

\usepackage{wasysym}
\usepackage{extarrows}
\usepackage{tikz}

\newtheorem{theorem}{Theorem}[section]
\newtheorem{corollary}{Corollary}[section]
\newtheorem{lemma}{Lemma}[section]
\newtheorem{remark}{Remark}[section]
\newtheorem{definition}{Definition}[section]
\newtheorem{proposition}{Proposition}[section]
\newtheorem{example}{Example}[section]
\newtheorem{problem}{Problem}[section]
\newtheorem{principle}{Principle}
\newtheorem{conjecture}{Conjecture}
\newtheorem{postulate}{Postulate}
\newtheorem{fact}{Fact}
\newtheorem{claim}{Claim}


% add new commands for comments here
\newcommand{\yx}[1]{\textit{\color{blue}[YX] : #1}}

\newcommand{\modify}[1]{{\color{red}#1}}

\usepackage{amssymb}
\usepackage{amsmath}
\usepackage{stmaryrd}
\usepackage{hyperref}

\usepackage{braket}


\title{\textbf{DiracDec C++/Coq Implementation\cite{ExampleCitation}}}
\author{Yingte Xu}

\begin{document}

\maketitle

%%%%%%%%%%%%%%%%
\section{Scalar: Minimal Example}

We first consider a minimal example of the scalars.

\subsection{Modeling AC by Sequence}
The problem is presented in the normal AC term-rewriting setting, with a minor modification of flattening the AC symbols. This is the model implemented in Coq, and we consider the 
C++ optimization only as a efficient searching algorithm for the rewriting trace.

\newcommand{\ZERO}{\mathsf{0}}
\newcommand{\ONE}{\mathsf{1}}
\newcommand{\ADDS}{\mathsf{ADDS}}
\newcommand{\MLTS}{\mathsf{MLTS}}
\newcommand{\CONJ}{\mathsf{CONJ}}

\begin{definition}[scalar syntax]
    The syntax for scalars is defined as follows.
    \[ 
    S ::= \ZERO\ |\ \ONE\ |\ \ADDS ( S^+ )\ |\ \MLTS ( S^+ )\ |\ \CONJ(S).
    \]
    Here $S^+$ means a sequence of one or more $S$ terms, i.e., AC symbols are flattened and have indefinite arities.
\end{definition}

The rewriting rules are presented as follows.

\begin{figure}[h]
    \begin{align*}
        & \text{(R-FLATTEN)} &&
        \textcolor{gray}{\text{A special rule to flatten all AC symbols within one call.}} \\
        & \text{(R-ADDSID)} &&
        \ADDS(a) \to a \\
        & \text{(R-ADDS0)} &&
        \ADDS((seq:\_ \_)\ \ZERO) \to \ADDS(seq) \\
        & && \textcolor{gray}{\text{This rule removes all $\ZERO$ occurances and keeps the order of remaining subterms.}} \\
        & \text{(R-MLTSID)} &&
        \MLTS(a) \to a \\
        & \text{(R-MLTS0)} &&
        \MLTS((seq: \_ \_)\ \ZERO) \to \ZERO \\
        & \text{(R-MLTS1)} &&
        \MLTS((seq: \_ \_)\ \ONE) \to \MLTS(seq) \\
        & && \textcolor{gray}{\text{Similar to (R-ADDS0).}} \\
        & \text{(R-MLTS2)} &&
        \MLTS((seq1: \_ \_ \_)\ \ADDS(a_1\ a_2 \cdots a_n)\ (seq2: \_ \_ \_)) \\
        & && \to \ADDS(\MLTS(seq1\ a_1\ seq2)\ \MLTS(seq1\ a_2\ seq2) \cdots \MLTS(seq1\ a_n\ seq2)) \\
        & && \textcolor{gray}{\text{This rule matches the first $\ADDS$ subterm in the list. $seq1$ or $seq2$ can be empty, but not both.}} \\
        & \text{(R-CONJ0)} &&
        \CONJ(\ZERO) \to \ZERO \\
        & \text{(R-CONJ1)} &&
        \CONJ(\ONE) \to \ONE \\
        & \text{(R-CONJ2)} &&
        \CONJ(\ADDS(a_1\ a_2\ \cdots a_n)) \to \ADDS(\CONJ(a_1)\ \CONJ(a_2) \cdots \CONJ(a_n)) \\
        & \text{(R-CONJ3)} &&
        \CONJ(\MLTS(a_1\ a_2\ \cdots a_n)) \to \MLTS(\CONJ(a_1)\ \CONJ(a_2) \cdots \CONJ(a_n)) \\
        & \text{(R-CONJ4)} &&
        \CONJ(\CONJ(a)) \to a
    \end{align*}
    \caption{The rewriting rules for scalars.}
\end{figure}

Notice that all the rules behaves the same as in the term count mapping implementation in C++. That is, the transformation from sequence representation to mapping representaiton commutes with all rewritings.

%%%%%%%%%%%%%%%%%%%%%%%%%%%
\section{Rewriting Control and Intermediate Language}
The associativity is already handled by the (R-FLATTEN) rule. In order to decide two terms $A$ and $B$ are equivalent under commutativity, we need to proof that $A$ can be transformed into $B$ with a structured permutation, which is described by the \textit{permutation tree}.

\begin{definition}[permutation tree]
    The syntax for permutation trees are inductively defined below:
    \[
    P ::= \mathsf{E}\ |\ [(i:P)^+].
    \]
    Here $i$ represents positive numbers.
\end{definition}
We always only consider \textit{well-formed} permutation trees. That is, if $P \equiv [i_1:P_1\ i_2:P_2\ \cdots\ i_n:P_n]$, then $\{i_1, ... i_n\}$ forms the set of integers from $0$ to $n-1$.

We can transform a term A with a suitable permutation tree. The transformation is defined as

\begin{align*}
    \texttt{apply}(A, P) :=\ & \texttt{match}\ P\ \texttt{with} \\
    & \quad |\ \textsf{E} \Rightarrow A \\
    & \quad |\ [i_1:P_1\ \cdots\ i_n:P_n] \Rightarrow \texttt{A.head}(\texttt{apply(A.args[$i_1$], $P_1$)}\ \cdots\ \texttt{apply(A.args[$i_n$], $P_n$)}) \\
    & \texttt{end}
\end{align*}


% \subsection{Coq Implementation}
% The following abilities should be implemented on the Coq side:
% \begin{itemize}
%     \item \textbf{Deciding $E$}. For the C++ implementation, the data structure itself already reflects $E$. Therefore, on the Coq side, this should also be decided automatically, implemneted as a lemma 
%     \[
%         \texttt{EQ\_E : forall (A\ B: Term), Optional(A = B)}
%     \]
%     \item \textbf{The computational function to output the equivalence lemma for all rules.} For example, for the (R-ADDS0) rule, there should be such a lemma:
%     \[
%         \texttt{E\_R\_ADDS0 : forall (A\ B: Term), Optional(A = B)}
%     \]
%     which outputs the proof for the equivalence proof when $A$ matches the left side of (R-ADDS0) and is rewriten into B.
% \end{itemize}

% With these Coq implementations, the C++ can output the script for equivalence proof in the following format:

% \begin{align*}
%     & \texttt{rewrite (EQ\_E S1 T1).}\\
%     & \texttt{rewrite (EQ\_RULE1 A1 B1).}\\
%     & \texttt{rewrite (EQ\_E S2 T2).}\\
%     & \texttt{rewrite (EQ\_RULE2 A2 B2).}\\
%     & \texttt{rewrite (EQ\_E S3 T3).}\\
%     & \texttt{rewrite (EQ\_RULE3 A3 B3).}\\
%     & \cdots
% \end{align*}

% where at each step, the C++ solver rewrites the subterm \texttt{Ai} using rule \texttt{RULEi}. The Coq prover will synchronize with C++ solver, because both systems replace all occurances of $Ai$ into $Bi$.

% \yx{Actually the \texttt{B} inputs can be inferred and is not necessary.}

% \yx{Question: How to decide the subterms \texttt{Si} and \texttt{Ti}.}

% \yx{Question: The syntax for terms that Coq accepts.}

\clearpage
\section{Typing System}
The typing system modifies the one for Calculus of Construction, adding special symbols for Dirac notations.
According to the CoC typing rules, the environment and context can be modeled by as two stacks. It is obvious that popping assumptions/definitions from any context or from a environment with empty context will not destroy the well-formedness.

One important reason for using dependent type is that, the types of Dirac notatinos (i.e., ket, bra and operators) depends on the basis.

\newcommand{\reduce}{\triangleright}

\newcommand{\SORT}{\mathsf{Sort}}
\newcommand{\WF}{\mathcal{WF}}
\newcommand{\Type}{\mathsf{Type}}
\newcommand{\BaseS}{\mathsf{Base}}


\newcommand{\SType}{\mathcal{S}}
\newcommand{\KType}{\mathcal{K}}
\newcommand{\BType}{\mathcal{B}}
\newcommand{\OType}{\mathcal{O}}

\newcommand{\ZEROK}{\mathbf{0}_\mathcal{K}}
\newcommand{\ZEROB}{\mathbf{0}_\mathcal{B}}
\newcommand{\ZEROO}{\mathbf{0}_\mathcal{O}}
\newcommand{\ONEO}{\mathbf{1}_\mathcal{O}}

\begin{figure}[h]
    \begin{gather*}
        \textbf{W-Empty} \qquad
        \frac{}{\WF([])[]}\\
        \\
        \textbf{W-Local-Assum} \qquad
        \frac{E[\Gamma]\vdash T:s \qquad s \in \mathcal{S} \qquad x \notin \Gamma}{\WF(E)[\Gamma::(x:T)]}
        \qquad \qquad
        \textbf{W-Local-Def} \qquad
        \frac{E[\Gamma]\vdash t : T \qquad x \notin \Gamma}{\WF(E)[\Gamma ::(x:=t:T)]} \\
        \\
        \textbf{W-Global-Assum} \qquad
        \frac{E[]\vdash T:s \qquad s \in \mathcal{S} \qquad c \notin E}{\WF(E; c:T)[]}
        \qquad \qquad
        \textbf{W-Global-Def} \qquad
        \frac{E[]\vdash t:T \qquad c \notin E}{\WF(E; c:=t:T)[]} \\
        \\
        \textbf{Ax-Type} \qquad
        \frac{\WF(E)[\Gamma]}{E[\Gamma]\vdash \Type(i):\Type(i+1)} \\
        \\
        \textbf{Var} \qquad
        \frac{\WF(E)[\Gamma] \qquad (x:T) \in \Gamma \text{ or } (x := t : T) \in \Gamma \text{ for some $t$}}{E[\Gamma] \vdash x : T} \\
        \\
        \textbf{Const} \qquad
        \frac{\WF(E)[\Gamma] \qquad (c:T) \in E \text{ or } (c := t : T) \in E \text{ for some $t$}}{E[\Gamma] \vdash c : T} \\
        \\
        \textbf{Prod-Type} \qquad
        \frac{E[\Gamma] \vdash T:\Type(i)\qquad E[\Gamma::(x:T)]\vdash U:\Type(i)}{E[\Gamma] \vdash \forall x: T, U : \Type(i)} \\
        \\
        \textbf{Lam} \qquad
        \frac{E[\Gamma] \vdash \forall x:T, U : s \qquad E[\Gamma :: (x:T)]\vdash t : U}{E[\Gamma] \vdash \lambda x:T . t : \forall x : T, U} \\
        \\
        \textbf{App} \qquad
        \frac{E[\Gamma] \vdash t:\forall x:U, T \qquad E[\Gamma] \vdash u:U}{E[\Gamma] \vdash (t\ u):T\{x/u\}} \\
        \\
        \textbf{Subtyping} \qquad
        \frac{E[\Gamma] \vdash U : s \qquad E[\Gamma] \vdash t : T \qquad E[\Gamma] \vdash T \leq_{\beta\delta\eta} U}{E[\Gamma]\vdash t:U}
    \end{gather*}
    \caption{
    The typing rules for CoC. 
    We only consider computational types, removing the sorts of $\mathsf{Set}$ and $\mathsf{Prop}$.}
\end{figure}

\begin{figure}[h]
    \begin{gather*}
        \frac{E[\Gamma]\vdash t =_{\beta\delta\eta} u}{E[\Gamma]\vdash t \leq_{\beta\delta\eta} u} 
        \qquad
        \frac{i \leq j}{E[\Gamma] \vdash \Type(i) \leq_{\beta\delta\eta} \Type(j)}
        \qquad
        \frac{E[\Gamma] \vdash T =_{\beta\delta\eta} U \qquad E[\Gamma :: (x:T)]\vdash T' \leq_{\beta\delta\eta} U'}{E[\Gamma] \vdash \forall x : T, T' \leq_{\beta\delta\eta} \forall x : U, U'}
    \end{gather*}
    \caption{The rules to decide subtyping.}
\end{figure}

\begin{figure}[h]
    \begin{gather*}
        \textbf{Type-Base} \qquad
        \frac{\WF(E)[\Gamma]}{E[\Gamma]\vdash \BaseS : \Type(1)} 
        \qquad \qquad
        \textbf{Type-Scalar} \qquad
        \frac{\WF(E)[\Gamma]}{E[\Gamma]\vdash \SType : \Type(1)} \\
        \\
        \textbf{Type-Ket} \qquad
        \frac{E[\Gamma]\vdash \sigma : \BaseS}{E[\Gamma]\vdash \KType(\sigma) : \Type(1)}
        \qquad \qquad
        \textbf{Type-Bra} \qquad
        \frac{E[\Gamma]\vdash \sigma : \BaseS}{E[\Gamma]\vdash \BType(\sigma) : \Type(1)} \\
        \\
        \textbf{Type-Opt} \qquad
        \frac{E[\Gamma]\vdash \sigma : \BaseS \qquad E[\Gamma]\vdash \tau : \BaseS}{E[\Gamma]\vdash \OType(\sigma, \tau) : \Type(1)}
    \end{gather*}
    \caption{Typing rules for Dirac notation types.}
\end{figure}

\begin{figure}[h]
    \begin{gather*}
        \textbf{Sca-0} \qquad
        \frac{\WF(E)[\Gamma]}{E[\Gamma] \vdash 0 : \SType}
        \qquad \qquad
        \textbf{Sca-1} \qquad
        \frac{\WF(E)[\Gamma]}{E[\Gamma] \vdash 1 : \SType}
        \qquad \qquad
        \textbf{Sca-Delta} \qquad
        \frac{            
            \begin{aligned}
                & E[\Gamma]\vdash \sigma : \BaseS \\
                & E[\Gamma]\vdash s : \sigma \qquad E[\Gamma]\vdash t : \sigma
            \end{aligned}} {E[\Gamma] \vdash \delta_{s, t} : \SType} \\
        \\
        \textbf{Sca-Add} \qquad
        \frac{E[\Gamma]\vdash a : \SType \qquad E[\Gamma]\vdash b : \SType}{E[\Gamma]\vdash a + b : \SType}
        \qquad \qquad
        \textbf{Sca-Mul} \qquad
        \frac{E[\Gamma]\vdash a : \SType \qquad E[\Gamma]\vdash b : \SType}{E[\Gamma]\vdash a \times b : \SType} \\
        \\
        \textbf{Sca-Conj} \qquad
        \frac{E[\Gamma] \vdash a : \SType}{E[\Gamma] \vdash a^*:\SType}
        \qquad \qquad
        \textbf{Sca-Dot} \qquad
        \frac{E[\Gamma]\vdash B : \BType(\sigma) \qquad E[\Gamma]\vdash K : \KType(\sigma)}{E[\Gamma] \vdash B \cdot K : \SType}
    \end{gather*}
    \caption{Scalar typing rules.}
\end{figure}

\begin{figure}[h]
    \begin{gather*}
        \textbf{Base-Def} \qquad
        \frac{\WF(E)[\Gamma] \qquad \text{for all $i$, } a_i \notin \Gamma \text{ and } a_i \notin E}
        {E[\Gamma]\vdash \{a_1, a_2, \cdots, a_n\} : \BaseS} \\
        \\
        \textbf{Base-Element} \qquad
        \frac{\WF(E)[\Gamma] \qquad c \in \{a_1, a_2, \cdots a_n\}}{E[\Gamma]\vdash c : \{a_1, a_2, \cdots, a_n\}} \\
        \\
        \textbf{Base-Prod} \qquad
        \frac{E[\Gamma]\vdash \sigma : \BaseS \qquad E[\Gamma]\vdash \tau : \BaseS}{E[\Gamma]\vdash \sigma \times \tau : \BaseS} \\
        \\
        \textbf{Pair-Base} \qquad
        \frac{
            \begin{aligned}
                & E[\Gamma]\vdash \sigma : \BaseS \qquad E[\Gamma]\vdash s : \sigma \\
                & E[\Gamma]\vdash \tau : \BaseS \qquad E[\Gamma]\vdash t : \tau
            \end{aligned}} {E[\Gamma]\vdash (s, t) : \sigma \times \tau} 
        \textbf{}
    \end{gather*}
    \caption{
    Typing rules for bases. 
    Here the definitions $\{a_1, a_2, \cdots a_n\}$ of different order elements are equivalent.}
\end{figure}


\begin{figure}[h]
    \begin{gather*}
        \textbf{Ket-0} \qquad
        \frac{E[\Gamma] \vdash \sigma : \BaseS}{E[\Gamma] \vdash \ZEROK(\sigma) : \KType(\sigma)} 
        \qquad \qquad
        \textbf{Ket-Base} \qquad
        \frac{E[\Gamma] \vdash \sigma : \BaseS \qquad E[\Gamma]\vdash t : \sigma}{E[\Gamma] \vdash \ket{t} : \KType(\sigma)} \\
        \\
        \textbf{Ket-Adj} \qquad
        \frac{E[\Gamma] \vdash B : \BType(\sigma)}{E[\Gamma] \vdash B^\dagger : \KType(\sigma)} 
        \qquad \qquad
        \textbf{Ket-Sca} \qquad
        \frac{E[\Gamma] \vdash a : \SType \qquad E[\Gamma] \vdash K : \KType(\sigma)}{E[\Gamma] \vdash a.K : \KType(\sigma)} \\
        \\
        \textbf{Ket-Add} \qquad
        \frac{E[\Gamma] \vdash K_1 : \KType(\sigma) \qquad E[\Gamma] \vdash K_2 : \KType(\sigma)}{E[\Gamma] \vdash K_1 + K_2 : \KType(\sigma)} \\
        \\
        \textbf{Ket-Dot} \qquad
        \frac{E[\Gamma] \vdash O : \OType(\sigma, \tau) \qquad E[\Gamma] \vdash K : \KType(\tau)}{E[\Gamma] \vdash O \cdot K : \KType(\sigma)} \\
        \\
        \textbf{Ket-Tsr} \qquad
        \frac{E[\Gamma] \vdash K_1 : \KType(\sigma) \qquad E[\Gamma] \vdash K_2 : \KType(\tau)} {E[\Gamma] \vdash K_1 \otimes K_2 : \KType(\sigma \times \tau)}
    \end{gather*}
    \caption{Ket typing rules.}
\end{figure}

\begin{figure}[h]
    \begin{gather*}
        \textbf{Bra-0} \qquad
        \frac{E[\Gamma] \vdash \sigma : \BaseS}{E[\Gamma] \vdash \ZEROB(\sigma) : \BType(\sigma)} 
        \qquad \qquad
        \textbf{Bra-Base} \qquad
        \frac{E[\Gamma] \vdash \sigma : \BaseS \qquad E[\Gamma]\vdash t : \sigma}{E[\Gamma] \vdash \bra{t} : \BType(\sigma)} \\
        \\
        \textbf{Bra-Adj} \qquad
        \frac{E[\Gamma] \vdash K : \KType(\sigma)}{E[\Gamma] \vdash K^\dagger : \BType(\sigma)} 
        \qquad \qquad
        \textbf{Bra-Sca} \qquad
        \frac{E[\Gamma] \vdash a : \SType \qquad E[\Gamma] \vdash B : \BType(\sigma)}{E[\Gamma] \vdash a.B : \BType(\sigma)} \\
        \\
        \textbf{Bra-Add} \qquad
        \frac{E[\Gamma] \vdash B_1 : \BType(\sigma) \qquad E[\Gamma] \vdash B_2 : \BType(\sigma)}{E[\Gamma] \vdash B_1 + B_2 : \BType(\sigma)} \\
        \\
        \textbf{Bra-Dot} \qquad
        \frac{E[\Gamma] \vdash B : \KType(\sigma) \qquad E[\Gamma] \vdash O : \OType(\sigma, \tau)}{E[\Gamma] \vdash B \cdot O : \BType(\tau)} \\
        \\
        \textbf{Bra-Tsr} \qquad
        \frac{E[\Gamma] \vdash B_1 : \BType(\sigma) \qquad E[\Gamma] \vdash B_2 : \BType(\tau)} {E[\Gamma] \vdash B_1 \otimes B_2 : \BType(\sigma \times \tau)}
    \end{gather*}
    \caption{Bra typing rules.}
\end{figure}


\begin{figure}[h]
    \begin{gather*}
        \textbf{Opt-0} \qquad
        \frac{E[\Gamma] \vdash \sigma : \BaseS \qquad E[\Gamma] \vdash \tau : \BaseS}{E[\Gamma] \vdash \ZEROO(\sigma, \tau) : \OType(\sigma, \tau)} 
        \qquad \qquad
        \textbf{Opt-1} \qquad
        \frac{E[\Gamma] \vdash \sigma : \BaseS}{E[\Gamma] \vdash \ONEO(\sigma) : \OType(\sigma, \sigma)} \\
        \\
        \textbf{Opt-Adj} \qquad
        \frac{E[\Gamma] \vdash O : \OType(\sigma, \tau)}{E[\Gamma] \vdash O^\dagger : \OType(\tau, \sigma)} 
        \qquad \qquad
        \textbf{Opt-Sca} \qquad
        \frac{E[\Gamma] \vdash a : \SType \qquad E[\Gamma] \vdash O : \OType(\sigma, \tau)}{E[\Gamma] \vdash a.O : \OType(\sigma, \tau)} \\
        \\
        \textbf{Opt-Add} \qquad
        \frac{E[\Gamma] \vdash O_1 : \OType(\sigma, \tau) \qquad E[\Gamma] \vdash O_2 : \OType(\sigma, \tau)}{E[\Gamma] \vdash O_1 + O_2 : \OType(\sigma, \tau)} \\
        \\
        \textbf{Opt-Outer} \qquad
        \frac{E[\Gamma]\vdash K : \KType(\sigma) \qquad E[\Gamma]\vdash B : \BType(\tau)}{E[\Gamma]\vdash K \cdot B : \OType(\sigma, \tau)} \\
        \\
        \textbf{Opt-Dot} \qquad
        \frac{E[\Gamma] \vdash O_1 : \OType(\sigma, \tau) \qquad E[\Gamma] \vdash O_2 : \OType(\tau, \rho)}{E[\Gamma] \vdash O_1 \cdot O_2 : \OType(\sigma, \rho)} \\
        \\
        \textbf{Opt-Tsr} \qquad
        \frac{E[\Gamma] \vdash O_1 : \OType(\sigma_1, \tau_1) \qquad E[\Gamma] \vdash O_2 : \OType(\sigma_2, \tau_2)} {E[\Gamma] \vdash O_1 \otimes O_2 : \OType(\sigma_1 \times \sigma_2, \tau_1 \times \tau_2)}
    \end{gather*}
    \caption{Operator typing rules.}
\end{figure}

\clearpage
\section{Conversions and Reductions}
\begin{figure}[h]
    \begin{gather*}
        \textbf{Alpha} \qquad \text{(bound variable renaming)}
        \qquad \qquad
        \textbf{Beta} \qquad
        \frac{}{E[\Gamma] \vdash ((\lambda x:T.t)\ u) \reduce_\beta t\{x/u\}} \\
        \\
        \textbf{Delta-Local} \qquad
        \frac{\WF(E)[\Gamma] \qquad (x:=t:T) \in \Gamma}{E[\Gamma] \vdash x \reduce_\Delta t} 
        \qquad
        \textbf{Delta-Global} \qquad
        \frac{\WF(E)[\Gamma] \qquad (c:=t:T) \in E}{E[\Gamma] \vdash c \reduce_\delta t} \\
        \\
        \textbf{Eta} \qquad
        \frac{E[\Gamma]\vdash t : \forall x : T , U}{E[\Gamma] \vdash t =_\eta \lambda x : T . (t\ x)}
    \end{gather*}
    \caption{Conversions and reductions in CoC.}
\end{figure}

\begin{figure}[h]
    \begin{gather*}
        \cdots
    \end{gather*}
    \caption{Dirac notation reduction rules}
\end{figure}

About big operators: maybe we should use the "sum operators"?

\clearpage

\bibliographystyle{plain}
\bibliography{ref}

\end{document}
