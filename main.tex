% This is samplepaper.tex, a sample chapter demonstrating the
% LLNCS macro package for Springer Computer Science proceedings;
% Version 2.21 of 2022/01/12
%
\documentclass[runningheads]{llncs}


% \usepackage[utf8]{inputenc}
% \usepackage[margin=0.8in]{geometry}




\usepackage{color}
\usepackage{amsmath}
%\usepackage{amssymb}
\usepackage{graphicx}
% \usepackage{amsthm}
\usepackage{stmaryrd}
\usepackage[all]{xy}
\usepackage{multirow}
\usepackage{paralist}
\usepackage{hhline}
\usepackage{bm}
\usepackage{longtable}
\usepackage{makecell}
\usepackage{mdframed}

\usepackage{wasysym}
\usepackage{extarrows}
\usepackage{tikz}

% add new commands for comments here
\newcommand{\yx}[1]{\textit{\color{blue}[YX] : #1}}
\newcommand{\lz}[1]{\textit{\color{red}[LZ] : #1}}

\newcommand{\modify}[1]{{\color{red}#1}}

\usepackage{amssymb}
\usepackage{amsmath}
\usepackage{stmaryrd}
\usepackage{hyperref}

\usepackage{braket}
\usepackage{cleveref}

\usepackage{tikz}
\usetikzlibrary{shapes.geometric, arrows}



\newenvironment{ruletable}[1]
{
    \begin{longtable}{cl}
    \caption{#1}\\
    \hline
    \textbf{Rule} & \textbf{Description} \\
    \hline
    \endfirsthead

    \hline
    \textbf{Rule} & \textbf{Description} \\
    \hline
    \endhead

    % \hline
    % \multicolumn{2}{r}{\textit{Continued on the next page}} \\
    \hline
    \endfoot

    \hline
    \endlastfoot
}
{
    \end{longtable}
}



%
\usepackage[T1]{fontenc}
% T1 fonts will be used to generate the final print and online PDFs,
% so please use T1 fonts in your manuscript whenever possible.
% Other font encondings may result in incorrect characters.
%
\usepackage{graphicx}
% Used for displaying a sample figure. If possible, figure files should
% be included in EPS format.
%
% If you use the hyperref package, please uncomment the following two lines
% to display URLs in blue roman font according to Springer's eBook style:
\usepackage{color}
\renewcommand\UrlFont{\color{blue}\rmfamily}
\urlstyle{rm}
%
\begin{document}
%
\title{Dirace: Practical Proof Automation of Dirac Notation Equations}


%%%%%%%%%%%%%%%%%%%%%%%%%%%%%%%%%%%%%%%%%%%%%%%%%%%%%%%%%%%%%%%%%%%%%%%%
% %
% %\titlerunning{Abbreviated paper title}
% % If the paper title is too long for the running head, you can set
% % an abbreviated paper title here
% %
% \author{First Author\inst{1}\orcidID{0000-1111-2222-3333} \and
% Second Author\inst{2,3}\orcidID{1111-2222-3333-4444} \and
% Third Author\inst{3}\orcidID{2222--3333-4444-5555}}
% %
% \authorrunning{F. Author et al.}
% % First names are abbreviated in the running head.
% % If there are more than two authors, 'et al.' is used.
% %
% \institute{Princeton University, Princeton NJ 08544, USA \and
% Springer Heidelberg, Tiergartenstr. 17, 69121 Heidelberg, Germany
% \email{lncs@springer.com}\\
% \url{http://www.springer.com/gp/computer-science/lncs} \and
% ABC Institute, Rupert-Karls-University Heidelberg, Heidelberg, Germany\\
% \email{\{abc,lncs\}@uni-heidelberg.de}}
%%%%%%%%%%%%%%%%%%%%%%%%%%%%%%%%%%%%%%%%%%%%%%%%%%%%%%%%%%%%%%%%%%%%%%%%


%
\maketitle              % typeset the header of the contribution
%
\begin{abstract}

Dirac notations are widely used in theoretical reasonings of quantum computation and quantum information.
The previous work proved that the equivalence of basic Dirac notations is decidable, and provided a
corresponding term rewriting system.
This work focuses on developing the theory and tool that practically solves this problem. Compared to the
previous result, this work includes a typing system and a simplified term rewriting system.
A much more efficient algorithm.
And a language and algorithm for labelled Dirac notation is proposed as the final contribution.
We implement the whole algorithm in C++ with a Mathematica backend for theories of scalars, which 
decides the 158 examples in seconds.


% \keywords{First keyword  \and Second keyword \and Another keyword.}
\end{abstract}
%
%
%



%%%%%%%%%%%%%%%%
\newcommand*{\sem}[1]{{\llbracket #1 \rrbracket}}
\newcommand{\DiracDec}{\textsf{DiracDec}}

\newcommand{\reduce}{\triangleright}

\newcommand{\Sort}{\mathsf{Sort}}
\newcommand{\WF}{\mathcal{WF}}

\newcommand{\Index}{\mathsf{Index}}
\newcommand{\Type}{\mathsf{Type}}
\newcommand{\Basis}{\mathsf{Basis}}

\newcommand{\SType}{\mathcal{S}}
\newcommand{\KType}{\mathcal{K}}
\newcommand{\BType}{\mathcal{B}}
\newcommand{\OType}{\mathcal{O}}
\newcommand{\SET}{\mathsf{Set}}

\newcommand{\ZEROK}{\mathbf{0}_\mathcal{K}}
\newcommand{\ZEROB}{\mathbf{0}_\mathcal{B}}
\newcommand{\ZEROO}{\mathbf{0}_\mathcal{O}}

\newcommand{\PAIR}{\mathsf{PAIR}}

\newcommand{\ZERO}{\mathsf{0}}
\newcommand{\ONE}{\mathsf{1}}
\newcommand{\ADDS}{\mathsf{ADDS}}
\newcommand{\ADD}{\mathsf{ADD}}
\newcommand{\MULS}{\mathsf{MULS}}
\newcommand{\MUL}{\mathsf{MUL}}
\newcommand{\CONJ}{\mathsf{CONJ}}
\newcommand{\CJG}{\mathsf{CJG}}
\newcommand{\ADJ}{\mathsf{ADJ}}
\newcommand{\DELTA}{\mathsf{DELTA}}
\newcommand{\DOT}{\mathsf{DOT}}
\newcommand{\SCR}{\mathsf{SCR}}
\newcommand{\TSR}{\mathsf{TSR}}
\newcommand{\KET}{\mathsf{KET}}
\newcommand{\BRA}{\mathsf{BRA}}
\newcommand{\ONEO}{\mathbf{1}_\mathcal{O}}
\newcommand{\OUTER}{\mathsf{OUTER}}
\newcommand{\MULK}{\mathsf{MULK}}
\newcommand{\MULB}{\mathsf{MULB}}
\newcommand{\MULO}{\mathsf{MULO}}

%%%%%%%%%%%%%%%%%%%%%%%%%%%

\section{Introduction}

In 1939, Dirac proposed his notation~\cite{dirac1939new} for quantum mechanics, which is designed to represent linear algebra formulae in a compact and convenient way.
For instance, $a\ket{\psi} + b\ket{\phi}$ indicates the addition of two vectors, i.e., the superposition of two states $\ket{\psi}$ and $\ket{\phi}$.
Dirac notation is also widely accepted as the working language in quantum computation and quantum information.
The reasonings of Dirac notations play a fundamental role, but there was few works on automating this procedure.


The absence of a working Dirac notation solver is also a main obstacle for proof automation of quantum programming languages.
In these works, Dirac notation are used to define the program states, operations and assertions. In order to automate the verification procedure,
we need to simplify and check the equivalence of pre-conditions. 


Recently, the work by Yingte et al.[] provides a theory to decide the equivalence of Dirac notations, as well as a prototype implementation in Mathematica.
They proved that the equivalence of basic Dirac notations are decidable.
Their system sticks with a pure term rewriting system, which allows to prove important properties such as confluence and termination.
Even though, there is still a gap to a practical solver for Dirac notation equivalence.
The Mathematica implementation is hard to be integrated into other tools. 
The second concern is the efficiency. To decide the theory $E$, \DiracDec\ iterates all possible permutations, which is simple and direct but has factorial complexity.
And lastly, the previous work does not consider labelled Dirac notation.
The original work cannot intergrate with well with the Scalar system in mathematica.
And there were a few examples that failded to decide.





This work refines the pure term-rewriting based system into a hybrid decision procedure,
overcomes the problems mentioned above and focuses on practical
automated equational reasonings. Our main contributions are:
\begin{itemize}
    \item We propose a rigorous typing system for Dirac notation. Defined symbols (e.g., transpose and trace) are modelled by functions.
        Also, the typing system removes the difficulty of disambiguation, therefore the language and the term rewriting system are much simplified.
    \item An efficient algorithm to decide the extra equational theories $E$ is provided. The basic idea is normalization by sorting, which considers the interplay of AC symbols, SUM-SWAP and $\alpha$-equivalence. This algorithm reduces the complexity from the factorial level to the polynomial level.
    \item The language to support variable labels is proposed.
\end{itemize}

\paragraph{Related Work} Other previous works explore the language and decision procedure to express quantum computation differently.





\section{Preliminary and Motivation}
The preliminary for the problem are two folds: Dirac notation and the equational logic in universal algebra.

Quantum states live in complex Hilbert spaces. We use vectors in the space to descirbe pure quantum states, and operations corresponds to linear transformations. 
In 1939, Dirac proposed his notation~\cite{dirac1939new} for quantum mechanics, which is essentially a wrapping of the linear algebra language. 
Dirac notation uses the ket $\ket{i}$ and the bra $\bra{i}$ to indicate bases of the space and the dual space. Together with other variable symbols, they are composed with each other in sequence, and the composition will be interpreted into different operations, depending on the type of operands. For example, $\braket{i|j}$ represents the inner product of $\bra{i}$ and $\ket{j}$, which is a scalar, while $\ket{i}\bra{j}$ represents the outer product, resulting in an operator. 
\[
    \braket{i|\phi}\braket{\psi|j} = \bra{i} (\ket{\phi}\bra{\psi}) \ket{j},
\]
and they are equivalent for all variables.
Dirac notation also use $\otimes$ to indicate the vectors and operators in the tensor product space.

In this manner, we can express long formulae in succint Dirac notations. The notation further enjoys the property that the interpretation is independent on the order of composition, thus parentheses can be omitted. For example, the formula \(\bra{i}\ket{\phi}\bra{\psi}\ket{j}\) can be understood as

With the concrete basis, Dirac notations can be interpreted as matrices. For example ...



We use universal algebra and equational logic to formally represent Dirac notations and the reasoning procedure in computers.

\yx{mention the dilemma of simplicity and efficiency}


\section{Language, Typing and Semantics}
The syntax of Dirac notations involves three layers: the index, the type and the term. 
Index represents classical datatypes, and they appear in type expression to denote the type of Hilbert spaces and sets.


\begin{definition}[language syntax]
    The syntax for type indices are defined as
    \begin{align*}
        \sigma ::=\ & x \mid \sigma_1 \times \sigma_2 \mid \mathsf{Bit}.
    \end{align*}
    The syntax for Dirac notation types is defined as 
    \begin{align*}
        T ::=\ & \Basis(\sigma) \mid \SType \mid \KType(\sigma) \mid \BType(\sigma) \mid \OType(\sigma_1, \sigma_2) \mid T_1 \to T_2 \mid \forall x.T \mid \SET(\sigma). \\
    \end{align*}
    The syntax for Dirac notation terms is defined as
    \begin{align*}
        e ::=\ & x \mid \lambda x : T.e \mid \mu x.e \mid e_1\ e_2 \mid e_1 \circ e_2 \\
        & |\ \hat{0} \mid \hat{1} \mid (e_1, e_2) \\
        & |\ 0 \mid 1 \mid \ADDS(e_1 \cdots e_n)  \mid e_1 \times \cdots \times e_n \mid e^*  \mid \delta_{e_1, e_2} \mid \DOT(e_1\ e_2) \\
        & |\ \ZEROK(\sigma) \mid \ZEROB(\sigma) \mid \ZEROO(\sigma_1, \sigma_2) \mid \ONEO(\sigma) \\
        & |\ \ket{e} \mid \bra{t} \mid e^\dagger \mid e_1.e_2 \mid \ADD(e_1 \cdots e_n) \mid e_1 \otimes e_2 \\
        & |\ \MULK(e_1\ e_2) \mid \MULB(e_1\ e_2) \mid \OUTER(e_1\ e_2) \mid \MULO(e_1\ e_2) \\
        & |\ \mathbf{U}(e) \mid e_1 \star e_2 \mid \sum_{e_1} e_2.
    \end{align*}
\end{definition}
 % after consideration I still think keep DOT as binary is the best choice. It reduces the unstability of complicated matching in sequences.
Here $i$ is a natural number and $\$i$ represents the $i$-th bound variable in de Bruijn notation. 
Compared to [?], this syntax for Dirac notations merges the symbols with overlapped properties, such as the addition and scaling symbols for ket, bra and operator.
Here $\ADDS$ and $\ADD$ are two different AC symbols representing the scalar addition and the linear algebra addition respectively. They will be denoted as $a_1 + \cdots + a_n$ and $X_1 + \cdots + X_n$.
There are five kinds of linear algebra multiplications among ket, bra and operator, whose properties are similar but still diverge to some extent. For example, the rules $(O_1 \cdot O_2) \cdot K\ \reduce\ O_1 \cdot (O_2 \cdot K)$ and $B \cdot (O_1 \cdot O_2) \ \reduce\ (B \cdot O_1) \cdot O_2$ indicate that the sorting of multiplication sequences depends on the subterm types. To avoid frequent but unnecessary type checkings, we encode the typing information by using five different symbols, namely $\DOT$, $\MULK$, $\MULB$, $\OUTER$ and $\MULO$. They are denoted as $B\cdot K$, $K_1 \cdot K_2$, $B_1 \cdot B_2$, $K \cdot B$ and $O_1 \cdot O_2$, respectively.

Usually, the sum body is specified by an abstraction. Therefore we use notation $\sum_s X$ to denote $\sum_{x \in s} \lambda x : T . X$ as well.
 


\subsection{Typing System}


The type checking of our language involves maintaining a well-formed environment and context $E[\Gamma]$, which is defined as follows.

\begin{definition}[environment and context]
    \begin{align*}
        E ::=\ & [] \mid E; x : \Index \mid E; x : T \mid E; x := t : T. \\
        \Gamma ::=\ & [] \mid \Gamma; x : \Index \mid \Gamma; x : T.
    \end{align*}
\end{definition}

The environment and the context are sequences of assumptions $x : T$ or definitions $x := t : T$.
With the environment, we can declare the type of variable symbols, and encode more operations on Dirac notations as definitions, such as the trace operator. Also, the existence of lambda abstractions requires a context of bound variables.


We say an expression $t$ has type $X$ in context $E[\Gamma]$, if the typing judgement $E[\Gamma] \vdash t : X$ can be proved through the typing rules in~\Cref{sec: full typing rules}. Here we present and explain the rules selectively. Firstly, well-formed contexts $\WF(E)[\Gamma]$ are built in the incremental way, e.g.:
\[
    \frac{}{\WF([])[]}
    \qquad
    \frac{\WF(E)[] \qquad x \notin E}{\WF(E; x : \Index)[]}
    \qquad
    \frac{E[]\vdash t:T \qquad x \notin E}{\WF(E; x:=t:T)[]}.
\]
Starting from an empty context, we can assume new index symbols, and assume or define symbols with checked types. Based on the well-formed context, typing judgements can be proved by information from $E[\Gamma]$, or built inductively. In the following rules, for example, the condition $x : \Index \in E[\Gamma]$ is true if $E$ or $\Gamma$ has the assumption in their sequences, and $\sigma \times \tau$ is a index if both $\sigma$ and $\tau$ are typed as the index. $\KType(\sigma)$ and $\OType(\sigma, \tau)$ will be valid types for kets and operators, if their arguments are typed as the index.
\begin{gather*}
    \frac{\WF(E)[\Gamma] \qquad x : \Index \in E[\Gamma]}{E[\Gamma] \vdash x : \Index}
    \qquad
    \frac{E[\Gamma] \vdash \sigma : \Index \qquad E[\Gamma] \vdash \tau : \Index}{E[\Gamma] \vdash \sigma \times \tau : \Index} \\
    \\
    \frac{E[\Gamma] \vdash \sigma : \Index}{E[\Gamma] \vdash \KType(\sigma) : \Type}
    \qquad
    \frac{E[\Gamma] \vdash \sigma : \Index \qquad E[\Gamma] \vdash \tau : \Index}{E[\Gamma] \vdash \OType(\sigma, \tau) : \Type}
\end{gather*}

The Dirac notations will then be typed accordingly. For example, the ket syntax $\ket{t}$ has type $\KType(\sigma)$, if $t$ is typed as a basis term of index $\sigma$. Also, the inner product of a bra and a ket with the same type index $\sigma$ is typed as the scalar. This corresponds to the constraint of inner products that vectors should be in the same Hilbert space.
\begin{gather*}
    \frac{E[\Gamma]\vdash t : \Basis(\sigma)}{E[\Gamma] \vdash \ket{t} : \KType(\sigma)}
    \qquad
    \frac{E[\Gamma]\vdash B : \BType(\sigma) \qquad E[\Gamma]\vdash K : \KType(\sigma)}{E[\Gamma] \vdash B \cdot K : \SType}
\end{gather*}

The typing for functions and applications follow the common practice. We specify the syntax and typing rule for functions of indices $\mu x. t$: here $x$ is a bound variable of typed as $\Index$, and the type $U\{x/u\}$ of application $(t u)$ is obtained by replacing $x$ with the index instance $u$.


\begin{gather*}
    \frac{E[\Gamma; x : T] \vdash t : U}{E[\Gamma] \vdash (\lambda x : T . t) : T \to U}
    \qquad
    \frac{E[\Gamma] \vdash t:U \to T \qquad E[\Gamma] \vdash u:U}{E[\Gamma] \vdash (t\ u):T} \\
    \\
    \frac{E[\Gamma; x : \Index] \vdash t : U}{E[\Gamma] \vdash (\mu x.t) : \forall x.U}
    \qquad
    \frac{E[\Gamma] \vdash t:\forall x.U \qquad E[\Gamma] \vdash u : \Index}{E[\Gamma] \vdash (t\ u):U\{x/u\}}
\end{gather*}


The big operator sum is modelled by folding a function on a set, therefore the typing rule is as follows:
\[
    \frac{E[\Gamma] \vdash s : \SET(\sigma) \qquad E[\Gamma] \vdash f : \Basis(\sigma) \to \KType(\tau)}{E[\Gamma] \vdash \sum_{s} f : \KType(\tau)}.
\]

And lastly we have the typing rules for composition $x \circ y$. As is the case for casual Dirac notation, the typing of composition depends on the types of operands.
\begin{gather*}
    \frac{E[\Gamma] \vdash x : \SType \qquad E[\Gamma] \vdash y : \KType(\sigma)}{E[\Gamma] \vdash x \circ y : \KType(\sigma)}
    \qquad
    \frac{E[\Gamma] \vdash x : \OType(\sigma, \tau) \qquad E[\Gamma] \vdash y : \KType(\tau)}{E[\Gamma] \vdash x \circ y : \KType(\sigma)}
\end{gather*}

\begin{lemma}
    The typing of expressions are decidable and unique.
\end{lemma}

\subsection{Semantics}

Semantics assign the meanings to the expressions, and the goal of the decision procedure is to decide whether
two expressions have the equivalent semantics. We consider two ways of definition: the axiomatic equations and the denotational semantics.

The denotational way interprets every expression as a linear algebra concept, and equivalence is considered in the common mathematical sense.
This explanation formalizes the original definition of Dirac notations, and best describes the target of the decision procedure.

The semantics by equations, on the other hand, is an abstraction and axiomization. From the operational view, each equation declares a 
valid rewriting operation, and two expressions are equiavlent if and only if they can be rewritten into the same form using the axioms.

\section{Decision procedure for Dirac notations}

The following two sections talk about how to decide the equivalence of Dirac notations.
In the previous work, the equational axioms are separated into two parts: a set $E$ of equations that can not be decided by rewriting, and the remaining equations to be decided by term rewriting.
Here, the equational theories $E$ inlude:
\begin{itemize}
    \item Commutativity of symbols $a + b$, $a\times b$ and $\delta_{s, t}$,
    \item $\alpha$-equivalence of bound variables, i.e., $\lambda x.t = \lambda y . t\{x/y\}$,
    \item swapping successive summations, i.e., $\sum_{i \in s_1} \sum_{j \in s_2} A = \sum_{j \in s_2} \sum_{i \in s_1}A$, and
    \item equational theories for scalars.
\end{itemize}

In this work, the general idea is to carry through the normalization procedure, so that semantical equivalence can be directly checked by the syntax of normal forms. The procedure of the normalization is displayed in~\Cref{fig: normalization}.


\tikzstyle{rewriting} = [rectangle, rounded corners, minimum width=3cm, minimum height=0.8cm,text centered, draw=black, fill=red!30]
\tikzstyle{process} = [rectangle, minimum width=3cm, minimum height=0.8cm, text centered, draw=black, fill=blue!30]
\tikzstyle{decision} = [diamond, minimum width=3cm, minimum height=0.8cm, text centered, draw=black, fill=green!30]
\tikzstyle{textstyle} = [text centered]
\tikzstyle{arrow} = [thick,->,>=stealth]

\begin{figure}
    \center
    \begin{tikzpicture}[node distance=1.1cm]

    % \node (rename) [process] {rename unique bound variable names};
    \node (rewrite1) [rewriting] {first rewritings};
    \node (rewrite1caption) [textstyle, right of=rewrite1] {expand definitions and simplification};
        
    \node (expansion) [process, below of=rewrite1] {variable expansion};
    \node (rewrite2) [rewriting, below of=expansion] {second rewritings};
    \node (sort) [process, below of=rewrite2] {sorting without bound variables};
    \node (swap) [process, below of=sort] {swap successive summation};
    \node (deBruijn) [process, below of=swap] {de Bruijn normalization};

    % \draw [arrow] (rename) -- (rewrite1);
    \draw [arrow] (rewrite1) -- (expansion);
    \draw [arrow] (expansion) -- (rewrite2);
    \draw [arrow] (rewrite2) -- (sort);
    \draw [arrow] (sort) -- (swap);
    \draw [arrow] (swap) -- (deBruijn);

    \end{tikzpicture}
    \caption{A flowchart for normalization of Dirac notations.}
    \label{fig: normalization}
\end{figure}

The last three steps deals with the equational theory $E$, while the first three steps use term rewriting to work on the structure of Dirac notations.


\subsection{Normalization modulo $E$ by Term Rewriting}
Term rewriting rules, written as $l -> r$, are used to normalize terms by recursively matching the subterms of the term with the left hand side $l$, and replace them with the corresponding right hand side $r$. The procedure terminates when no more rewritings can be made, and the order of rewritings will be irrelevant if the term rewriting system is \textit{confluent}, which is a desirable property.

To express the term rewriting system, the previous work adheres to use the naive universal algebra, where each function symbol has a fixed arity, and the left hand side pattern only allows constant symbols and variables. This constraint enables them to check the confluence and termination of their term rewriting system with other tools.

Here to enable more efficient algorithms, our langauge uses AC symbols with a variable arity.

The full list of rewriting rules are in~\Cref{sec: rewriting rules}. Here we present some of them to illustrate the design idea.

When using variable arity AC symbols, the associativity can be normalized by the following flattening rule:

\[
a_1 + \cdots + (b_1 + \cdots + b_m) + \cdots + a_n
\ \reduce\ a_1 + \cdots + b_1 + \cdots + b_m + \cdots + a_n,
\]

while commutativity is left for a sorting algorithm later on.

The term rewriting rule follows the system in the previous work.

The general idea is to reduce all the possible calculations, and transform multiplication into tensor product as much as possible.

And some rules exists for completeness. As an example, given the following rule snippet
\begin{align*}
    %
    & \text{(R-DOT10)}
    && (B \cdot O) \cdot K \ \reduce\ B \cdot (O \cdot K), \\
    %
    & \text{(R-DOT11)}
    && \bra{(s, t)} \cdot ((O_1 \otimes O_2) \cdot K) \ \reduce\ ((\bra{s} \cdot O_1) \otimes (\bra{t} \cdot O_2)) \cdot K, \\
    %
    & \text{(R-MULB10)}
    && \bra{(s, t)} \cdot (O_1 \otimes O_2)\ \reduce\ (\bra{s} \cdot O_1) \otimes (\bra{t} \cdot O_2),
\end{align*}
the normalization of term $(\bra{(s, t)} \cdot (O_1 \otimes O_2)) \cdot K$ have two rewriting paths: (a). apply (R-MULB10) and get $((\bra{s} \cdot O_1) \otimes (\bra{t} \cdot O_2)) \cdot K$, or
(b). first apply (R-DOT10) and sort the term into $\bra{(s, t)} \cdot ((O_1 \otimes O_2) \cdot K)$, and then apply (R-DOT11) to get the same result. Here (R-DOT11) is for completeness of cases similar to this example.

One important technique revealed in the previous work is the expansion of variables:

\begin{gather*}
    \frac{E[\Gamma] \vdash K : \KType(\sigma)}{E[\Gamma] \vdash K \ \reduce\ \sum_{i \in \mathbf{U}(\sigma)}(\bra{i}\cdot K).\ket{i}}
    \qquad \qquad
    \frac{E[\Gamma] \vdash B : \BType(\sigma)}{E[\Gamma] \vdash B \ \reduce\ \sum_{i \in \mathbf{U}(\sigma)}(B \cdot \ket{i}).\bra{i}} \\
    \\
    \frac{E[\Gamma] \vdash O : \OType(\sigma, \tau)}{E[\Gamma] \vdash O \ \reduce\ \sum_{i \in \mathbf{U}(\sigma)} \sum_{j \in \mathbf{U}(\tau)}(\bra{i} \cdot O \cdot \ket{j}).(\ket{i} \cdot \bra{j})}
\end{gather*}
The above three rules are obviously not terminating, and this is why we have the rewriting-expansion-rewriting steps in the decision procedure. On the other hand, we discovered that doing the expansion on all variables only once is already sufficient.

\begin{lemma}
    Let $\text{expand}(e)$ indicate the result of expanding all variables in $e$ once.
    For all well-typed term $e$ in $E[\Gamma]$, we have $\text{expand}(\text{expand}(e)) \downarrow = \text{expand}(e)\downarrow$.
\end{lemma}
\begin{proof}
\end{proof}



\section{Deciding Equational Theory $E$}

The previous work did not consider the algorithm to decide equational theory $E$. In their Mathematica implementation, it is implemented by a unification, which tries to find a substitution of summation bound variables that makes the two expressions syntactically equivalent. To decide AC-equivalence and (SUM-SWAP), they iterate through all the permutations, and the complexity is factorial to the number of AC symbol arguments and successive summations.

A standard approach to decide this permutation equivalence is to normalize by sorting with a given order. For example, given the dictionary order $a < b < c$, the term $b + c + a$ (and any other AC equivalent ones) will be normalized into $a + b + c$. For our setting, there are two related difficulties: how to assign such an order to all terms in our language, and how to normalize with respect to AC-equivalence and (SUM-SWAP) at the same time.






At last, we will prove that equivalence by this normalization procedure is sound in the semantics. 

\begin{theorem}[soundness]
    For any well-formed context $E[\Gamma]$ and well-typed expressions $e_1$ and $e_2$, if $e_1\downarrow = e_2\downarrow$, then $\sem{e_1} = \sem{e_2}$.
\end{theorem}
\begin{proof}
    
\end{proof}

% Labelled Dirac Notation




\newcommand{\var}{\mathsf{var}}
\newcommand{\reg}{\mathsf{Reg}}
\newcommand{\DType}{\mathcal{D}}
\newcommand{\cR}{\mathcal{R}}
\newcommand{\cN}{\mathcal{N}}
\newcommand{\tD}{\tilde{D}}
\newcommand{\te}{\tilde{e}}
\newcommand{\tT}{\tilde{T}}
\newcommand{\tADD}{\widetilde{ADD}}
\newcommand{\bU}{\mathbf{U}}
\newcommand{\<}{\langle}
\newcommand{\simp}{\mathsf{Simp}}
\newcommand{\List}{\mathsf{list}}
\renewcommand{\>}{\rangle}


\section{Labelled Dirac Notation}

Labelled Dirac notation uses register names to indicate the quantum system of vectors and operators. This allows us to talk aboud the states and operations locally, without referring to the whole system. For instance, assume $Q$ and $R$ are two registers, we have
\[
    M_Q \cdot \ket{\Phi}_{(Q, R)} = ((M \otimes I) \ket{\Phi})_{(Q, R)}.
\]
On the left hand side, the state of two subsystems is $\ket{\Phi}$, and we apply quantum operation $M$ on the system $Q$. It is equivalent to extend the operation using identity operators on other subsystems (i.e. the cylinder extension), and consider the application in the whole system.


In this section, we introduce the language and typing of registers and labelled Dirac notations, and demonstrate how to transform the equivalence problem into the Dirac notations studied above.

We assume the set of register symbols is $\cR$. The syntax of quantum registers is defined as follows.
\begin{definition}[quantum registers]
  \begin{align*}
    R ::= r\in\cR \mid (R, R).
  \end{align*}
\end{definition}

Registers can be composed by pairs of $(R, R)$, and this structure corresponds to the structure of tensor product spaces in Dirac notations.
To reason about the registers, we define their variable set as the enumeration of all register symbols involved.

\begin{definition}[register variable set]
The variable set of a register is defined inductively by:
\begin{itemize}
    \item $\var(R) = \{r\}$, if $R\equiv r$; or
    \item $\var(R) = \var(R_1) \cup\var(R_2)$, if $R\equiv (R_1,R_2)$.
\end{itemize}
\end{definition}

% \textbf{Remark: } Set operations: $\cup$ for union; $\cap$ for intersection; $\setminus$ for difference. So,
% % \begin{itemize}
% $ S_1 \cap S_2 \equiv S_1 \cup S_2 \setminus (S_1 \setminus S_2) \setminus (S_2 \setminus S_1) $.
% \end{itemize}

\begin{definition}[labelled Dirac notation]
  The \textbf{labelled Dirac notation} includes all Dirac notation symbols and the generators defined below.
  Here, $s\subseteq \cR$ is a register variable set.
  \begin{align*}
    T & ::= \DType(s,s) \mid \mathsf{Reg}(\sigma) \\
    e & ::= R \mid |i\>_r \mid {}_r\<i| \mid e_R \mid e_{R;R} \mid
    e \otimes e \otimes \cdots \otimes e \mid e \cdot e \\
  \end{align*}
\end{definition}

$\DType(s_1, s_2)$ is the unified type for all labelled Dirac notations, where $s_1$ indicates the codomain systems and $s_2$ indicates the domain systems. For instance, labelled ket has type $\DType(s_1, \emptyset)$, and labelled bra has type $\DType(\emptyset, s_2)$.
$\reg(\sigma)$ are types for registers $R$, and the index $\sigma$ indicates the type of Hilbert space represented by the register.
Terms also include the labelled notation $e_R$ for bra, ket and $e_{R;R}$ for operators. We introduce new dot and tensor product symbols for labelled Dirac notations. In labelled Dirac notation, the structure of tensor product does not matter. Therefore $\otimes$ is an AC symbol.
Following the unified type $\DType(s,s)$, all kinds of multiplications are represented by the same dot product $e \cdot e$.
Finally, $\ket{i}_r$ and ${}_r\bra{i}$ are labelled basis for the normal form of labelled Dirac notations, where $r$ are registers symbols in $\mathcal{R}$. 

\subsubsection*{Typing rules}
Some typing rules are introduced here.
The rule for $\DType(s_1, s_2)$ requires that all registers in variable set $s_1$ and $s_2$ are well-typed.
The rule for $K_R$ demonstrates how a register label is added to the Dirac notation, and the rule for $D^\dagger$ shows that labelled Dirac notation also have symbols for calculation, such as the adjoint.

\[
    \frac{E[\Gamma] \vdash  \sigma : \Index}{E[\Gamma] \vdash \reg(\sigma) : \Type}
    \qquad
    \frac{E[\Gamma] \vdash r : \reg(\sigma_r) \text{ for all $r$ in $s_1$ and $s_2$} }{E[\Gamma] \vdash \DType(s_1, s_2) : \textsf{Type}} 
\]

\[
    \frac{E[\Gamma] \vdash R : \reg(\sigma)\qquad E[\Gamma] \vdash K : \KType(\sigma)}{E[\Gamma] \vdash K_R : \DType(\var R, \emptyset)}
    \qquad
    \frac{E[\Gamma] \vdash D : \DType(s_1,s_2)}{E[\Gamma] \vdash D^\dagger : \DType(s_2,s_1)}
\]

The dot and the tensor product symbols are different from those in unlabelled Dirac notation. Since the goal of labels is to replace the order and structure of tensor products by the reference to registers, the tensor product becomes an AC symbol. The typing still checks whether the component subsystems are disjoint with each other.
\[
    \frac{
        E[\Gamma] \vdash D_i : \DType(s_i,s_i') \qquad
        \bigcap_i s_i = \emptyset \qquad
        \bigcap_i s_i' = \emptyset
    }
    {E[\Gamma] \vdash D_1 \otimes \cdots \otimes D_i : \DType(\bigcup_i s_i, \bigcup_i s_i')}.
\]
As for the dot product, the disjointness is considered except registers contracted by multiplication.
\[
    \qquad
    \frac{
        \begin{aligned}
            E[\Gamma] \vdash D_1 : \DType(s_1,s_1') \\
            E[\Gamma] \vdash D_2 : \DType(s_2,s_2')
        \end{aligned}
        \qquad 
        \begin{aligned}
            s_1 \cap s_2 \backslash s_1' = \emptyset \\
            s_2' \cap s_1' \backslash s_2 = \emptyset
        \end{aligned}
    }
    {E[\Gamma] \vdash D_1\cdot D_2 : \DType(s_1 \cup (s_2\backslash s_1'), s_2' \cup (s_1'\backslash s_2))}.
\]

\subsection{Normalization}
The main idea of normalizing labelled Dirac notations is \textit{label elimination}. It expands $e_R$ and $e_{R;R}$ by tensor products of labelled basis, which are formalized as rules below.

\begin{align*}
    & K_R \ \reduce\ \sum_{i_{r_1}\in\bU(\sigma_{r_1})}\cdots \sum_{i_{r_n}\in\bU(\sigma_{r_n})} (\<i_R|\cdot K). (|i_{r_1}\>_{r_i}\otimes\cdots\otimes|i_{r_n}\>_{r_n}) \\
    %
    & B_R \ \reduce\ \sum_{i_{r_1}\in\bU(\sigma_{r_1})}\cdots \sum_{i_{r_n}\in\bU(\sigma_{r_n})} (B\cdot |i_R\>). ({}_{r_1}\<i_{r_1}|\otimes\cdots\otimes{}_{r_n}\<i_{r_n}|) \\
    %
    & O_{R,R'} \ \reduce\ \sum_{i_{r_1}\in\bU(\sigma_{r_1})}\cdots \sum_{i_{r_n}\in\bU(\sigma_{r_n})}
    \sum_{i_{r'_1}\in\bU(\sigma_{r'_1})}\cdots \sum_{i_{r'_{n'}}\in\bU(\sigma_{r'_{n'}})} \\
    & \qquad (\<i_R|\cdot O\cdot |i_{R'}\>).(|i_{r_1}\>_{r_i}\otimes\cdots\otimes|i_{r_n}\>_{r_n} \otimes {}_{r'_1}\<i_{r'_1}|\otimes\cdots\otimes{}_{r'_{n'}}\<i_{r'_{n'}}|)
\end{align*}

This step reduces all terms in the form of $K_R$ or $O_{R;R}$.
Other symbols on labelled Dirac notations are also reduced by rules like $(D_1 \cdot D_2)^\dagger \ \reduce\ D_2^\dagger \cdot D_1^\dagger$.

The final part of rules operates on sum and dot product. They will lift summation to the outside, and eliminate the bra-ket pairs whenever possible.

\begin{align*}
    & \textrm{(R-SUM-PUSHD0)}
    && X_1 \otimes \cdots (\sum_{i \in M} D) \cdots \otimes X_2\ \reduce\ \sum_{i \in M} (X_1 \otimes \cdots D \cdots \otimes X_n) \\
    %
    & \textrm{(R-SUM-PUSHD1)}
    && (\sum_{i \in M} D_1) \cdot D_2 \ \reduce\ \sum_{i \in M} (D_1 \cdot D_2) \\
    %
    & \textrm{(R-L-SORT0)}
    && A : \DType(s_1, s_2), B : \DType(s_1', s_2'), s_2 \cap s_1'=\emptyset \Rightarrow A \cdot B \ \reduce\ A \otimes B \\
    %
    & \textrm{(R-L-SORT1)}
    && {}_r\bra{i}\cdot\ket{j}_r \ \reduce\ \delta_{i, j} \\
    %
    & \textrm{(R-L-SORT2)}
    && {}_r\bra{i}\cdot(Y_1 \otimes \cdots \otimes \ket{j}_r \otimes \cdots \otimes Y_m) \ \reduce\ \delta_{i, j}.(Y_1  \otimes \cdots \otimes Y_m) \\
    %
\end{align*}

These rules are added to the rewriting system in ~\Cref{sec: decide} and executed together.
In the end, if there are no variables of $\DType(s_1, s_2)$, the expression will be reduced to the addition of big operator sum, and each sum body is Dirac notation scalar with labelled basis.

\begin{lemma}
    For a well-typed term $e$ in $E[\Gamma]$, if $e$ does not contain variables of $\DType(s_1, s_2)$ type, the normal form of $e$ will be
    \[
    \sum_{i}\cdots\sum_{j} a_1 . (\ket{i}_{p} \otimes \cdots \otimes \bra{j}_{q})
    + \cdots +
    \sum_{k}\cdots\sum_{l} a_m . (\ket{k}_{r} \otimes \cdots \otimes \bra{l}_{s})
    \]
    Where $a_i$ are scalar typed Dirac notations.
\end{lemma}




\section{Implementation and Case Study}



\section{Conclusion}

\clearpage

%%%%%%%%%%%%%%%%%%%%%%%%%%%

\section{Consideration}
\begin{itemize}
    \item Sorting requires that alpha equivalent terms have the same syntax.
    \item de Bruijn expression satisfies this requirement, but the typing and substitution becomes very complicated in the type index scenario.
    \item only compute de Bruijn form in the equivalence checking and sorting phase.
    \item pipeline: 
        \begin{enumerate}
            \item preprocessing
            \item rename unique bound variable names
            \item 1st rewritings
            \item variable expansion
            \item 2nd rewritings 
            \item sorting modulo bound variables
            \item sorting successive sum. The order depends on there occurances in the last sorting result. If no occurance, then the order will depend on the set (for sum) and the type (for lambda abstraction only).
            \item computing de Bruijn
        \end{enumerate}

    \item I found that typing of inner bound variables also requires modifying the context.
    Need to modify the tree visiting algorithm
    (Typed term rewriting with bound variables is not that easy)    
\end{itemize}

\section{TODO}
\begin{itemize}
    \item implemented trace output for the whole pipeline.
    \item better output.
    \item add Mathematica support for scalars
    \item output trace to Coq
    \item consider labelled Dirac notations
    \item better trace output
    \item better error logic
    \item better format output
    \item remove term bank and hash consing
\end{itemize}

\section{Things to Note}
\begin{itemize}
    \item The context should also be maintained during rewriting matching.
    \item We don't allow eta reduction. It will intertwine with SUM-SWAP and break the confluence.
    \item In each Delta reduction, the bound variables are replaced with unique variables. This should help solve the problem of conflicting variable names during substitution.
    \item I guess it may still be necessary to try all different bound variable assignments.
    \item the $\bar{U_{AC}}$, $\bar{U_{BC}}$ terms can be modelled by quantum registers.
\end{itemize}

%%%%%%%%%%%%%%%%%%%%%%%%%%

\section{\texttt{Diracoq} language}

\begin{align*}
    cmd ::=&\ \texttt{Def}(\texttt{ID}\ term) \\
        &|\ \texttt{Def}(\texttt{ID}\ term\ type) \\
        &|\ \texttt{Var}(\texttt{ID}\ term) \\
        &|\ \texttt{Check}(term) \\
        &|\ \texttt{Show}(\texttt{ID}) \\
        &|\ \texttt{ShowAll} \\
        &|\ \texttt{Normalize}(term) \mid \texttt{Normalize}(term\ \textsf{Trace}) \\
        &|\ \texttt{CheckEq}(term\ term) \\
    type ::=&\ \texttt{Type} \mid \texttt{Arrow}(type\ type) \\
            &|\ \texttt{Base} \\
    term ::=&\ \texttt{Type} \mid \texttt{fun}(\texttt{ID}\ type\ term) \mid \texttt{apply}(term\ term) \mid \texttt{ID} \\
\end{align*}

Comment \texttt{(* ... *)} can be inserted between commands.

These are tye parsing rules for different expressions:
\begin{itemize}
    \item \texttt{Def ID := term.} --- \texttt{Def(ID term)}
    \item \texttt{Def ID := term : type.} --- \texttt{Def(ID term type)}
    \item \texttt{Var ID : type.} --- \texttt{Var(ID type)}
    \item \texttt{Check ID.} --- \texttt{Check(term)}
    \item \texttt{Show ID.} --- \texttt{Show(ID)}
    \item \texttt{ShowAll.} --- \texttt{ShowAll}
    \item \texttt{Normalize term.} --- \texttt{Normalize(term)}
    \item \texttt{Normalize term with trace.} --- \texttt{Normalize(term Trace)}
    \item \texttt{Check term = term.} --- \texttt{CheckEq(term term)}
    \item \texttt{T1 -> T2} --- \texttt{Arrow(T1 T2)}.
    \item \texttt{forall x. T} --- \texttt{Forall(x T)}
    \item \texttt{(e1, e2)} --- \texttt{PAIR(e1 e2)}
    \item \texttt{fun x : T => e} --- \texttt{fun(x T e)}
    \item \texttt{idx x => e} --- \texttt{idx(x e)}
    \item \texttt{e1 @ e2} --- \texttt{COMPO(e1 e2)}
    \item \texttt{e1 + ... + en} --- \texttt{ADDG(e1 ... en)}
    \item \texttt{e1 * ... * en} --- \texttt{STAR(e1 ... en)}
    \item \texttt{e1\^{}*} --- \texttt{CONJ(e1)}
    \item \texttt{delta(e1, e2)} --- \texttt{DELTA(e1 e2)}
    \item \texttt{|e>} --- \texttt{KET(e)}
    \item \texttt{<e|} --- \texttt{BRA(e)}
    \item \texttt{e1\^{}D} --- \texttt{ADJ(e1)}
    \item \texttt{e1.e2} --- \texttt{SCR(e1 e2)}
    \item \texttt{Sum(i in s, e)} --- \texttt{SSUM(i s e)}
\end{itemize}

These symbols will be transformed into internal language:
\begin{itemize}
    \item $A \circ B$ : $S \circ S$, $S \circ K$, $S \circ B$, $S \circ O$, $K \circ S$, $K \circ K$, $K \circ B$, $B \circ S$, $B \circ K$, $B \circ B$, $B \circ O$, $O \circ S$, $O \circ K$, $O \circ O$, $f\circ a$ (arrow), $f\circ a$ (index).
    \item \texttt{STAR(a ... b)} :  $\sigma_1 \times \sigma_2$, $\MULS(a \cdots b)$, $O_1 \otimes O_2$, $ M_1 \star M_2$
    \item \texttt{ADDG(e1 ... en)} : $\ADDS(e1 \cdots en)$, $\ADD(e1 \cdots en)$
    \item \texttt{SSUM(i S e)} : \texttt{SUM(s FUN(i T e))}
\end{itemize}

% \begin{credits}
%     \subsubsection{\ackname} A bold run-in heading in small font size at the end of the paper is
%     used for general acknowledgments, for example: This study was funded
%     by X (grant number Y).
% \end{credits}
    
    


%
% ---- Bibliography ----
%
% BibTeX users should specify bibliography style 'splncs04'.
% References will then be sorted and formatted in the correct style.
%
\bibliographystyle{splncs04}
\bibliography{ref}
%

\appendix

\section{Full Typing Rules}

\label{sec: full typing rules}

This section includes the full list of typing rules.

\begin{itemize}
    \item Rules for a well-formed environment and context.
    \[
        \textbf{W-Empty} \qquad
        \frac{}{\WF([])[]}
    \]
    \[
        \textbf{W-AssumE-Index} \qquad
        \frac{\WF(E)[] \qquad x \notin E}{\WF(E; x : \Index)[]}
    \]
    \[
        \textbf{W-AssumE-Term} \qquad
        \frac{E[] \vdash T : \Type \qquad x \notin E}{\WF(E; x:T)[]}
    \]
    \[
        \textbf{W-Def-Term} \qquad
        \frac{E[]\vdash t:T \qquad x \notin E}{\WF(E; x:=t:T)[]}
    \]
    \[
        \textbf{W-AssumC-Index} \qquad
        \frac{\WF(E)[\Gamma]}{\WF(E)[\Gamma; x : \Index]}
    \]
    \[
        \textbf{W-AssumC-Term} \qquad
        \frac{E[\Gamma] \vdash T : \Type}{\WF(E)[\Gamma; x : T]}
    \]


    \item Rules for type indices.
    \[
        \textbf{Index-Var} \qquad
        \frac{\WF(E)[\Gamma] \qquad x : \Index \in E[\Gamma]}{E[\Gamma] \vdash x : \Index}
    \]
    \[
        \textbf{Index-Prod} \qquad
        \frac{E[\Gamma] \vdash \sigma : \Index \qquad E[\Gamma] \vdash \tau : \Index}{E[\Gamma] \vdash \sigma \times \tau : \Index}
    \]
    \[
        \textbf{Index-Qudit} \qquad
        \frac{\WF(E)[\Gamma]}{E[\Gamma] \vdash \mathsf{bool} : \Index}
    \]


    \item Rules for types.
    \[
        \textbf{Type-Lam} \qquad
        \frac{E[\Gamma] \vdash T : \Type \qquad E[\Gamma] \vdash U : \Type}{E[\Gamma] \vdash T \to U : \Type}
    \]
    \[
        \textbf{Type-Index} \qquad
        \frac{E[\Gamma ; x : \Index] \vdash U : \Type}{E[\Gamma] \vdash \forall x.U : \Type}
    \]
    \[
        \textbf{Type-Basis} \qquad
        \frac{E[\Gamma] \vdash \sigma : \Index}{E[\Gamma] \vdash \Basis(\sigma) : \Type}
    \]
    \[
        \textbf{Type-Ket} \qquad
        \frac{E[\Gamma] \vdash \sigma : \Index}{E[\Gamma] \vdash \KType(\sigma) : \Type}
    \]
    \[
        \textbf{Type-Bra} \qquad
        \frac{E[\Gamma] \vdash \sigma : \Index}{E[\Gamma] \vdash \BType(\sigma) : \Type}
    \]
    \[
        \textbf{Type-Opt} \qquad
        \frac{E[\Gamma] \vdash \sigma : \Index \qquad E[\Gamma] \vdash \tau : \Index}{E[\Gamma] \vdash \OType(\sigma, \tau) : \Type}
    \]
    \[
        \textbf{Type-Scalar} \qquad
        \frac{\WF(E)[\Gamma]}{E[\Gamma] \vdash \SType : \Type}
    \]
    \[
        \textbf{Type-Set} \qquad
        \frac{E[\Gamma] \vdash \sigma : \Index}{E[\Gamma] \vdash \SET(\sigma) : \Type}
    \]
    \[
        \textbf{Type-Register} \qquad
        \frac{E[\Gamma] \vdash  \sigma : \Index}{E[\Gamma] \vdash \reg(\sigma) : \Type}
    \]
    \[
        \textbf{Type-Labelled} \qquad
        \frac{E[\Gamma] \vdash r : \reg(\sigma_r) \text{ for all $r$ in $s_1$ and $s_2$} }{E[\Gamma] \vdash \DType(s_1, s_2) : \textsf{Type}} 
    \]

    \item Rules for variable and function typings. Here $U\{x/u\}$ means replacing the bound variable $x$ with $u$ in $U$.
    \[
        \textbf{Term-Var} \qquad
        \frac{
            \begin{aligned}
                & \WF(E)[\Gamma] \\
                & (x:T) \in E[\Gamma] \text{ or } (x := t : T) \in E \text{ for some $t$}
            \end{aligned}
        }
        {E[\Gamma] \vdash x : T}
    \]
    \[
        \textbf{Lam} \qquad
        \frac{E[\Gamma; x : T] \vdash t : U}{E[\Gamma] \vdash (\lambda x : T . t) : T \to U}
    \]
    \[
        \textbf{Index} \qquad
        \frac{E[\Gamma; x : \Index] \vdash t : U}{E[\Gamma] \vdash (\mu x.t) : \forall x.U}
    \]
    \[
        \textbf{App-Lam} \qquad
        \frac{E[\Gamma] \vdash t:U \to T \qquad E[\Gamma] \vdash u:U}{E[\Gamma] \vdash (t\ u):T}
    \]
    \[
        \textbf{App-Index} \qquad
        \frac{E[\Gamma] \vdash t:\forall x.U \qquad E[\Gamma] \vdash u : \Index}{E[\Gamma] \vdash (t\ u):U\{x/u\}}
    \]
    
    \item Basis term typing rules. 
    \[  \textbf{Basis-0} \qquad
        \frac{\WF(E[\Gamma])}{E[\Gamma] \vdash 0 : \Basis(\mathsf{bool})}
    \]
    \[  \textbf{Basis-1} \qquad
        \frac{\WF(E[\Gamma])}{E[\Gamma] \vdash 1 : \Basis(\mathsf{bool})}
    \]
    \[
        \textbf{Basis-Pair} \qquad
        \frac{E[\Gamma]\vdash s : \Basis(\sigma) \qquad E[\Gamma]\vdash t : \Basis(\tau)} {E[\Gamma]\vdash (s, t) : \Basis(\sigma \times \tau)}
    \]


    \item Composition typing rules. 
    \[  \textbf{Compo-SS} \qquad
        \frac{E[\Gamma] \vdash x : \SType \qquad E[\Gamma] \vdash y : \SType}{E[\Gamma] \vdash x \circ y : \SType}
    \]
    \[  \textbf{Compo-SK} \qquad
        \frac{E[\Gamma] \vdash x : \SType \qquad E[\Gamma] \vdash y : \KType(\sigma)}{E[\Gamma] \vdash x \circ y : \KType(\sigma)}
    \]
    \[  \textbf{Compo-SB} \qquad
        \frac{E[\Gamma] \vdash x : \SType \qquad E[\Gamma] \vdash y : \BType(\sigma)}{E[\Gamma] \vdash x \circ y : \BType(\sigma)}
    \]
    \[  \textbf{Compo-SO} \qquad
        \frac{E[\Gamma] \vdash x : \SType \qquad E[\Gamma] \vdash y : \OType(\sigma, \tau)}{E[\Gamma] \vdash x \circ y : \OType(\sigma, \tau)}
    \]
    \[  \textbf{Compo-KS} \qquad
        \frac{E[\Gamma] \vdash x : \KType(\sigma) \qquad E[\Gamma] \vdash y : \SType}{E[\Gamma] \vdash x \circ y : \KType(\sigma)}
    \]
    \[  \textbf{Compo-KK} \qquad
        \frac{E[\Gamma] \vdash x : \KType(\sigma) \qquad E[\Gamma] \vdash y : \KType(\tau)}{E[\Gamma] \vdash x \circ y : \KType(\sigma \times \tau)}
    \]
    \[  \textbf{Compo-KB} \qquad
        \frac{E[\Gamma] \vdash x : \KType(\sigma) \qquad E[\Gamma] \vdash y : \BType(\tau)}{E[\Gamma] \vdash x \circ y : \OType(\sigma, \tau)}
    \]
    \[  \textbf{Compo-BS} \qquad
        \frac{E[\Gamma] \vdash x : \BType(\sigma) \qquad E[\Gamma] \vdash y : \SType}{E[\Gamma] \vdash x \circ y : \BType(\sigma)}
    \]
    \[  \textbf{Compo-BK} \qquad
        \frac{E[\Gamma] \vdash x : \BType(\sigma) \qquad E[\Gamma] \vdash y : \KType(\sigma)}{E[\Gamma] \vdash x \circ y : \SType}
    \]
    \[  \textbf{Compo-BB} \qquad
        \frac{E[\Gamma] \vdash x : \BType(\sigma) \qquad E[\Gamma] \vdash y : \BType(\tau)}{E[\Gamma] \vdash x \circ y : \BType(\sigma \times \tau)}
    \]
    \[  \textbf{Compo-BO} \qquad
        \frac{E[\Gamma] \vdash x : \BType(\sigma) \qquad E[\Gamma] \vdash y : \OType(\sigma, \tau)}{E[\Gamma] \vdash x \circ y : \BType(\tau)}
    \]
    \[  \textbf{Compo-OS} \qquad
        \frac{E[\Gamma] \vdash x : \OType(\sigma, \tau) \qquad E[\Gamma] \vdash y : \SType}{E[\Gamma] \vdash x \circ y : \OType(\sigma, \tau)}
    \]
    \[  \textbf{Compo-OK} \qquad
        \frac{E[\Gamma] \vdash x : \OType(\sigma, \tau) \qquad E[\Gamma] \vdash y : \KType(\tau)}{E[\Gamma] \vdash x \circ y : \KType(\sigma)}
    \]
    \[  \textbf{Compo-OO} \qquad
        \frac{E[\Gamma] \vdash x : \OType(\sigma, \tau) \qquad E[\Gamma] \vdash y : \OType(\sigma', \tau')}{E[\Gamma] \vdash x \circ y : \OType(\sigma \times \sigma', \tau \times \tau')}
    \]
    \[
        \textbf{Compo-DD} \qquad
        \frac{
            \begin{aligned}
                E[\Gamma] \vdash x : \DType(s_1,s_1') \\
                E[\Gamma] \vdash y : \DType(s_2,s_2')
            \end{aligned}
            \qquad 
            \begin{aligned}
                s_1 \cap s_2 \backslash s_1' = \emptyset \\
                s_2' \cap s_1' \backslash s_2 = \emptyset
            \end{aligned}
        }
        {E[\Gamma] \vdash x\circ y : \DType(s_1 \cup (s_2\backslash s_1'), s_2' \cup (s_1'\backslash s_2))}
    \]


    \item Scalar term typing rules.
    \[
        \textbf{Sca-0} \qquad
        \frac{\WF(E)[\Gamma]}{E[\Gamma] \vdash 0 : \SType}
    \]
    \[
        \textbf{Sca-1} \qquad
        \frac{\WF(E)[\Gamma]}{E[\Gamma] \vdash 1 : \SType}
    \]
    \[
        \textbf{Sca-Delta} \qquad
        \frac{ E[\Gamma]\vdash s : \Basis(\sigma) \qquad E[\Gamma]\vdash t : \Basis(\sigma) } {E[\Gamma] \vdash \delta_{s, t} : \SType}
    \]
    \[
        \textbf{Sca-Add} \qquad
        \frac{E[\Gamma]\vdash a_i : \SType \text{ for all $i$}}{E[\Gamma]\vdash a_1 + \cdots + a_n : \SType}
    \]
    \[
        \textbf{Sca-Mul} \qquad
        \frac{E[\Gamma]\vdash a_i : \SType \text{ for all $i$}}{E[\Gamma]\vdash a_1 \times \cdots \times a_n : \SType}
    \]
    \[
        \textbf{Sca-Conj} \qquad
        \frac{E[\Gamma] \vdash a : \SType}{E[\Gamma] \vdash a^*:\SType}
    \]
    \[
        \textbf{Sca-Dot} \qquad
        \frac{E[\Gamma]\vdash B : \BType(\sigma) \qquad E[\Gamma]\vdash K : \KType(\sigma)}{E[\Gamma] \vdash B \cdot K : \SType}
    \]
    \[
        \textbf{Sca-Sum} \qquad
        \frac{E[\Gamma] \vdash s : \SET(\sigma) \qquad E[\Gamma] \vdash f : \Basis(\sigma) \to \SType}{E[\Gamma] \vdash \sum_{s} f : \SType}
    \]

    \item Ket term typing rules.
    \[
        \textbf{Ket-0} \qquad
        \frac{E[\Gamma] \vdash \sigma : \Index}{E[\Gamma] \vdash \ZEROK(\sigma) : \KType(\sigma)}
    \]
    \[
        \textbf{Ket-Basis} \qquad
        \frac{E[\Gamma]\vdash t : \Basis(\sigma)}{E[\Gamma] \vdash \ket{t} : \KType(\sigma)}
    \]
    \[
        \textbf{Ket-Adj} \qquad
        \frac{E[\Gamma] \vdash B : \BType(\sigma)}{E[\Gamma] \vdash B^\dagger : \KType(\sigma)}
    \]
    \[
        \textbf{Ket-Scr} \qquad
        \frac{E[\Gamma] \vdash a : \SType \qquad E[\Gamma] \vdash K : \KType(\sigma)}{E[\Gamma] \vdash a.K : \KType(\sigma)}
    \]
    \[
        \textbf{Ket-Add} \qquad
        \frac{E[\Gamma] \vdash K_i : \KType(\sigma) \text{ for all $i$}}{E[\Gamma] \vdash K_1 + \cdots + K_n : \KType(\sigma)}
    \]
    \[
        \textbf{Ket-MulK} \qquad
        \frac{E[\Gamma] \vdash O : \OType(\sigma, \tau) \qquad E[\Gamma] \vdash K : \KType(\tau)}{E[\Gamma] \vdash O \cdot K : \KType(\sigma)}
    \]
    \[
        \textbf{Ket-Tsr} \qquad
        \frac{E[\Gamma] \vdash K_1 : \KType(\sigma) \qquad E[\Gamma] \vdash K_2 : \KType(\tau)} {E[\Gamma] \vdash K_1 \otimes K_2 : \KType(\sigma \times \tau)}
    \]
    \[
        \textbf{Ket-Sum} \qquad
        \frac{E[\Gamma] \vdash s : \SET(\sigma) \qquad E[\Gamma] \vdash f : \Basis(\sigma) \to \KType(\tau)}{E[\Gamma] \vdash \sum_{s} f : \KType(\tau)}
    \]

    \item Bra term typing rules.
    \[
        \textbf{Bra-0} \qquad
        \frac{E[\Gamma] \vdash \sigma : \Index}{E[\Gamma] \vdash \ZEROB(\sigma) : \BType(\sigma)}
    \]
    \[
        \textbf{Bra-Basis} \qquad
        \frac{E[\Gamma]\vdash t : \Basis(\sigma)}{E[\Gamma] \vdash \bra{t} : \BType(\sigma)}
    \]
    \[
        \textbf{Bra-Adj} \qquad
        \frac{E[\Gamma] \vdash K : \KType(\sigma)}{E[\Gamma] \vdash K^\dagger : \BType(\sigma)}
    \]
    \[
        \textbf{Bra-Scr} \qquad
        \frac{E[\Gamma] \vdash a : \SType \qquad E[\Gamma] \vdash B : \BType(\sigma)}{E[\Gamma] \vdash a.B : \BType(\sigma)}
    \]
    \[
        \textbf{Bra-Add} \qquad
        \frac{E[\Gamma] \vdash B_i : \BType(\sigma) \text{ for all $i$}}{E[\Gamma] \vdash B_1 + \cdots + B_n : \BType(\sigma)}
    \]
    \[
        \textbf{Bra-MulB} \qquad
        \frac{E[\Gamma] \vdash B : \KType(\sigma) \qquad E[\Gamma] \vdash O : \OType(\sigma, \tau)}{E[\Gamma] \vdash B \cdot O : \BType(\tau)}
    \]
    \[
        \textbf{Bra-Tsr} \qquad
        \frac{E[\Gamma] \vdash B_1 : \BType(\sigma) \qquad E[\Gamma] \vdash B_2 : \BType(\tau)} {E[\Gamma] \vdash B_1 \otimes B_2 : \BType(\sigma \times \tau)}
    \]
    \[
        \textbf{Bra-Sum} \qquad
        \frac{E[\Gamma] \vdash s : \SET(\sigma) \qquad E[\Gamma] \vdash f : \Basis(\sigma) \to \BType(\tau)}{E[\Gamma] \vdash \sum_{s} f : \BType(\tau)}
    \]

    \item Operator term typing rules.
    \[
        \textbf{Opt-0} \qquad
        \frac{E[\Gamma] \vdash \sigma : \Index \qquad E[\Gamma] \vdash \tau : \Index}{E[\Gamma] \vdash \ZEROO(\sigma, \tau) : \OType(\sigma, \tau)}
    \]
    \[
        \textbf{Opt-1} \qquad
        \frac{E[\Gamma] \vdash \sigma : \Index}{E[\Gamma] \vdash \ONEO(\sigma) : \OType(\sigma, \sigma)}
    \]
    \[
        \textbf{Opt-Adj} \qquad
        \frac{E[\Gamma] \vdash O : \OType(\sigma, \tau)}{E[\Gamma] \vdash O^\dagger : \OType(\tau, \sigma)}
    \]
    \[
        \textbf{Opt-Scr} \qquad
        \frac{E[\Gamma] \vdash a : \SType \qquad E[\Gamma] \vdash O : \OType(\sigma, \tau)}{E[\Gamma] \vdash a.O : \OType(\sigma, \tau)}
    \]
    \[
        \textbf{Opt-Add} \qquad
        \frac{E[\Gamma] \vdash O_i : \OType(\sigma, \tau) \text{ for all $i$}}{E[\Gamma] \vdash O_1 + \cdots + O_n : \OType(\sigma, \tau)}
    \]
    \[
        \textbf{Opt-Outer} \qquad
        \frac{E[\Gamma]\vdash K : \KType(\sigma) \qquad E[\Gamma]\vdash B : \BType(\tau)}{E[\Gamma]\vdash K \cdot B : \OType(\sigma, \tau)}
    \]
    \[
        \textbf{Opt-Mulo} \qquad
        \frac{E[\Gamma] \vdash O_1 : \OType(\sigma, \tau) \qquad E[\Gamma] \vdash O_2 : \OType(\tau, \rho)}{E[\Gamma] \vdash O_1 \cdot O_2 : \OType(\sigma, \rho)}
    \]
    \[
        \textbf{Opt-Tsr} \qquad
        \frac{E[\Gamma] \vdash O_1 : \OType(\sigma_1, \tau_1) \qquad E[\Gamma] \vdash O_2 : \OType(\sigma_2, \tau_2)} {E[\Gamma] \vdash O_1 \otimes O_2 : \OType(\sigma_1 \times \sigma_2, \tau_1 \times \tau_2)}
    \]
    \[
        \textbf{Opt-Sum} \qquad
        \frac{E[\Gamma] \vdash s : \SET(\sigma) \qquad E[\Gamma] \vdash f : \Basis(\sigma) \to \OType(\tau, \rho)}{E[\Gamma] \vdash \sum_{s} f : \OType(\tau, \rho)}
    \]

    \item Set term typing rules.
    \[
        \textbf{Set-U} \qquad
        \frac{E[\Gamma] \vdash \sigma : \Index}{E[\Gamma] \vdash \mathbf{U}(\sigma) : \SET(\sigma)}
    \]
    \[
        \textbf{Set-Prod} \qquad
        \frac{E[\Gamma] \vdash A : \SET(\sigma) \qquad E[\Gamma] \vdash B : \SET(\tau)}{E[\Gamma] \vdash A \star B : \SET(\sigma \times \tau)}
    \]


    \item Register term typing rules.
    \[
        \textbf{Reg-Var} \qquad
        \frac{\WF(E[\Gamma]) \qquad r : \reg(\sigma) \in E}{E[\Gamma] \vdash r : \reg(\sigma)}
    \]
    \[
        \textbf{Reg-Pair}\qquad
        \frac{
            \begin{aligned}
                E[\Gamma] \vdash R : \reg(\sigma) \\
                E[\Gamma] \vdash Q : \reg(\tau)
            \end{aligned}
            \qquad \var(R) \cap \var(Q) = \emptyset
        }{E[\Gamma] \vdash (R,Q) : \reg(\sigma \times \tau)}
    \]

    \item Typing rules for labelled Dirac notations.
    \[
        \textbf{L-Basis-Ket}\qquad 
        \frac{r : \reg(\sigma) \in E\qquad E[\Gamma] \vdash i : \Basis(\sigma)}{E[\Gamma] \vdash |i\>_r : \DType(\{r\}, \emptyset)}
    \]
    \[
        \textbf{L-Basis-Bra}\qquad 
        \frac{r : \reg(\sigma) \in E \qquad E[\Gamma] \vdash i : \Basis(\sigma)}{E[\Gamma] \vdash {}_r\<i| : \DType(\emptyset, \{r\})}
    \]
    \[
        \textbf{L-Ket}\qquad 
        \frac{E[\Gamma] \vdash R : \reg(\sigma)\qquad E[\Gamma] \vdash K : \KType(\sigma)}{E[\Gamma] \vdash K_R : \DType(\var R, \emptyset)}
    \]
    \[
        \textbf{L-Bra}\qquad 
        \frac{E[\Gamma] \vdash R : \reg(\sigma)\qquad E[\Gamma] \vdash B : \BType(\sigma)}{E[\Gamma] \vdash B_R : \DType(\emptyset, \var R)}
    \]
    \[
        \textbf{L-Opt}\qquad 
        \frac{
            \begin{aligned}
                E[\Gamma] \vdash R_1 : \reg (\sigma_1) \\
                E[\Gamma] \vdash R_2 : \reg (\sigma_2)
            \end{aligned}
            \qquad
            E[\Gamma] \vdash O : \OType(\sigma_1, \sigma_2)
        }
        {E[\Gamma] \vdash O_{R_1;R_2} : \DType(\var R_1, \var R_2)}
    \]
    \[
        \textbf{L-Conj}\qquad 
        \frac{E[\Gamma] \vdash D : \DType(s_1,s_2)}{E[\Gamma] \vdash D^\dagger : \DType(s_2,s_1)}
    \]
    \[
        \textbf{L-Scl}\qquad 
        \frac{E[\Gamma] \vdash S : \SType\qquad E[\Gamma] \vdash D : \DType(s_1,s_2)}{E[\Gamma] \vdash S.D : \DType(s_1,s_2)}
    \]
    \[
        \textbf{L-Add}\qquad
        \frac{E[\Gamma] \vdash D_i : \DType(s_1,s_2)\quad \text{forall } i}{E[\Gamma] \vdash D_1+\cdots+D_n : \DType(s_1,s_2)}
    \]
    \[
        \textbf{L-Tsr}\qquad
        \frac{
            E[\Gamma] \vdash D_i : \DType(s_i,s_i') \qquad
            \bigcap_i s_i = \emptyset \qquad
            \bigcap_i s_i' = \emptyset
        }
        {E[\Gamma] \vdash D_1 \otimes \cdots \otimes D_i : \DType(\bigcup_i s_i, \bigcup_i s_i')}
    \]
    \[
        \textbf{L-Dot}\qquad
        \frac{
            \begin{aligned}
                E[\Gamma] \vdash D_1 : \DType(s_1,s_1') \\
                E[\Gamma] \vdash D_2 : \DType(s_2,s_2')
            \end{aligned}
            \qquad 
            \begin{aligned}
                s_1 \cap s_2 \backslash s_1' = \emptyset \\
                s_2' \cap s_1' \backslash s_2 = \emptyset
            \end{aligned}
        }
        {E[\Gamma] \vdash D_1\cdot D_2 : \DType(s_1 \cup (s_2\backslash s_1'), s_2' \cup (s_1'\backslash s_2))}
    \]
    \[
        \textbf{L-Sum}\qquad
        \frac{E[\Gamma] \vdash s : \SET(\sigma) \qquad E[\Gamma] \vdash f : \Basis(\sigma) \to \DType(s_1, s_2)}{E[\Gamma] \vdash \sum_{s} f : \DType(s_1, s_2)}
    \]

\end{itemize}


\section{Axiomatic Semantics}

The full list of equational axioms are provided below.
    \begin{align*}
        & \textsc{(Ax-Scalar)} &&
        (B \cdot K)^* = K^\dagger \cdot B^\dagger
        \\
            & \textsc{(Ax-Delta)} &&
        \delta_{s, t}^* = \delta_{s, t}
        \qquad
        \bra{s} \cdot \ket{t} = \delta_{s, t}
        \\ & &&
        \delta_{s, s} = 1
        \qquad
        s \neq t \vdash \delta_{s, t} = 0
        \qquad
        \delta_{s, t} = \delta_{t, s}
        \\
            & \textsc{(Ax-Linear)} &&
        \mathbf{0} + D = D
        \qquad
        D_1 + D_2 = D_2 + D_1
        \\ & &&
        (D_1 + D_2) + D_3 = D_1 + (D_2 + D_3)
        \\ & &&
        0.D = \mathbf{0}
        \qquad
        a.\mathbf{0} = \mathbf{0}
        \qquad
        1.D = D
        \\ & &&
        a.(b.D) = (a \times b).D
        \qquad
        (a + b).D = a.D + b.D
        \\ & &&
        a.(D_1 + D_2) = a.D_1 + a.D_2
        \\
        & \textsc{(Ax-Bilinear)} &&
        D \cdot \mathbf{0} = \mathbf{0} 
        \qquad
        D_1 \cdot (a.D_2) = a.(D_1 \cdot D_2)
        \\ & &&
        D_0 \cdot (D_1 + D_2) = D_0 \cdot D_1 + D_0 \cdot D_2
        \\ & &&
        \mathbf{0} \cdot D = \mathbf{0}
        \qquad
        (a.D_1) \cdot D_2 = a.(D_1 \cdot D_2)
        \\ & &&
        (D_1 + D_2) \cdot D_0 = D_1 \cdot D_0 + D_2 \cdot D_0
        \\ 
        & &&
        D \otimes \mathbf{0} = \mathbf{0}
        \qquad
        D_1 \otimes (a.D_2) = a.(D_1 \otimes D_2)
        \\ & &&
        D_0 \otimes (D_1 + D_2) = D_0 \otimes D_1 + D_0 \otimes D_2
        \\ & &&
        \mathbf{0} \otimes D = \mathbf{0} 
        \qquad
        (a.D_1) \otimes D_2 = a.(D_1 \otimes D_2)
        \\ & &&
        (D_1 + D_2) \otimes D_0 = D_1 \otimes D_0 + D_2 \otimes D_0
        \\ 
            & \textsc{(Ax-Adjoint)} &&
        \mathbf{0}^\dagger = \mathbf{0}
        \qquad
        (D^\dagger)^\dagger = D 
        \qquad
        (a.D)^\dagger = a^*.(D^\dagger)
        \\ & &&
        (D_1 + D_2)^\dagger = D_1^\dagger + D_2^\dagger
        \\
        & && (D_1 \cdot D_2)^\dagger = D_2^\dagger \cdot D_1^\dagger
        \qquad
        (D_1 \otimes D_2)^\dagger = D_1^\dagger \otimes D_2^\dagger
        \\
            & \textsc{(Ax-Comp)} &&
        D_0 \cdot (D_1 \cdot D_2) = (D_0 \cdot D_1) \cdot D_2
        \\ & &&
        (D_1 \otimes D_2) \cdot (D_3 \otimes D_4) = (D_1 \cdot D_3) \otimes (D_2 \cdot D_4)
        % \\ & &&
        % (K_1 \cdot B_1) \cdot (K_2 \cdot B_2) = (B_1 \cdot K_2) . (K_1 \otimes B_2)
        \\ & &&
        (K_1 \cdot B) \cdot K_2 = (B \cdot K_2).K_1
        \qquad
        B_1 \cdot (K \cdot B_2) = (B_1 \cdot K).B_2
        % \\ & &&
        % (K \cdot B) \cdot O = K \cdot (B \cdot O)
        % \qquad
        % O \cdot (K \cdot B) = (O \cdot  K) \cdot B
        % \\
        %     & &&
        % (K_1 \cdot B_1) \otimes (K_2 \cdot B_2) = (K_1 \otimes K_2) \cdot (B_1 \otimes B_2)
        \\ 
            & &&
        (B_1 \otimes B_2) \cdot (K_1 \otimes K_2) = (B_1 \cdot K_1) \times (B_2 \cdot K_2)
        \\ 
            & \textsc{(Ax-Ground)} &&
        \mathbf{1}_\mathcal{O}^\dagger = \mathbf{1}_\mathcal{O}
        \qquad
        \textbf{1}_\mathcal{O} \cdot D = D 
        \qquad
        \mathbf{1}_\mathcal{O} \otimes \mathbf{1}_\mathcal{O} = \mathbf{1}_\mathcal{O} 
        \\ & &&
        \ket{t}^\dagger = \bra{t}
        \qquad
        \ket{s} \otimes \ket{t} =\ket{(s, t)} 
    \end{align*}
\yx{to be continued}




\section{Rewriting Rules}

\label{sec: rewriting rules}

This section includes all the rewriting rules used in the system. Related rules are collected in the same table. 

\renewcommand{\arraystretch}{1.2} % Increases row height by 50%

\begin{ruletable}{Reductions for the definitions and function applications.}
    %
    BETA-ARROW
    & $((\lambda x : T.t)\ u)\ \reduce\ t\{x/u\}$ \\
    %
    BETA-INDEX
    & $((\mu x.t)\ u)\ \reduce\ t\{x/u\}$ \\
    %
    DELTA
    & $(c:=t:T) \in E \Rightarrow c\ \reduce\ t$
\end{ruletable}

\begin{ruletable}{The special to flatten all AC symbols within one call.}
    %
    R-FLATTEN
    & $a_1 + \cdots + (b_1 + \cdots + b_m) + \cdots + a_n$ \\
    & $\reduce\ a_1 + \cdots + b_1 + \cdots + b_m + \cdots + a_n$ \\ 
    \\
    & $a_1 \times \cdots \times (b_1 \times \cdots \times b_m) \times \cdots \times a_n$ \\
    & $\reduce\ a_1 \times \cdots \times b_1 \times \cdots \times b_m \times \cdots \times a_n$ \\
    \\
    & $X_1 + \cdots + (X_1' + \cdots + X_m') + \cdots + X_n$ \\
    & $\reduce\ X_1 + \cdots + X_1' + \cdots + X_m' + \cdots + X_n$
\end{ruletable}

\begin{ruletable}{Rules for scalar symbols.}
    %
    R-CONJ5
    & $ \delta_{s, t}^* \ \reduce\ \delta_{s, t}$ \\
    %
    R-CONJ6
    & $ (B \cdot K)^* \ \reduce\ K^\dagger \cdot B^\dagger $ \\
    %
    R-DOT0
    & $ \ZEROB(\sigma) \cdot K \ \reduce\ 0 $ \\
    %
    R-DOT1
    & $ B \cdot \ZEROK(\sigma) \ \reduce\ 0 $ \\
    %
    R-DOT2
    & $ (a.B) \cdot K \ \reduce\ a \times (B \cdot K) $ \\
    %
    R-DOT3
    & $ B \cdot (a.K) \ \reduce\ a \times (B \cdot K) $ \\
    %
    R-DOT4
    & $ (B_1 + \cdots + B_n) \cdot K \ \reduce\ B_1 \cdot K + \cdots + B_n \cdot K $ \\
    %
    R-DOT5
    & $ B \cdot (K_1 + \cdots + K_n) \ \reduce\ B \cdot K_1 + \cdots + B \cdot K_n $ \\
    %
    R-DOT6
    & $ \bra{s} \cdot \ket{t} \ \reduce\ \delta_{s, t} $ \\
    %
    R-DOT7
    & $ (B_1 \otimes B_2) \cdot \ket{(s, t)} \ \reduce\ (B_1 \cdot \ket{s}) \times (B_2 \cdot \ket{t}) $ \\
    %
    R-DOT8
    & $ \bra{(s, t)} \cdot (K_1 \otimes K_2) \ \reduce\ (\bra{s} \cdot K_1) \times (\bra{t} \cdot K_2) $ \\
    %
    R-DOT9
    & $ (B_1 \otimes B_2) \cdot (K_1 \otimes K_2) \ \reduce\ (B_1 \cdot K_1) \times (B_2 \cdot K_2) $ \\
    %
    R-DOT10
    & $ (B \cdot O) \cdot K \ \reduce\ B \cdot (O \cdot K) $ \\
    %
    R-DOT11
    & $ \bra{(s, t)} \cdot ((O_1 \otimes O_2) \cdot K) \ \reduce\ ((\bra{s} \cdot O_1) \otimes (\bra{t} \cdot O_2)) \cdot K $ \\
    %
    R-DOT12
    & $ (B_1 \otimes B_2) \cdot ((O_1 \otimes O_2) \cdot K) \ \reduce\ ((B_1 \cdot O_1) \otimes (B_2 \cdot O_2)) \cdot K $ \\
    %
    R-DELTA0
    & $ \delta_{a, a} \ \reduce\ 1$ \\
    %
    R-DELTA1
    & $ \delta_{(a, b), (c, d)} \ \reduce\ \delta_{a, c} \times \delta_{b, d}$ \\
\end{ruletable}


\begin{ruletable}{Rules for scaling.}
    %
    R-SCR0
    & $ 1.X \ \reduce\ X $ \\
    %
    R-SCR1
    & $ a.(b.X) \ \reduce\ (a \times b).X $ \\
    %
    R-SCR2
    & $ a.(X_1 + \cdots + X_n) \ \reduce\ a.X_1 + \cdots + a.X_n $ \\
    %
    R-SCRK0
    & $ K : \KType(\sigma) \Rightarrow 0.K \ \reduce\ \ZEROK(\sigma)$ \\
    %
    R-SCRK1
    & $ a.\ZEROK(\sigma) \ \reduce\ \ZEROK(\sigma) $ \\
    % 
    R-SCRB0
    & $ B : \BType(\sigma) \Rightarrow 0.B \ \reduce\ \ZEROB(\sigma)$ \\
    % 
    R-SCRB1
    & $ a.\ZEROB(\sigma) \ \reduce\ \ZEROB(\sigma) $ \\
    % 
    R-SCRO0
    & $ O : \OType(\sigma, \tau) \Rightarrow 0.O \ \reduce\ \ZEROO(\sigma, \tau)$ \\
    % 
    R-SCRO1
    & $ a.\ZEROO(\sigma, \tau) \ \reduce\ \ZEROO(\sigma, \tau) $ \\
\end{ruletable}

\begin{ruletable}{Rules for addition. }
    %
    R-ADDID
    & $+(X) \ \reduce\ X$ \\
    %
    R-ADD0
    & $Y_1 + \cdots + X + \cdots + X + \cdots + Y_n \ \reduce\ Y_1 + \cdots + Y_n + \cdots + (1+1).X$ \\
    %
    R-ADD1
    & $Y_1 + \cdots + X + \cdots + a.X + \cdots + Y_n \ \reduce\ Y_1 + \cdots + Y_n + (1+a).X$ \\
    %
    R-ADD2
    & $Y_1 + \cdots + a.X + \cdots + X + \cdots + Y_n \ \reduce\ Y_1 + \cdots + Y_n + (a+1).X$ \\
    %
    R-ADD3
    & $Y_1 + \cdots + a.X + \cdots + b.X + \cdots + Y_n \ \reduce\ Y_1 + \cdots + Y_n + (a+b).X$ \\
    %
    R-ADDK0
    & $K_1 + \cdots + \ZEROK(\sigma) + \cdots + K_n\ \reduce K_1 + \cdots + K_n$ \\
    %
    R-ADDB0
    & $B_1 + \cdots + \ZEROB(\sigma) + \cdots + B_n \ \reduce\ B_1 + \cdots + B_n$ \\
    %
    R-ADDO0
    & $O_1 + \cdots + \ZEROO(\sigma, \tau) + \cdots + O_n \ \reduce\ O_1 + \cdots + O_n$ 
    \\
\end{ruletable}

\begin{ruletable}{Rules for adjoint.}
    %
    R-ADJ0
    & $ (X^\dagger)^\dagger \ \reduce\ X $ \\
    %
    R-ADJ1
    & $ (a.X)^\dagger \ \reduce\ (a^*).(X^\dagger) $ \\
    %
    R-ADJ2
    & $ (X_1 + \cdots + X_n)^\dagger \ \reduce\ X_1^\dagger + \cdots + X_n^\dagger $ \\
    %
    R-ADJ3
    & $ (X \otimes Y)^\dagger \ \reduce\ X^\dagger \otimes Y^\dagger$ \\
    %
    R-ADJK0
    & $ \ZEROB(\sigma)^\dagger \ \reduce\ \ZEROK(\sigma) $ \\
    %
    R-ADJK1
    & $ \bra{t}^\dagger \ \reduce\ \ket{t} $ \\
    %
    R-ADJK2
    & $ (B \cdot O)^\dagger\ \reduce\ O^\dagger \cdot B^\dagger $ \\
    %
    R-ADJB0
    & $ \ZEROK(\sigma)^\dagger \ \reduce\ \ZEROB(\sigma) $ \\
    %
    R-ADJB1
    & $ \ket{t}^\dagger \ \reduce\ \bra{t} $ \\
    %
    R-ADJB2
    & $ (O \cdot K)^\dagger\ \reduce\ K^\dagger \cdot O^\dagger $ \\
    %
    R-ADJO0
    & $ \ZEROO(\sigma, \tau)^\dagger \ \reduce\ \ZEROO(\tau, \sigma) $ \\
    %
    R-ADJO1
    & $ \ONEO(\sigma)^\dagger \ \reduce\ \ONEO(\sigma)$ \\
    %
    R-ADJO2
    & $ (K \cdot B)^\dagger \ \reduce\ B^\dagger \cdot K^\dagger$ \\
    %
    R-ADJO3
    & $ (O_1 \cdot O_2)^\dagger \ \reduce\ O_2^\dagger \cdot O_1^\dagger $ \\
\end{ruletable}

\begin{ruletable}{Rules for tensor product.}
    R-TSR0
    & $ (a.X_1) \otimes X_2 \ \reduce\ a.(X_1 \otimes X_2) $ \\
    %
    R-TSR1
    & $ X_1 \otimes (a.X_2) \ \reduce\ a.(X_1 \otimes X_2) $ \\
    %
    R-TSR2
    & $ (X_1 + \cdots + X_n) \otimes X' \reduce X_1 \otimes X' + \cdots + X_n \otimes X' $ \\
    %
    R-TSR3
    & $ X' \otimes (X_1 + \cdots + X_n) \reduce X' \otimes X_1 + \cdots + X' \otimes X_n $ \\
    %
    R-TSRK0
    & $ K : \KType(\tau) \Rightarrow \ZEROK(\sigma) \otimes K\ \reduce\ \ZEROK(\sigma \times \tau) $ \\
    %
    R-TSRK1
    & $ K : \KType(\tau) \Rightarrow K \otimes \ZEROK(\sigma)\ \reduce\ \ZEROK(\tau \times \sigma) $ \\
    %
    R-TSRK2
    & $\ket{s} \otimes \ket{t} \ \reduce\ \ket{(s, t)}$ \\
    %
    R-TSRB0
    & $ B : \BType(\tau) \Rightarrow \ZEROB(\sigma) \otimes B\ \reduce\ \ZEROB(\sigma \times \tau) $ \\
    %
    R-TSRB1
    & $ B : \BType(\tau) \Rightarrow B \otimes \ZEROB(\sigma)\ \reduce\ \ZEROB(\tau \times \sigma) $ \\
    %
    R-TSRB2
    & $\bra{s} \otimes \bra{t} \ \reduce\ \bra{(s, t)}$ \\
    %
    R-TSRO0
    & $ O : \OType(\sigma, \tau) \Rightarrow O \otimes \ZEROO(\sigma', \tau') \ \reduce\ \ZEROO(\sigma \times \sigma', \tau \times \tau') $ \\
    % 
    R-TSRO1
    & $ O : \OType(\sigma, \tau) \Rightarrow \ZEROO(\sigma', \tau') \otimes O \ \reduce\ \ZEROO(\sigma' \times \sigma, \tau' \times \tau) $ \\
    %
    R-TSRO2
    & $\ONEO(\sigma) \otimes \ONEO(\tau)\ \reduce\ \ONEO(\sigma \times \tau)$ \\
    %
    R-TSRO3
    & $ (K_1 \cdot B_1) \otimes (K_2 \cdot B_2)\ \reduce\ (K_1 \otimes K_2) \cdot (B_1 \otimes B_2) $ \\
    %
\end{ruletable}

\begin{ruletable}{Rule for $O\cdot K$.}
    %
    R-MULK0
    & $ \ZEROO(\sigma, \tau) \cdot K \ \reduce\ \ZEROK(\sigma) $ \\
    %
    R-MULK1
    & $ O : \OType(\sigma, \tau) \Rightarrow O \cdot \ZEROK(\tau) \ \reduce\ \ZEROK(\sigma) $ \\
    %
    R-MULK2
    & $ \ONEO(\sigma) \cdot K \ \reduce K $ \\
    %
    R-MULK3
    & $ (a.O) \cdot K \ \reduce\ a.(O \cdot K) $ \\
    %
    R-MULK4
    & $ O \cdot (a.K) \ \reduce\ a.(O \cdot K) $ \\
    %
    R-MULK5
    & $ (O_1 + \cdots + O_n) \cdot K \ \reduce\ O_1 \cdot K + \cdots + O_n \cdot K $ \\
    %
    R-MULK6
    & $ O \cdot (K_1 + \cdots + K_n) \ \reduce\ O \cdot K_1 + \cdots + O \cdot K_n $ \\
    %
    R-MULK7
    & $ (K_1 \cdot B) \cdot K_2 \ \reduce\ (B \cdot K_2).K_1 $ \\
    %
    R-MULK8
    & $ (O_1 \cdot O_2) \cdot K \ \reduce\ O_1 \cdot (O_2 \cdot K) $ \\
    %
    R-MULK9
    & $ (O_1 \otimes O_2) \cdot ((O_1' \otimes O_2') \cdot K)\ \reduce\ ((O_1 \cdot O_1') \otimes (O_2 \cdot O_2')) \cdot K $ \\
    %
    R-MULK10
    & $ (O_1 \otimes O_2) \cdot \ket{(s, t)}\ \reduce\ (O_1 \cdot \ket{s}) \otimes (O_2 \cdot \ket{t}) $ \\
    %
    R-MULK11
    & $ (O_1 \otimes O_2) \cdot (K_1 \otimes K_2)\ \reduce\ (O_1 \cdot K_1) \otimes (O_2 \cdot K_2) $
\end{ruletable}


\begin{ruletable}{Rule for $B\cdot O$.}
    %
    R-MULB0
    & $ B \cdot \ZEROO(\sigma, \tau) \ \reduce\ \ZEROB(\tau) $ \\
    %
    R-MULB1
    & $ O : \OType(\sigma, \tau) \Rightarrow \ZEROB(\sigma) \cdot O\ \reduce\ \ZEROB(\tau) $ \\
    %
    R-MULB2
    & $ B \cdot \ONEO(\sigma) \ \reduce B $ \\
    %
    R-MULB3
    & $ (a.B) \cdot O \ \reduce\ a.(B \cdot O) $ \\
    %
    R-MULB4
    & $ B \cdot (a.O) \ \reduce\ a.(B \cdot O) $ \\
    %
    R-MULB5
    & $ (B_1 + \cdots + B_n) \cdot O \ \reduce\ B_1 \cdot O + \cdots + B_n \cdot O $ \\
    %
    R-MULB6
    & $ B \cdot (O_1 + \cdots + O_n) \ \reduce\ B \cdot O_1 + \cdots + B \cdot O_n $ \\
    %
    R-MULB7
    & $ B_1 \cdot (K \cdot B_2) \ \reduce\ (B_1 \cdot K).B_2 $ \\
    %
    R-MULB8
    & $ B \cdot (O_1 \cdot O_2) \ \reduce\ (B \cdot O_1) \cdot O_2 $ \\
    %
    R-MULB9
    & $ (B \cdot (O_1' \otimes O_2')) \cdot (O_1 \otimes O_2) \ \reduce\ B \cdot ((O_1' \otimes O_2') \cdot (O_1 \otimes O_2)) $ \\
    %
    R-MULB10
    & $ \bra{(s, t)} \cdot (O_1 \otimes O_2)\ \reduce\ (\bra{s} \cdot O_1) \otimes (\bra{t} \cdot O_2) $ \\
    %
    R-MULB11
    & $ (B_1 \otimes B_2) \cdot (O_1 \otimes O_2)\ \reduce\ (B_1 \cdot O_1) \otimes (B_2 \cdot O_2) $
\end{ruletable}

\begin{ruletable}{Rules for $K \cdot B$.}
    %
    R-OUTER0
    & $ B : \BType(\tau) \Rightarrow \mathbf{0}_\mathcal{K}(\sigma) \cdot B\ \reduce\ \mathbf{0}_\mathcal{O}(\sigma, \tau) $ \\
    %
    R-OUTER1
    & $ K : \KType(\sigma) \Rightarrow K \cdot \mathbf{0}_\mathcal{B}(\tau)\ \reduce\ \mathbf{0}_\mathcal{O}(\sigma, \tau) $ \\
    %
    R-OUTER2
    & $ (a.K) \cdot B\ \reduce\ a.(K \cdot B) $ \\
    %
    R-OUTER3
    & $ K \cdot (a.B)\ \reduce\ a.(K \cdot B) $ \\
    %
    R-OUTER4
    & $ (K_1 + \cdots + K_n) \cdot B\ \reduce\ K_1 \cdot B + \cdots + K_n \cdot B $ \\
    %
    R-OUTER5
    & $ K \cdot (B_1 + \cdots + B_n)\ \reduce\ K \cdot B_1 + \cdots + K \cdot B_n $ \\
\end{ruletable}

\begin{ruletable}{Rules for $O_1 \cdot O_2$.}
    %
    R-MULO0
    & $ O : \OType(\tau, \rho) \Rightarrow \ZEROO(\sigma, \tau) \cdot O\ \reduce\ \ZEROO(\sigma, \rho) $ \\
    %
    R-MULO1
    & $ O : \OType(\sigma, \tau) \Rightarrow O \cdot \ZEROO(\tau, \rho)\ \reduce\ \mathbf{0}_\mathcal{O}(\sigma, \rho) $ \\
    %
    R-MULO2
    & $ \ONEO(\sigma) \cdot O \ \reduce\ O $ \\
    %
    R-MULO3
    & $ O \cdot \ONEO(\sigma) \ \reduce\ O $ \\
    %
    R-MULO4
    & $ (K \cdot B) \cdot O \ \reduce\ K \cdot (B \cdot O) $ \\
    %
    R-MULO5
    & $ O \cdot (K \cdot B) \ \reduce\ (O \cdot K) \cdot B $ \\
    %
    R-MULO6
    & $ (a.O_1) \cdot O_2 \ \reduce\ a.(O_1 \cdot O_2) $ \\
    %
    R-MULO7
    & $ O_1 \cdot (a.O_2) \ \reduce\ a.(O_1 \cdot O_2) $ \\
    %
    R-MULO8
    & $ (O_1 + \cdots + O_n) \cdot O'\ \reduce\ O_1 \cdot O' + \cdots + O_n \cdot O' $ \\
    %
    R-MULO9
    & $ O' \cdot (O_1 + \cdots + O_n)\ \reduce\ O' \cdot O_1 + \cdots + O' \cdot O_n $ \\
    %
    R-MULO10
    & $ (O_1 \cdot O_2) \cdot O_3\ \reduce\ O_1 \cdot (O_2 \cdot O_3) $ \\
    %
    R-MULO11
    & $ (O_1 \otimes O_2) \cdot (O_1' \otimes O_2')\ \reduce\ (O_1 \cdot O_1') \otimes (O_2 \cdot O_2') $ \\
    %
    R-MULO12
    & $ (O_1 \otimes O_2) \cdot ((O_1' \otimes O_2') \cdot O_3)\ \reduce\ ((O_1 \cdot O_1') \otimes (O_2 \cdot O_2')) \cdot O_3 $ \\  
\end{ruletable}

\begin{ruletable}{Rules for sets.}
    %
    R-SET0
    & $ \mathbf{U}(\sigma) \star \mathbf{U}(\tau) \ \reduce\ \mathbf{U}(\sigma \times \tau) $
\end{ruletable}

\begin{ruletable}{Rules for sum operators.}
    %
    R-SUM-CONST0
    & $ \sum_{x \in s} 0 \ \reduce\ 0 $ \\
    %
    R-SUM-CONST1
    & $ \sum_{x \in s} \ZEROK(\sigma)\ \reduce\ \ZEROK(\sigma) $ \\
    %
    R-SUM-CONST2
    & $ \sum_{x \in s} \ZEROB(\sigma)\ \reduce\ \ZEROB(\sigma) $ \\
    %
    R-SUM-CONST3
    & $ \sum_{x \in s} \ZEROO(\sigma, \tau)\ \reduce\ \ZEROO(\sigma, \tau) $ \\
    %
    R-SUM-CONST4
    & $ \ONEO(\sigma) \ \reduce\ \sum_{i \in \mathbf{U}(\sigma)} \ket{i} \cdot \bra{i} $
\end{ruletable}

\begin{ruletable}{Rules for eliminating $\delta_{s, t}$. These rules match the $\delta$ operator modulo the commutativity of its arguments.}
    %
    R-SUM-ELIM0
    & $ i \text{ free in } t \Rightarrow \sum_{i \in \mathbf{U}(\sigma)} \sum_{k_1 \in s_1} \cdots \sum_{k_n \in s_n} \delta_{i, t}$ \\
    & $ \reduce\ \sum_{k_1 \in s_1} \cdots \sum_{k_n \in s_n}  1$ \\
    \\
    %
    R-SUM-ELIM1
    & $ i \text{ free in } t \Rightarrow $ \\
    & $ \sum_{i \in \mathbf{U}(\sigma)} \sum_{k_1 \in s_1} \cdots \sum_{k_n \in s_n} (a_1 \times \cdots \times \delta_{i, t} \times \cdots \times a_n) $ \\
    & $ \reduce\ \sum_{k_1 \in s_1} \cdots \sum_{k_n \in s_n} a_1\{i/t\} \times \cdots \times a_n\{i/t\} $ \\
    \\
    %
    R-SUM-ELIM2
    & $ i \text{ free in } t \Rightarrow \sum_{i \in \mathbf{U}(\sigma)} \sum_{k_1 \in s_1} \cdots \sum_{k_n \in s_n} (\delta_{i, t}.A) $ \\
    & $ \reduce\ \sum_{k_1 \in s_1} \cdots \sum_{k_n \in s_n} A\{i/t\} $ \\
    \\
    %
    R-SUM-ELIM3
    & $ i \text{ free in } t \Rightarrow $ \\
    & $ \sum_{i \in \mathbf{U}(\sigma)} \sum_{k_1 \in s_1} \cdots \sum_{k_n \in s_n} (a_1 \times \cdots \times \delta_{i, t} \times \cdots \times a_n).A $ \\
    & $ \reduce\ \sum_{k_1 \in s_1} \cdots \sum_{k_n \in s_n}  (a_1\{i/t\} \times \cdots \times a_n\{i/t\}).A\{i/t\} $ \\
    \\
    %
    R-SUM-ELIM4
    & $ \sum_{i \in M} \sum_{j \in M} \sum_{k_1 \in s_1} \cdots \sum_{k_n \in s_n}  \delta_{i, j}$ \\ 
    & $\reduce\ \sum_{j \in M} \sum_{k_1 \in s_1} \cdots \sum_{k_n \in s_n} 1 $ \\
    \\
    %
    R-SUM-ELIM5
    & $ \sum_{i \in M} \sum_{j \in M} \sum_{k_1 \in s_1} \cdots \sum_{k_n \in s_n} (a_1 \times \dots \times \delta_{i, j} \times \cdots \times a_n) $ \\
    & $ \reduce\ \sum_{j \in M} \sum_{k_1 \in s_1} \cdots \sum_{k_n \in s_n} (a_1\{j/i\} \times \cdots \times a_n\{j/i\}) $ \\
    \\
    %
    R-SUM-ELIM6
    & $ \sum_{i \in M} \sum_{j \in M} \sum_{k_1 \in s_1} \cdots \sum_{k_n \in s_n} (\delta_{i, j}.A) $ \\
    & $ \reduce\ \sum_{j \in M} \sum_{k_1 \in s_1} \cdots \sum_{k_n \in s_n} A\{j/i\} $ \\
    \\
    %
    R-SUM-ELIM7
    & $ \sum_{i \in M} \sum_{j \in M} \sum_{k_1 \in s_1} \cdots \sum_{k_n \in s_n} (a_1 \times \cdots \times \delta_{i, j} \times \cdots \times a_n).A $ \\
    & $ \reduce\ \sum_{j \in M} \sum_{k_1 \in s_1} \cdots \sum_{k_n \in s_n} (a_1\{j/i\} \times \cdots \times a_n\{j/i\}).A\{j/i\} $ \\
    \\
    %
    R-SUM-ELIM8
    & $ \sum_{i \in M} \sum_{j \in M} \sum_{k_1 \in s_1} \cdots \sum_{k_n \in s_n} ((a_1 \times \cdots \times \delta_{i, j} \times \cdots \times a_n) + $ \\
    & $ \cdots + (b_1 \times \cdots \times \delta_{i, j} \times \cdots \times b_n)).A $ \\
    & $ \reduce\ \sum_{j \in M} \sum_{k_1 \in s_1} \cdots \sum_{k_n \in s_n} ((a_1\{j/i\} \times \cdots \times a_n\{j/i\}) + $ \\
    & $ \cdots + (b_1\{j/i\} \times \cdots \times b_n\{j/i\})).A\{j/i\} $ \\
\end{ruletable}

\begin{ruletable}{Rules for pushing terms into sum operators. Because we apply type checking on variables, and stick to unique bound variables, these operations are always sound.}
    %
    R-SUM-PUSH0
    & $ b_1 \times \cdots \times (\sum_{i \in M} a) \times \cdots \times b_n$ \\
    & $\reduce\ \sum_{i \in M} (b_1 \times \cdots \times a \times \cdots \times b_n) $ \\
    %
    R-SUM-PUSH1
    & $ (\sum_{i \in M} a)^* \ \reduce\ \sum_{i \in M} a^* $ \\
    %
    R-SUM-PUSH2
    & $ (\sum_{i \in M} X)^\dagger \ \reduce\ \sum_{i \in M} X^\dagger $ \\
    %
    R-SUM-PUSH3
    & $ a.(\sum_{i \in M} X) \ \reduce\ \sum_{i \in M} (a.X) $ \\
    %
    R-SUM-PUSH4
    & $ (\sum_{i \in M} a).X \ \reduce\ \sum_{i \in M} (a.X) $ \\
    %
    R-SUM-PUSH5
    & $ (\sum_{i \in M} B)\cdot K \ \reduce\ \sum_{i \in M}(B \cdot K) $ \\
    %
    R-SUM-PUSH6
    & $ (\sum_{i \in M} O)\cdot K \ \reduce\ \sum_{i \in M}(O \cdot K) $ \\
    %
    R-SUM-PUSH7
    & $ (\sum_{i \in M} B)\cdot O \ \reduce\ \sum_{i \in M}(B \cdot O) $ \\
    %
    R-SUM-PUSH8
    & $ (\sum_{i \in M} K)\cdot B \ \reduce\ \sum_{i \in M}(K \cdot B) $ \\
    %
    R-SUM-PUSH9
    & $ (\sum_{i \in M} O_1)\cdot O_2 \ \reduce\ \sum_{i \in M}(O_1 \cdot O_2) $ \\
    %
    R-SUM-PUSH10
    & $ B \cdot (\sum_{i \in M} K) \ \reduce\ \sum_{i \in M}(B \cdot K) $ \\
    %
    R-SUM-PUSH11
    & $ O \cdot (\sum_{i \in M} K) \ \reduce\ \sum_{i \in M}(O \cdot K) $ \\
    %
    R-SUM-PUSH12
    & $ B \cdot (\sum_{i \in M} O) \ \reduce\ \sum_{i \in M}(B \cdot O) $ \\
    %
    R-SUM-PUSH13
    & $ K \cdot (\sum_{i \in M} B) \ \reduce\ \sum_{i \in M}(K \cdot B) $ \\
    %
    R-SUM-PUSH14
    & $ O_1 \cdot (\sum_{i \in M} O_2) \ \reduce\ \sum_{i \in M}(O_1 \cdot O_2) $ \\
    %
    R-SUM-PUSH15
    & $ (\sum_{i \in M} X_1) \otimes X_2 \ \reduce\ \sum_{i \in M} (X_1 \otimes X_2) $ \\
    %
    R-SUM-PUSH16
    & $ X_1 \otimes (\sum_{i \in M} X_2) \ \reduce\ \sum_{i \in M} (X_1 \otimes X_2) $
\end{ruletable}


\begin{ruletable}{Rules for addition and index in sum.}
    %
    R-SUM-ADDS0
    & $\sum_{i \in M}(a_1 + \cdots + a_n) \ \reduce\ (\sum_{i \in M} a_1) + \cdots + (\sum_{i \in M} a_n) $ \\
    % %
    % R-SUM-ADDS1
    % & $\sum_{i \in M}(b_1 \times \cdots \times (a_1 + \cdots + a_n) \times \cdots \times b_m) $ \\
    % & $ \reduce\ (\sum_{i \in M} (b_1 \times \cdots \times a_1 \times \cdots \times b_m) + \cdots + (\sum_{i \in M} (b_1 \times \cdots \times a_n \times \cdots \times b_m)) $ \\
    %
    R-SUM-ADD0
    & $\sum_{i \in M}(X_1 + \cdots + X_n) \ \reduce\ (\sum_{i \in M} X_1) + \cdots + (\sum_{i \in M} X_n) $ \\
    %
    R-SUM-INDEX0
    & $ \sum_{i \in \mathbf{U}(\sigma \times \tau)} A \ \reduce\ \sum_{j \in \mathbf{U}(\sigma)} \sum_{k \in \mathbf{U}(\tau)} A\{i/(j, k)\} $ \\
    %
    R-SUM-INDEX1
    & $ \sum_{i \in M_1 \star M_2} A \ \reduce\ \sum_{j \in M_1} \sum_{k \in M_2} A\{i/(j, k)\} $
\end{ruletable}


\begin{ruletable}{Rules for bool index.}
    %
    R-BIT-DELTA
    & $\delta_{0, 1} \ \reduce\ 0$ \\
    %
    R-BIT-ONEO
    & $\ONEO(\textsf{bool})\ \reduce\ \ket{0}\bra{0} + \ket{1}\bra{1} $ \\
    %
    R-BIT-SUM
    & $\sum_{i \in \mathbf{U}(\textsf{bool})} A\ \reduce\ A\{i/0\} + A\{i/1\}$
\end{ruletable}

\begin{ruletable}{Rules about addition and sum.}
    R-MULS2
    & $b_1 \times \cdots \times (a_1 + \cdots + a_n) \times \cdots \times b_m$ \\
    & $\reduce\ (b_1 \times \cdots \times a_1 \times \cdots \times b_m) + \cdots + (b_1 \times \cdots \times a_n \times \cdots \times b_m)$ \\
    \\
    %
    R-SUM-ADD1
    & $ Y_1 + \cdots + Y_n + \sum_{i \in M}(a+b).X$ \\
    & $ \reduce\ Y_1 + \cdots + \sum_{i \in M}(a.X) + \cdots + \sum_{i \in M}(b.X) + Y_n$ \\
    \\
    %
    R-SUM-FACTOR
    & $X_1 + \cdots + (\sum_{k_1 \in s_1}\cdots\sum_{k_n \in s_n}A) $ \\
    & $ + (\sum_{k_1 \in s_1}\cdots\sum_{k_n \in s_n}A) + \cdots +X_n$ \\
    & $X_1 + \cdots + (\sum_{k_1 \in s_1}\cdots\sum_{k_n \in s_n}(1+1).A) + \cdots +X_n$ \\
    \\
    & $X_1 + \cdots + (\sum_{k_1 \in s_1}\cdots\sum_{k_n \in s_n}a.A) $ \\
    & $ + (\sum_{k_1 \in s_1}\cdots\sum_{k_n \in s_n}A) + \cdots +X_n$ \\
    & $X_1 + \cdots + (\sum_{k_1 \in s_1}\cdots\sum_{k_n \in s_n}(a+1).A) + \cdots +X_n$ \\
    \\
    & $X_1 + \cdots + (\sum_{k_1 \in s_1}\cdots\sum_{k_n \in s_n}a.A) $ \\
    & $ + (\sum_{k_1 \in s_1}\cdots\sum_{k_n \in s_n}b.A) + \cdots +X_n$ \\
    & $X_1 + \cdots + (\sum_{k_1 \in s_1}\cdots\sum_{k_n \in s_n}(a+b).A) + \cdots +X_n$ \\
\end{ruletable}

\begin{ruletable}{Rules to eliminate labels in Dirac notations.}
    %
    R-L-EXPAND
    & $K_R \ \reduce\ \sum_{i_{r_1}\in\bU(\sigma_{r_1})}\cdots \sum_{i_{r_n}\in\bU(\sigma_{r_n})} (\<i_R|\cdot K). (|i_{r_1}\>_{r_i}\otimes\cdots\otimes|i_{r_n}\>_{r_n})$ \\
    \\
    %
    & $B_R \ \reduce\ \sum_{i_{r_1}\in\bU(\sigma_{r_1})}\cdots \sum_{i_{r_n}\in\bU(\sigma_{r_n})} (B\cdot |i_R\>). ({}_{r_1}\<i_{r_1}|\otimes\cdots\otimes{}_{r_n}\<i_{r_n}|)$\\
    \\
    %
    & $O_{R,R'} \ \reduce\ \sum_{i_{r_1}\in\bU(\sigma_{r_1})}\cdots \sum_{i_{r_n}\in\bU(\sigma_{r_n})}
    \sum_{i_{r'_1}\in\bU(\sigma_{r'_1})}\cdots \sum_{i_{r'_{n'}}\in\bU(\sigma_{r'_{n'}})}$ \\
    & $(\<i_R|\cdot O\cdot |i_{R'}\>).(|i_{r_1}\>_{r_i}\otimes\cdots\otimes|i_{r_n}\>_{r_n} \otimes {}_{r'_1}\<i_{r'_1}|\otimes\cdots\otimes{}_{r'_{n'}}\<i_{r'_{n'}}|)$
\end{ruletable}



\begin{ruletable}{Rules for labelled Dirac notations.}
    %
    R-ADJDK
    & $ ({}_r\<i|)^\dagger \ \reduce\ |i\>_r$ \\
    %
    R-ADJDB
    & $ (|i\>_r)^\dagger \ \reduce\ {}_r\<i|$ \\
    R-ADJD0
    & $ (D_1 \otimes \cdots \otimes D_n)^\dagger \ \reduce\ D_1^\dagger \otimes \cdots \otimes D_n^\dagger$ \\
    %
    R-ADJD1
    & $ (D_1\cdot D_2)^\dagger \ \reduce\ D_2^\dagger \cdot D_1^\dagger$ \\
    %
    R-SCRD0
    & $ D_1 \otimes \cdots \otimes (a.D_n) \otimes \cdots \otimes D_m \ \reduce\ a.(D_1 \otimes \cdots \otimes D_m) $ \\
    %
    R-SCRD1
    & $ (a.D_1) \cdot D_2 \ \reduce\ a.(D_1 \cdot D_2) $ \\
    %
    R-SCRD2
    & $ D_1 \cdot (a.D_2) \ \reduce\ a.(D_1 \cdot D_2) $ \\
    %
    R-TSRD0
    & $ X_1 \otimes \cdots \otimes (D_1 + \cdots + D_n) \otimes \cdots X_m$ \\
    & $ \reduce\ X_1 \otimes \cdots D_1 \cdots \otimes X_m + \cdots + X_1 \otimes \cdots D_n  \cdots \otimes X_m $ \\
    %
    R-DOTD0
    & $ (D_1 + \cdots + D_n) \cdot D\ \reduce\ D_1 \cdot D + \cdots + D_n \cdot D $ \\
    %
    R-DOTD1
    & $ D \cdot (D_1 + \cdots + D_n)\ \reduce\ D \cdot D_1 + \cdots + D \cdot D_n $ \\
    %
    R-SUM-PUSHD0
    & $ X_1 \otimes \cdots (\sum_{i \in M} D) \cdots \otimes X_2\ \reduce\ \sum_{i \in M} (X_1 \otimes \cdots D \cdots \otimes X_n) $ \\
    %
    R-SUM-PUSHD1
    & $ (\sum_{i \in M} D_1) \cdot D_2 \ \reduce\ \sum_{i \in M} (D_1 \cdot D_2) $ \\
    %
    R-SUM-PUSHD2
    & $ D_1 \cdot (\sum_{i \in M} D_2) \ \reduce\ \sum_{i \in M} (D_1 \cdot D_2) $
\end{ruletable}

\begin{ruletable} {Rules to simplify dot product in labelled Dirac notations.}
    %
    R-L-SORT0
    & $ A : \DType(s_1, s_2), B : \DType(s_1', s_2'), s_2 \cap s_1'=\emptyset \Rightarrow A \cdot B \ \reduce\ A \otimes B $ \\
    %
    R-L-SORT1
    & ${}_r\bra{i}\cdot\ket{j}_r \ \reduce\ \delta_{i, j}$ \\
    %
    R-L-SORT2
    & ${}_r\bra{i}\cdot(Y_1 \otimes \cdots \otimes \ket{j}_r \otimes \cdots \otimes Y_m) \ \reduce\ \delta_{i, j}.(Y_1  \otimes \cdots \otimes Y_m)$ \\
    %
    R-L-SORT3
    & $(X_1 \otimes \cdots \otimes {}_r\bra{i} \otimes \cdots \otimes X_n) \cdot \ket{j}_r \ \reduce\ \delta_{i,j}.(X_1 \otimes \cdots \otimes X_n)$ \\
    %
    R-L-SORT1
    & $ (X_1 \otimes \cdots \otimes {}_r\bra{i} \otimes \cdots \otimes X_n) \cdot (Y_1 \otimes \cdots \otimes \ket{j}_r \otimes \cdots \otimes Y_m) $ \\
    & $\reduce\ \delta_{i,j}.(X_1 \otimes \cdots \otimes X_n) \cdot (Y_1 \otimes \cdots \otimes Y_m)$
\end{ruletable}

\end{document}