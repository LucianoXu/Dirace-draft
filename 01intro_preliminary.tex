
\section{Introduction}


Dirac notation~\cite{dirac1939new}, a.k.a.\, bra-ket notation, is a
mathematical formalism for representing quantum states using linear
algebra notation: for instance, Dirac notation uses the linear
combination \( a\ket{\psi} + b\ket{\phi} \) to represent the
superposition of the two quantum states \( \ket{\psi} \) and
\( \ket{\phi} \). Another essential ingredient of Dirac notation
is the tensor product $\otimes$, which is used to describe composite
states: for instance, Dirac notation uses the tensor expression
\( \ket{\psi} \otimes \ket{\phi} \) to denote the composition of
the two quantum states \( \ket{\psi} \) and \( \ket{\phi} \). It is
also common to use a variant of Dirac notation, called labelled Dirac
notation, to describe composite quantum states. In labelled Dirac
notation, bras and kets are tagged with labels that identify the
subsystems on which they operate. For instance, the labelled tensor
expression
\( \ket{\psi}_{S'} \otimes \ket{\phi}_{S} \)
states that $\ket{\psi}_S$ and $\ket{\psi}_{S'}$ describe two quantum
states over some subsystem $S$ and $S'$ respectively. By considering
the relationship between $S$ and $S'$, one can obtain identities for
free, e.g.\, 
%
$$\ket{\phi}_S \otimes \ket{\psi}_{T} = \ket{\psi}_{T} \otimes \ket{\phi}_S
\mbox{if}~S\cap S'=\emptyset$$
%
In turn, commutativity of the tensor product ensures that one can
reason locally about quantum systems, and contributes to making
labelled Dirac notation a convenient, compositional formalism to
reason about quantum states---in a way that is similar to the way
bunched logics support compositional reasoning about mutable states.

Labelled Dirac notation is also commonly used to express assertions
about quantum programs. Specifically, many quantum Hoare logics use
labelled Dirac notation to express program assertions~\cite{DBLP:conf/lics/ZhouBHYY21,Zhou2023}. These logics also use
implicitly equational reasoning between labelled Dirac expressions, to
glue applications of proof rules---similar to the use of the rule of
consequence in the setting of classical program. Therefore, it is
essential for the verification of quantum programs to have automated
means of proving equality of two complex expressions based on labelled
Dirac notation.


\subsubsection*{Contributions}
This paper presents an automated tool, called D-Hammer, for reasoning
about labelled Dirac notation. D-Hammer uses a rich, dependently typed
language, to formalize labelled Dirac notation, and supports common
idioms used to describe quantum systems, including big operators of
the form $\sum_{i\in I} a_i \ket{\phi_i}_S$ to represent indexed
superpositions of states. The semantics of typable expressions is
given in terms of Hilbert spaces, tailored to interpreting tensor
product as an AC symbol. We leverage this interpretation to define a
rich equational theory for labelled Dirac notation, and prove its
soundness with respect to our denotational semantics. Finally, we
define an efficient normalization procedure to prove equivalence of
two expressions. We then evaluate our procedure with respect to
examples from the literature. Our evaluation covers examples based on
(vanilla, unlabelled) Dirac notation, as well as examples based on
labelled Dirac notation. The main conclusions are that our approach
outperforms prior work by~\cite{diracdec} and is able to prove complex
examples of reasoning with labelled Dirac notation from the
literature, including examples from prior work on quantum separation
logic~\cite{DBLP:conf/lics/ZhouBHYY21}.



%% In 1939, Dirac introduced his notation for quantum mechanics~\cite{dirac1939new}, designed to represent linear algebraic formulas in a compact and convenient form. For example, the expression \( a\ket{\psi} + b\ket{\phi} \) represents the superposition of two quantum states, \( \ket{\psi} \) and \( \ket{\phi} \). Today, Dirac notation is widely accepted as the standard language in quantum computation and quantum information. Its reasoning forms the foundation of research and applications, much like boolean and integer logic do in classical computer science.

%% In quantum algorithm formalizations and quantum programming languages, Dirac notation is frequently used in equational proofs, which are critical, repetitive, and often time-intensive. These notations also play a key role in defining program states, operations, and assertions in quantum programming languages. To automate the verification of these programs, we need tools that can simplify and check the equivalence of preconditions. However, unlike the well-established SAT and SMT solvers for classical logic, a practical solver for Dirac notation equivalence remains an unmet need, creating a barrier for progress in several fields.

%% Recently, Xu et al.~\cite{diracdec} proposed a theory for deciding the equivalence of Dirac notation, alongside a prototype implementation in Mathematica called DiracDec. They demonstrated that the equivalence of basic Dirac notation is decidable. Their algorithm, based on a pure term rewriting system, has been proven to be confluent and terminating. Despite this, there remains a gap between DiracDec and a practical solver for Dirac notation equivalence.

%% One challenge is the efficiency of the algorithm when dealing with equivalences beyond the scope of term rewriting. DiracDec decides the entire equational theory by rewriting modulo \( E \), where \( E \) represents a set of axioms that cannot be decided by normalization alone, such as the associativity and commutativity (AC) of certain function symbols. DiracDec uses a direct but inefficient algorithm to decide these axioms, which searches through all possible permutations and exhibits factorial complexity. This inefficiency becomes particularly evident in "computational examples" containing many AC symbols, which are time-consuming to process.

%% Usability is another area where DiracDec falls short. It does not support labelled Dirac notation, which uses registers to denote subsystems and express states locally. Additionally, DiracDec's typing system only does not provide context for variable typing assumptions or the definition of symbols, which are required in practical scenarios. To avoid type checking during term rewriting, DiracDec separates symbols (e.g., multiplication) for different types, leading to unnecessary complexity. Moreover, integrating the Mathematica implementation into other projects as a solver is challenging.

%% The system design of DiracDec reflects a trade-off between simplicity and efficiency. While the simplicity of term rewriting allows for strong theoretical results, it limits optimization opportunities. Building on DiracDec, this work aims to develop a practical solver. We transform the term rewriting system into a hybrid algorithm, overcoming the challenges mentioned above. Our main technical contributions are:
%% \begin{itemize}
%%     \item An efficient algorithm for deciding the equational theories in \( E \), based on equivalence checking through normalization, with the normal form for \( E \) being obtained via sorting.
%%     \item Support for constant register labels, and reducing the equivalence decision of labelled Dirac notation to the unlabelled case.
%%     \item A more user-friendly \CC\ solver, D-Hammer, featuring an abstract language and typing system. We also support the definition of symbols (e.g., transpose and trace) using function syntax.
%% \end{itemize}

%% We evaluated D-Hammer against the DiracDec benchmark and new examples involving labelled Dirac notation. The results show significant improvements in both decidability and efficiency compared to DiracDec.

%% % D-Hammer successfully decides all the examples that are expressable in its language, including those failed by DiracDec because of complexity or insufficient decision power.


\section{Motivation and Preliminaries}
\subsection{Standard Dirac Notation}

Dirac notation, also known as bra-ket notation, provides an intuitive
and concise mathematical framework for describing quantum states and
operations in quantum mechanics.  We write $\cH$ for a Hilbert space,
i.e., a vector space equipped with an standard inner product
$\langle\bu,\bv\rangle \in \bC$ for $\bu,\bv\in\cH$.  Dirac notation
consists of the following components that reflect the basic postulates
of quantum mechanics:
\begin{itemize}
  \item Ket $|u\rangle$ is a column vector that denotes a quantum state $\bu$ in the state Hilbert space $\cH$. For example, the computational bases of qubit system are commonly written as $|0\rangle = \matrix{1 \\ 0}$ and $|1\rangle = \matrix{0 \\ 1}$.
  \item Bra $\langle u|$ is a row vector, the conjugate transpose of $|u\rangle$, that denotes the dual state of $|u\rangle$. 
  %It is alternative to interpret as a linear mapping from $\cH$ to $\bC$ defined as $\langle u| : |v\rangle \mapsto \langle \bu,\bv\rangle$. It is the conjugate transpose of $|u\rangle$.
  For example, $\langle 0| = \matrix{1 & 0}$ and $\langle 1| = \matrix{0 & 1}$.
  \item Inner product $\langle u|v\rangle \triangleq \langle \bu,\bv\rangle$ which indicates the probability amplitude for $|u\rangle$ to collapse into $|v\rangle$. 
  By convention, it is computed by matrix multiplication of two states, e.g., 
  $\langle 0|1\rangle = \matrix{1 & 0} \matrix{0 \\ 1} = 0.$
  \item Outer product $|u\rangle\langle v| \triangleq |w\rangle \mapsto (\langle v|w\rangle) |u\rangle$. Any linear map, such as unitary tranformation, measurement operator, etc, can be decomposed as the sum of outer products. It is also computed by matrix multiplication, e.g., 
  $|1\rangle\langle 0| = \matrix{0 \\ 1} \matrix{1 & 0}  = \matrix{0 & 0 \\ 1 & 0}.$
  \item Tensor product $|u\rangle\otimes|v\rangle$ (or simply $|u\rangle|v\rangle$ or $|uv\rangle$), $\langle u|\otimes\langle v|$ (or simply $\langle u|\langle v|$ or $\langle uv|$) for describing the state, dual state and linear map of composite systems respectively. It is computed by the the Kronecker product of matrices, e.g., $\langle 0|\langle 1| = {\color{blue}\matrix{1 & 0}}\otimes {\color{red}\matrix{1 & 0}} = \matrix{({\color{blue}1}*{\color{red}1}) & ({\color{blue}1}*{\color{red}0}) & ({\color{blue}0}*{\color{red}1}) & ({\color{blue}0}*{\color{red}0})} = \matrix{1&0&0&0}$.
\end{itemize}

\subsection{Labelled Dirac Notation and Motivating Example}
Labelled Dirac notation is a generalization of Dirac notation for
describing multi-body quantum systems. The need for labelled Dirac
notation is motivated by the following example:
\begin{example}
  \label{example1}
  Let $p,q,r$ be three qubits and initially in the (unnormalized)
  $\ghz$ state $|\ghz\rangle \triangleq |000\rangle+|111\rangle$.
  Applying the 3-qubit Toffoli gate ($\ccnot$) with control qubits
  $p,r$ and target qubit $q$ to GHZ is equivalent to applying 2-qubit
  $\cnot$ gate with control qubit $r$ or $p$ and target qubit $q$.
  Using Dirac notation, the identity is written as:
\begin{align*}
  \ccnot|\ghz\rangle &= (I\otimes \cnot)|\ghz\rangle \\
  &= (\swap\otimes I)(I\otimes \cnot)(\swap\otimes I)|\ghz\rangle
\end{align*}
Formalizing the statement in Dirac notation requires the following
steps: 1. Arrange qubits in the conventional order, here we choose
$p,r,q$ to simplify the representation of $\ccnot$; 2. Lift the local
operation $\cnot$ to the global system. For $\cnot$ acting on $r,q$ is
easier, since $r,q$ consistent with the chosen order, and we only need
to tensoring it by an identity operator $I$ on $p$, i.e., $(I\otimes
\cnot)$.  For $\cnot$ acting on $p,q$, realize that $p,q$ is not
adjacent in the chosen order, we additionally need to use $\swap$ to
temporarily exchange the qubits $p$ and $r$, i.e., globally, the
$(\swap\otimes I)$ before and after $(I\otimes \cnot)$ (lifting of
$\cnot$ on $r,q$).



Roughly speaking, encoding in standard Dirac notation needs to
tensoring identity operators and employing additional $\swap$ gates
since generally the conventional order cannot guarantee 1. the order
of all local operations is consistent with it, 2. the local operations
involve only adjacent qubits in the conventional order.
\end{example}
To address the limitations of Dirac notations, physicists routinely
use labels (or subscripts) to indicate the systems on which quantum
states or operations are applied, thereby avoiding unnecessary lifting
and swap gates. For example, rewriting the above example using labels
yields:
\begin{align*}
  \ccnot_{prq}|\ghz\rangle_{pqr} = 
  \cnot_{rq}|\ghz\rangle_{pqr}
  = \cnot_{pq}|\ghz\rangle_{pqr}
\end{align*}
This formalization avoids determining and maintaining the conventional
order of qubits. nor lifting using additional $I$ and $\swap$s. In
this setting, the tensor products become an associative and
commutative (AC) symbol, so we can rearrange qubits as needed to
perform calculations. For our example, we can perform the calculation as
follows:
\begin{align*}
  &\ccnot_{prq}|\ghz\rangle_{pqr}
  = (\ccnot|000\rangle + \ccnot|111\rangle)_{prq}
  = (|000\rangle + |110\rangle)_{prq}\\
  &\cnot_{rq}|\ghz\rangle_{pqr}
  = (\cnot|00\rangle)_{rq}|0\rangle_p + (\cnot|11\rangle)_{rq}|1\rangle_p
  =(|000\rangle + |101\rangle)_{rqp}\\
  &\cnot_{pq}|\ghz\rangle_{pqr}
  = (\cnot|00\rangle)_{pq}|0\rangle_r + (\cnot|11\rangle)_{pq}|1\rangle_r
  = (|000\rangle + |101\rangle)_{pqr}
\end{align*}
where the RHS of each line are equivalent.  In addition, labelled
Dirac notation can conveniently describe local measurements, partial
traces (the state or evolution of subsystems in multi-body systems),
and partial inner products (the representation of partial traces in
pure states), which are sufficient to handle the mathematical formulas
of quantum mechanics in multi-body systems.

Labelled Dirac notation is not only ubiquitous in the description of multi-body systems, but also plays an important role in program logics. Just as in classical program logic, variable names are used to construct logical formulas instead of describing them as global memory functions, in quantum program logic, variable names are used to label the subsystems on which quantum gates act, rather than lifting them to the global system. Indeed, the latter would cause exponential length of the formula w.r.t the number of variables, as discussed in [cite].

The next sections introduce the formal language, labelled Dirac notation and the systematical way to reduce labels and prove equalities. 
For this purpose, we use the following motivating example as the main role in our explanations.
Instead of GHZ states, it is a similar but simpler example on Bell states:
\begin{example}
    \label{ex: motivating}
    Let \( q \) and \( r \) represent two quantum systems in the Hilbert space \( \mathcal{H}_T \). Let \( M \) be a quantum operation acting on \( \mathcal{H}_T \), and let \( \ket{\Phi} = \sum_{i \in T} \ket{i} \otimes \ket{i} \) be the maximally entangled state. Then, it holds that
    \[
    M_q \ket{\Phi}_{(q, r)} = M_r^T \ket{\Phi}_{(q, r)}.
    \]
\end{example}
Following the introduction on labels, we can consider the global system $(q,r)$ and transform the equation above into the plain Dirac notation:
\begin{align}
    (M \otimes I) \ket{\Phi} = (I \otimes M^T) \ket{\Phi}.
    \label{eq: motivating plain}
\end{align}
Together with corresponding explanations for this example, we show how an automated system is built and used to solve equalities likewise.

% We will use the following
% generalization of Example \ref{example1} as motivating example:
% \begin{example}[Apply multiplexer to $\ghz$ state]
%   \label{ex: motivating}
%   Let $p,q,r$ be three quantum systems with the same Hilbert space \( \mathcal{H}_T \) (i.e., with computational bases $\{|i\>\}_{i\in T}$) and GHZ state $|\ghz\> = \sum_i|iii\>\<iii|$. Consider the linear operator sets $U_{ij}$ and the corresponding multiplexers $M\triangleq \sum_{ij}|ij\>\<ij|\otimes U_{ij}$ and $N\triangleq \sum_i|i\>\<i|\otimes U_{ii}$. We have:
%   $$M_{prq}|\ghz\>_{prq} = N_{rq}|\ghz\>_{pqr} = N_{pq}|\ghz\>_{pqr}.$$
% \end{example}

% An interesting property of quantum mechanics is that for two maximally entangled states, applying a quantum operator \( M \) to one subsystem is equivalent to applying \( M^T \) (the transpose of \( M \)) to the other subsystem. This relationship holds regardless of the spatial separation between the two systems, and it can be expressed as an equation in labelled Dirac notation.
% \begin{example}
%     \label{ex: motivating}
%     Let \( q \) and \( r \) represent two quantum systems in the Hilbert space \( \mathcal{H}_T \). Let \( M \) be a quantum operation acting on \( \mathcal{H}_T \), and let \( \ket{\Phi} = \sum_{i \in T} \ket{i} \otimes \ket{i} \) be the maximally entangled state. Then, we have the following equation:
%     \[
%     M_q \ket{\Phi}_{(q, r)} = M_r^T \ket{\Phi}_{(q, r)}.
%     \]
% \end{example}

% A little explanation on the notation and terms.
% Quantum states are represented as vectors in complex Hilbert spaces, and operations on these states are described by linear transformations, or operators. Dirac notation uses the ket \( \ket{i} \) and the bra \( \bra{i} \) to denote basis vectors in a Hilbert space and its dual space, respectively. These symbols can be composed together in various ways to form more complex expressions.
% For example, $M \ket{i}$ represents the operator-vector multiplication, and the tensor product \(\ket{i}\otimes\ket{i}\) describes the product state on multiple quantum systems.



% In this equation, \( q \) and \( r \) are labels denoting the respective subsystems in the Dirac notation. $M_q$ means that the operation $M$ is applied on system $q$, and $\ket{\Phi}_{(q, r)}$ denotes the quantum state $\ket{\Phi}$ on the (ordered) composed system $(q, r)$.
% To reason about their equivalence, we can remove the labels by extending them to the global system $(q, r)$:
% \begin{align}
%     (M \otimes I) \ket{\Phi} = (I \otimes M^T) \ket{\Phi}.
%     \label{eq: motivating unlabelled}
% \end{align}
% Here the identity operator $I$ are inserted in suitable positions to extend the operator $M$, a special case of extending to the global system called cylinder extension.

% \subsubsection{Universal Algebra}
% We use universal algebra and equational logic to formally represent Dirac notation and the reasoning procedure. A universal algebra defines a signature of function symbols, with terms constructed from constants, variables, and function applications. 
% Other basic concepts like substitution of variables or pattern matching are also defined.
% In our case of Dirac notation, the signature consists of constructors and operations like $\ket{i}$ or $A \otimes B$.
% The reasoning process is guided by equational logic, which defines an equivalence relation that is compatible with substitution and term construction. This relation formalizes the intuitive concept of equivalence in algebra.

