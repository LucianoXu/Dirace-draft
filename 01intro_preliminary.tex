
\section{Introduction}


Dirac notation~\cite{dirac1939new}, a.k.a.\, bra-ket notation, is a
mathematical formalism for representing quantum states using linear
algebra notation: for instance, Dirac notation uses the linear
combination \( a\ket{\psi} + b\ket{\phi} \) to represent the
superposition of the two quantum states \( \ket{\psi} \) and
\( \ket{\phi} \). Another essential ingredient of Dirac notation
is the tensor product $\otimes$, which is used to describe composite
states: for instance, Dirac notation uses the tensor expression
\( \ket{\psi} \otimes \ket{\phi} \) to denote the composition of
the two quantum states \( \ket{\psi} \) and \( \ket{\phi} \). It is
also common to use a variant of Dirac notation, called labelled Dirac
notation, to describe composite quantum states. In labelled Dirac
notation, bras and kets are tagged with labels that identify the
subsystems on which they operate. For instance, the labelled tensor
expression
\( \ket{\psi}_{S'} \otimes \ket{\phi}_{S} \)
states that $\ket{\psi}_S$ and $\ket{\psi}_{S'}$ describe two quantum
states over some subsystem $S$ and $S'$ respectively. By considering
the relationship between $S$ and $S'$, one can obtain identities for
free, e.g.\, 
%
$$\ket{\phi}_S \otimes \ket{\psi}_{T} = \ket{\psi}_{T} \otimes \ket{\phi}_S
\mbox{if}~S\cap S'=\emptyset$$
%
In turn, commutativity of the tensor product ensures that one can
reason locally about quantum systems, and contributes to making
labelled Dirac notation a convenient, compositional formalism to
reason about quantum states---in a way that is similar to the way
bunched logics support compositional reasoning about mutable states.

Labelled Dirac notation is also commonly used to express assertions
about quantum programs. Specifically, many quantum Hoare logics use
labelled Dirac notation to express program assertions, including
pre-conditions and post-conditions~\cite{TODO}. These logics also use
implicitly equational reasoning between labelled Dirac expressions, to
glue applications of proof rules---similar to the use of the rule of
consequence in the setting of classical program. Therefore, it is
essential for the verification of quantum programs to have automated
means of proving equality of two complex expressions based on labelled
Dirac notation.


\subsubsection*{Contributions}
This paper presents an automated tool, called D-Hammer, for reasoning
about labelled Dirac notation. D-Hammer uses a rich, dependently typed
language, to formalize labelled Dirac notation, and supports common
idioms used to describe quantum systems, including big operators of
the form $\sum_{i\in I} a_i \ket{\phi_i}_S$ to represent indexed
superpositions of states. The semantics of typable expressions is
given in terms of Hilbert spaces, tailored to interpreting tensor
product as an associative and commutative operator. We leverage this
interpretation to define a rich equational theory for labelled Dirac
notation, and prove its soundness with respect to our denotational
semantics. Finally, we define an efficient normalization procedure to
prove equivalence of two expressions. We then evaluate our procedure
with respect to examples from the literature. Our evaluation covers
examples based on (vanilla, unlabelled) Dirac notation, as well as
examples based on labelled Dirac notation. The main conclusions are
that our approach outperforms prior work by~\cite{diracdec} and is
able to prove complex examples of reasoning with labelled Dirac
notation from the literature, including examples from prior work on
quantum separation logic~\cite{DBLP:conf/lics/ZhouBHYY21}.



%% In 1939, Dirac introduced his notation for quantum mechanics~\cite{dirac1939new}, designed to represent linear algebraic formulas in a compact and convenient form. For example, the expression \( a\ket{\psi} + b\ket{\phi} \) represents the superposition of two quantum states, \( \ket{\psi} \) and \( \ket{\phi} \). Today, Dirac notation is widely accepted as the standard language in quantum computation and quantum information. Its reasoning forms the foundation of research and applications, much like boolean and integer logic do in classical computer science.

%% In quantum algorithm formalizations and quantum programming languages, Dirac notation is frequently used in equational proofs, which are critical, repetitive, and often time-intensive. These notations also play a key role in defining program states, operations, and assertions in quantum programming languages. To automate the verification of these programs, we need tools that can simplify and check the equivalence of preconditions. However, unlike the well-established SAT and SMT solvers for classical logic, a practical solver for Dirac notation equivalence remains an unmet need, creating a barrier for progress in several fields.

%% Recently, Xu et al.~\cite{diracdec} proposed a theory for deciding the equivalence of Dirac notation, alongside a prototype implementation in Mathematica called DiracDec. They demonstrated that the equivalence of basic Dirac notation is decidable. Their algorithm, based on a pure term rewriting system, has been proven to be confluent and terminating. Despite this, there remains a gap between DiracDec and a practical solver for Dirac notation equivalence.

%% One challenge is the efficiency of the algorithm when dealing with equivalences beyond the scope of term rewriting. DiracDec decides the entire equational theory by rewriting modulo \( E \), where \( E \) represents a set of axioms that cannot be decided by normalization alone, such as the associativity and commutativity (AC) of certain function symbols. DiracDec uses a direct but inefficient algorithm to decide these axioms, which searches through all possible permutations and exhibits factorial complexity. This inefficiency becomes particularly evident in "computational examples" containing many AC symbols, which are time-consuming to process.

%% Usability is another area where DiracDec falls short. It does not support labelled Dirac notation, which uses registers to denote subsystems and express states locally. Additionally, DiracDec's typing system only does not provide context for variable typing assumptions or the definition of symbols, which are required in practical scenarios. To avoid type checking during term rewriting, DiracDec separates symbols (e.g., multiplication) for different types, leading to unnecessary complexity. Moreover, integrating the Mathematica implementation into other projects as a solver is challenging.

%% The system design of DiracDec reflects a trade-off between simplicity and efficiency. While the simplicity of term rewriting allows for strong theoretical results, it limits optimization opportunities. Building on DiracDec, this work aims to develop a practical solver. We transform the term rewriting system into a hybrid algorithm, overcoming the challenges mentioned above. Our main technical contributions are:
%% \begin{itemize}
%%     \item An efficient algorithm for deciding the equational theories in \( E \), based on equivalence checking through normalization, with the normal form for \( E \) being obtained via sorting.
%%     \item Support for constant register labels, and reducing the equivalence decision of labelled Dirac notation to the unlabelled case.
%%     \item A more user-friendly \CC\ solver, D-Hammer, featuring an abstract language and typing system. We also support the definition of symbols (e.g., transpose and trace) using function syntax.
%% \end{itemize}

%% We evaluated D-Hammer against the DiracDec benchmark and new examples involving labelled Dirac notation. The results show significant improvements in both decidability and efficiency compared to DiracDec.

%% % D-Hammer successfully decides all the examples that are expressable in its language, including those failed by DiracDec because of complexity or insufficient decision power.


\section{Motivation and Preliminaries}


An interesting property of quantum mechanics is that for two maximally entangled states, applying a quantum operator \( M \) to one subsystem is equivalent to applying \( M^T \) (the transpose of \( M \)) to the other subsystem. This relationship holds regardless of the spatial separation between the two systems, and it can be expressed as an equation in labelled Dirac notation.
\begin{example}
    \label{ex: motivating}
    Let \( q \) and \( r \) represent two quantum systems in the Hilbert space \( \mathcal{H}_T \). Let \( M \) be a quantum operation acting on \( \mathcal{H}_T \), and let \( \ket{\Phi} = \sum_{i \in T} \ket{i} \otimes \ket{i} \) be the maximally entangled state. Then, we have the following equation:
    \[
    M_q \ket{\Phi}_{(q, r)} = M_r^T \ket{\Phi}_{(q, r)}.
    \]
\end{example}

A little explanation on the notation and terms.
Quantum states are represented as vectors in complex Hilbert spaces, and operations on these states are described by linear transformations, or operators. Dirac notation uses the ket \( \ket{i} \) and the bra \( \bra{i} \) to denote basis vectors in a Hilbert space and its dual space, respectively. These symbols can be composed together in various ways to form more complex expressions.
For example, $M \ket{i}$ represents the operator-vector multiplication, and the tensor product \(\ket{i}\otimes\ket{i}\) describes the product state on multiple quantum systems.



In this equation, \( q \) and \( r \) are labels denoting the respective subsystems in the Dirac notation. $M_q$ means that the operation $M$ is applied on system $q$, and $\ket{\Phi}_{(q, r)}$ denotes the quantum state $\ket{\Phi}$ on the (ordered) composed system $(q, r)$.
To reason about their equivalence, we can remove the labels by extending them to the global system $(q, r)$:
\begin{align}
    (M \otimes I) \ket{\Phi} = (I \otimes M^T) \ket{\Phi}.
    \label{eq: motivating unlabelled}
\end{align}
Here the identity operator $I$ are inserted in suitable positions to extend the operator $M$, a special case of extending to the global system called cylinder extension.






\subsubsection{Universal Algebra}
We use universal algebra and equational logic to formally represent Dirac notation and the reasoning procedure. A universal algebra defines a signature of function symbols, with terms constructed from constants, variables, and function applications. 
Other basic concepts like substitution of variables or pattern matching are also defined.
In our case of Dirac notation, the signature consists of constructors and operations like $\ket{i}$ or $A \otimes B$.
The reasoning process is guided by equational logic, which defines an equivalence relation that is compatible with substitution and term construction. This relation formalizes the intuitive concept of equivalence in algebra.

