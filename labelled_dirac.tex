

\newcommand{\fst}{\mathsf{fst}\ }
\newcommand{\snd}{\mathsf{snd}\ }

\section{Labelled Dirac Notation}

\begin{definition}[quantum registers]
  \begin{align*}
    R ::= x\ |\ (R, R)\ |\ \fst R\ |\ \snd R
  \end{align*}
  We define the following relations for quantum registers:
  \begin{itemize}
    \item $R$ \textbf{belongs to} $Q$, written as $R\ \text{in}\ Q$,
    \item $R$ \textbf{is disjoint with} $Q$, written as $R \| Q$.
  \end{itemize}
\end{definition}

\textbf{Remark:} We still have a speical algorithm deciding the relations.

\begin{definition}[register set]
  \begin{align*}
    S ::= \emptyset\ |\ \{ R \} \ |\ S \cup S\ |\ S \setminus S
  \end{align*}
\end{definition}

\textbf{Remark: } $ S_1 \cap S_2 \equiv S_1 \cup S_2 \setminus (S_1 \setminus S_2) \setminus (S_2 \setminus S_1) $

\subsubsection*{\textsf{REG}}
\begin{gather*}
  \fst (R_1, R_2) \reduce R_1 \qquad 
  \snd (R_1, R_2) \reduce R_2 \qquad
  (\fst R, \snd R) \reduce R
\end{gather*}


\subsubsection*{\textsf{RSET}}
\begin{gather*}
  S \cup \emptyset \reduce S
  \qquad
  S \cup S \reduce S
  \qquad
  \{ \fst R \} \cup \{ \snd R \} \reduce R \\
  S \setminus \emptyset \reduce S
  \qquad
  \emptyset \setminus S \reduce \emptyset
  \qquad
  S \setminus S \reduce \emptyset \\
  (S_1 \cup S_2) \setminus X \reduce (S_1 \setminus X) \cup (S_2 \setminus X)
  \qquad
  S_1 \setminus (S_2 \cup S_3) \reduce (S_1 \setminus S_2) \setminus S_3 \\
  \\
  \frac{ R_1 \textrm{ in } R_2 }{ \{ R_1 \} \cup \{ R_2 \} \reduce \{ R_2 \} }
  \qquad
  \frac{ R_1 \textrm{ in } R_2 }{ \{ R_1 \} \setminus \{ R_2 \} \reduce \emptyset } \\
  \\
  \frac{ R_1 \textrm{ in } R_2 }{ \{R_2\} \setminus \{R_1\} \reduce (\{\fst R_2\} \setminus \{R_1\}) \cup (\{\snd R_2\} \setminus \{R_1\})}
  \qquad
  \frac{R_1 \| R_2}{ \{ R_1 \} \setminus \{ R_2 \} \reduce \{ R_1 \}} 
\end{gather*}

\begin{definition}[labelled core language]
  The \textbf{labelled core langauge} includes all symbols in the core language of Dirac notation, as well as symbols for the three new sorts.
  \begin{align*}
    & && S ::=\ \tilde{B} \cdot \tilde{K} \\
    & \tilde{\mathcal{K}}\text{(labelled ket)}&& \tilde{K} ::=\ K_{R}\ |\ \tilde{B}^\dagger\ |\ S.\tilde{K}\ |\ \tilde{K} + \tilde{K}\ |\ \tilde{O} \cdot \tilde{K}\ |\ \tilde{K} \otimes \tilde{K} \\
    & \tilde{\mathcal{B}}\text{(labelled bra)} && \tilde{B} ::=\ B_{R}\ |\ \tilde{K}^\dagger\ |\ S.\tilde{B}\ |\ \tilde{B} + \tilde{B}\ |\ \tilde{B} \cdot \tilde{O}\ |\ \tilde{B} \otimes \tilde{B} \\
    & \tilde{\mathcal{O}}\text{(labelled operator)} &&\tilde{O} ::=\ O_{R; R}\ |\ \tilde{K}\otimes\tilde{B}\ |\ \tilde{O}^\dagger\ |\ S.\tilde{O}\ |\ \tilde{O} + \tilde{O}\ |\ \tilde{O} \cdot \tilde{O}\ |\ \tilde{O} \otimes \tilde{O}
  \end{align*}
  In other words, we don't allow variables for labelled core language for now.
\end{definition}

\subsubsection*{\textsf{LABEL-CORE}}
We generally copied the symbols ($\dagger, S.\tilde{X}, +, \cdot, \otimes$) from the core language. Therefore we also need a copy of the corresponding rewriting rules.

\subsubsection*{\textsf{TSR-DECOMP}}

\begin{gather*}
  \ket{(s, t)}_{(Q, R)} \reduce \ket{s}_{Q} \otimes \ket{t}_{R}
  \qquad
  \bra{(s, t)}_{(Q, R)} \reduce \bra{s}_{Q} \otimes \bra{s}_{R} \\
  {\mathbf{0}_\mathcal{K}}_{(Q, R)} \reduce {\mathbf{0}_\mathcal{K}}_{Q} \otimes {\mathbf{0}_\mathcal{K}}_{R}
  \qquad
  {\mathbf{0}_\mathcal{B}}_{(Q, R)} \reduce {\mathbf{0}_\mathcal{B}}_{Q} \otimes {\mathbf{0}_\mathcal{B}}_{R} \\
  {\mathbf{0}_\mathcal{O}}_{(Q, R); (S, T)} \reduce {\mathbf{0}_\mathcal{O}}_{(Q, S)} \otimes {\mathbf{0}_\mathcal{O}}_{(R, T)} \\
  {\mathbf{0}_\mathcal{O}}_{(Q_1, Q_2); R} \reduce {\mathbf{0}_\mathcal{O}}_{Q_1; \fst R} \otimes {\mathbf{0}_\mathcal{O}}_{Q_2; \fst R}
  \qquad
  {\mathbf{0}_\mathcal{O}}_{Q; (R_1, R_2)} \reduce {\mathbf{0}_\mathcal{O}}_{\fst Q; R_1} \otimes {\mathbf{0}_\mathcal{O}}_{\snd Q; R_2} \\
  {\mathbf{1}_\mathcal{O}}_{(Q, R); (Q, R)} \reduce {\mathbf{1}_\mathcal{O}}_{Q; Q} \otimes {\mathbf{1}_\mathcal{O}}_{R; R} \\
  {\mathbf{1}_\mathcal{O}}_{(Q_1, Q_2); R} \reduce {\mathbf{1}_\mathcal{O}}_{Q_1; \fst R} \otimes {\mathbf{1}_\mathcal{O}}_{Q_2; \fst R}
  \qquad
  {\mathbf{1}_\mathcal{O}}_{Q; (R_1, R_2)} \reduce {\mathbf{1}_\mathcal{O}}_{\fst Q; R_1} \otimes {\mathbf{1}_\mathcal{O}}_{\snd Q; R_2}
\end{gather*}
\begin{gather*}
  (K_1 \otimes K_2)_{(Q, R)} \reduce {K_1}_{Q} \otimes {K_2}_{R}
  \qquad
  (B_1 \otimes B_2)_{(Q, R)} \reduce {B_1}_{Q} \otimes {B_2}_{R} \\
  (O_1 \otimes O_2)_{(Q, R); (S, T)} \reduce {O_1}_{Q; S} \otimes {O_2}_{R; T} \\
  (O_1 \otimes O_2)_{(Q_1, Q_2); R} \reduce {O_1}_{Q_1; \fst R} \otimes {O_2}_{Q_2; \snd R}
  \qquad
  (O_1 \otimes O_2)_{Q; (R_1, R_2)} \reduce {O_1}_{\fst Q; R_1} \otimes {O_2}_{\snd Q; R_2}
\end{gather*}


\subsubsection*{\textsf{TSR-COMP}}
\begin{gather*}
  {K_1}_{\fst R} \otimes {K_1}_{\snd R} \reduce (K_1 \otimes K_2)_{R}
  \qquad
  {B_1}_{\fst R} \otimes {B_2}_{\snd R} \reduce (B_1 \otimes B_2)_{R} \\
  {O_1}_{\fst Q; \fst R} \otimes {O_2}_{\snd Q; \snd R} \reduce (O_1 \otimes O_2)_{Q; R}
\end{gather*}

\subsubsection*{\textsf{DOT-TSR}}
\begin{align*}
  \frac{R \| S}{{O_1}_{Q; R} \cdot {O_2}_{S; T} \reduce {O_1}_{Q; R} \otimes {O_2}_{S; T}}
\end{align*}

\subsubsection*{\textsf{LABEL-LIFT}}
\begin{gather*}
  (K_{R})^\dagger \reduce (K^\dagger)_{R}
  \qquad
  (B_{R})^\dagger \reduce (B^\dagger)_{R}
  \qquad
  (O_{Q; R})^\dagger \reduce (O^\dagger)_{Q; R} \\
  (K_{R})^\top \reduce (K^\top)_{R}
  \qquad
  (B_{R})^\top \reduce (B^\top)_{R}
  \qquad
  (O_{Q; R})^\top \reduce (O^\top)_{Q; R} \\
  (S.K)_{R} \reduce S.(K_{R})
  \qquad
  (S.B)_{R} \reduce S.(B_{R})
  \qquad
  (S.O)_{Q; R} \reduce S.(O_{Q; R}) \\
  (K_1 + K_2)_{R} \reduce {K_1}_{R} + {K_2}_{R}
  \qquad
  (B_1 + B_2)_{R} \reduce {B_1}_{R} + {B_2}_{R}
  \qquad
  (O_1 + O_2)_{Q; R} \reduce {O_1}_{Q; R} + {O_2}_{Q; R}
\end{gather*}
\begin{gather*}
  {O_1}_{Q; R} \cdot {O_2}_{R; S} \reduce (O_1 \cdot O_2)_{Q; S}
  \qquad
  O_{Q; R} \cdot K_{R} \reduce (O \cdot K)_{Q}
  \qquad
  B_{Q} \cdot O_{Q; R} \reduce (B \cdot O)_{R} \\
  B_{R} \cdot K_{R} \reduce B \cdot K \\
  (K \otimes B)_{Q; R} \reduce K_{Q} \otimes B_{R}
\end{gather*}

\subsubsection*{\textsf{OPT-EXT}}
I think the concept ``cylinder extension'' is only limited to endomorphisms. Besides, one quantum register should a sub-register of the other one, which is defined as follows:

\begin{definition}[sub-register]
  \begin{gather*}
    \fst R \preceq R \qquad \snd R \preceq R \qquad Q \preceq (Q, R) \qquad R \preceq (Q, R)
    \qquad 
    \frac{Q \preceq R \qquad R \preceq S}{Q \preceq S}
  \end{gather*}

  And we can further calculate the ``position'' of sub-register, which will be utilized during cylinder extension: assume $Q$ is a sub-register of $R$, then the position of $Q$ in $R$ is a string defined as follows:
  \begin{align*}
    & \textrm{pos}(\fst R, R) = 0 \\
    & \textrm{pos}(\snd R, R) = 1 \\
    & \textrm{pos}(Q, (Q, R)) = 0 \\
    & \textrm{pos}(R, (Q, R)) = 1 \\
    & \textrm{pos}(Q, S) = \textrm{pos}(R, S)\ \textrm{pos}(Q, R)
  \end{align*}
\end{definition}

\textbf{Remark:} For a well-formed quantum register, the sub-register position is well-defined.

\begin{definition}[cylinder extension]
  \begin{align*}
    \textrm{ext}(O, \epsilon) \equiv O
    && \textrm{ext}(O, p::0) \equiv O \otimes \mathbf{1}_\mathcal{O}
    && \textrm{ext}(O, p::1) \equiv \mathbf{1}_\mathcal{O} \otimes O
  \end{align*}
\end{definition}

\subsubsection*{\textsf{CYLINDER-EXT}}
\begin{gather*}
  \frac{Q \textrm{ is a subterm of } R \textrm{ at } p}{O_{Q; Q} \cdot K_{R} \reduce (\textrm{ext}(O, p) \cdot K)_{R}}
  \qquad
  \frac{Q \textrm{ is a subterm of } R \textrm{ at } p}{{O_1}_{Q; Q} \cdot {O_2}_{R; T} \reduce (\textrm{ext}(O_1, p) \cdot O_2)_{R; T}} \\
  \\
  \frac{R \textrm{ is a subterm of } Q \textrm{ at } p}{B_{Q} \cdot O_{R; R} \reduce (B \cdot \textrm{ext}(O, p))_{Q}}
  \qquad
  \frac{R \textrm{ is a subterm of } Q \textrm{ at } p}{{O_1}_{T; Q} \cdot {O_2}_{R; T} \reduce (O_1 \cdot \textrm{ext}(O_2, p))_{T; Q}} \\
\end{gather*}


\subsection{Labelled Extended Language}
\begin{definition}[labelled extended language]
  The \textbf{labelled extended language} consists of the symbols in labelled core langauge and unlabelled extended language, and add the new symbols of transpose and big-op for labelled bra, ket and operators, which is described in the following.
  \begin{align*}
    \tilde{K} ::= \tilde{B}^\top\ |\ \sum_{i \in M} \tilde{K}
    && \tilde{B} ::= \tilde{K}^\top\ |\ \sum_{i \in M} \tilde{B}
    && \tilde{O} ::= \tilde{O}^\top\ |\ \sum_{i \in M} \tilde{O}
  \end{align*}
  
\end{definition}


\subsubsection*{\textsf{LABEL-SUM}}
\begin{gather*}
  (\sum_{i \in M} K)_{R} \reduce \sum_{i \in M} (K_{R})
  \qquad
  (\sum_{i \in M} B)_{R} \reduce \sum_{i \in M} (B_{R})
  \qquad
  (\sum_{i \in M} O)_{Q; R} \reduce \sum_{i \in M} (O_{Q; R})
\end{gather*}

\subsubsection*{\textsf{LABEL-TEMP}}
\yx{These are ad-hoc rules for the examples for now. They are still not organized and require further investigations.}
\begin{gather*}
  \frac{R \textsf{ in } Q}{O_{P; R} \cdot (K_{Q} \otimes \tilde{K'}) \reduce (O_{P; R} \cdot K_{Q}) \otimes \tilde{K'}} 
  \qquad
  \frac{R \textsf{ in } Q}{O_{P; R} \cdot (K_{Q} \otimes \tilde{B}) \reduce (O_{P; R} \cdot K_{Q}) \otimes \tilde{B}} \\
  \\
  \frac{R \textsf{ in } Q}{{O_1}_{P; R} \cdot ({O_2}_{Q; T} \otimes \tilde{O_3}) \reduce ({O_1}_{P; R} \cdot {O_2}_{Q; T}) \otimes \tilde{O_3}} \\
  \\
  \frac{Q \textsf{ in } R}{(B_{R} \otimes \tilde{B'}) \cdot O_{Q;P} \reduce (B_{R} \cdot O_{Q;P}) \otimes \tilde{B'}} 
  \qquad
  \frac{Q \textsf{ in } R}{({O_1}_{T;R} \otimes \tilde{O'}) \cdot {O_2}_{Q;P} \reduce ({O_1}_{T;R} \cdot {O_2}_{Q;P}) \otimes \tilde{O'}} \\
  \\
  B_{R} \cdot (K_{R} \otimes \tilde{B'}) \reduce (B_{R} \cdot K_{R}) . \tilde{B'} 
  \qquad
  (\tilde{K'} \otimes B_{R}) \cdot K_{R} \reduce (B_{R} \cdot K_{R}) . \tilde{K'}
\end{gather*}
