
\section{Related Work}


The first step is to formalize the language for Dirac notation. In DiracDec, the language has a concrete design, where the same syntax for different types corresponds to distinct symbols. Our goal is to transition from this concrete design to a more abstract one, which aligns more closely with the conventional Dirac notation we use and simplifies the term rewriting system.

Automated theorem proving has seen significant advancements in recent years, particularly through the development of satisfiability modulo theories (SMT) solvers,
% SMT solvers extend propositional satisfiability by integrating theories such as arithmetic, arrays, and bit-vectors, making them suitable for solving a wide range of verification problems in both hardware and software domains. 
including prominent tools like Z3. These solvers have become essential in various fields such as formal verification, synthesis, and model checking. 
Equational reasoning is another crucial area of research within automated theorem proving, which focuses on solving problems that involve equations between terms in an algebraic structure. 
Equational provers, such as Vampire and E, have played a pivotal role in addressing the challenge of proving equations in first-order logic, employing sophisticated algorithms like superposition and term rewriting.

Formal verification of quantum computation is receiving increasing attentions during these years. See~\cite{Lewis2023} for a comprehensive review. 
Verification frameworks in Coq include the foundational formalization CoqQ and quantum circuit language QWIRE. 
Verifications of quantum programs are also considered, such as the Hoare logic based methods [] and model checking based methods [].
The equational reasoning of Dirac notation is crucial for establishing property proofs in these works.
Verification theoreis and tools based on other languages are also proposed, such as PyZX for reasoning and simplification of ZX-calculus.




\section{Conclusion and Future Work}
Based on the first equational reasoning tool for Dirac notation called DiracDec, this work improves and extends the theory for practical applications, and provides the solver D-Hammer. Experiments show that the tool demonstrates advantages in decidability, efficiency and usability. 

We expect D-Hammer to have applications in areas like quantum program verification or proofs of post-quantum cryptography protocols in the future.
One promising following up is to connect D-Hammer with theorem provers like Coq. It involves transforming theorem prover expressions into D-Hammer, and verify the proof trace of D-Hammer in theorem provers. Besids, most quantum program verifiers nowadays depend on matrix calculations. D-Hammer can serve as the replacement for matrix methods to enable symbolic deductions.

