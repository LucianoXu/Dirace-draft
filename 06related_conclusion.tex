\section{Related Work}

\paragraph*{Comparison with DiracDec~\cite{diracdec}}
Xu \emph{et al}~\cite{diracdec} define a language and an associate and
commutative rewriting system for Dirac notation, and implement their
language and rewriting system in Mathematica. Our approach follows a
similar pattern. However, there are significant differences in terms
of scope, expressiveness, and efficiency. We review these differences
below.

The most obvious difference is that D-Hammer targets labelled Dirac
notation, whereas DiracDec targets plain Dirac notation. As already
mentioned, the use of labelled Dirac notation is essential in many
applications; in particular, labelled Dirac notation can simplify
notation and proofs, and is in general better suited for writing and
reasoning about complex, many-body, quantum states. However, there are
several other important differences. First, our language leverages
higher-order functions to provide a compact and expressive
representation of big operators. Second, our language adopts AC
symbols with indefinite arities, which leads to compact representation
of terms, and eases AC reasoning. Third, the relation with Mathematica
is fundamentally different. DiracDec is implemented in Mathematica; as
a consequence, its behavior, and in particular, its typing rules are
constrained by the lack of typing in Mathematica. In contrast,
D-Hammer is designed as a separate tool; as a consequence, it benefits
from an improved representation of terms (e.g.\, AC operators with
indefinite arity), a more expressive type system (e.g.\, dependent
types), and a more efficient rewriting engine. D-Hammer still relies
on Mathematica to reason about functions that are not natively
supported by rewriting rules, but these interactions are constrained
and do not have negative effects on the overall efficiency of the
system. This is reflected in our experimental comparison of D-Hammer
and DiracDec, which shows how the former outperforms the latter.


\paragraph*{Comparison with ZX Calculus}
The ZX calculus~\cite{DBLP:conf/icalp/CoeckeD08,coecke2011interacting}
is a graphical calculus for quantum states. A main appeal of the ZX
calculus is that its foundations are grounded in categorical quantum
mechanics~\cite{DBLP:conf/lics/AbramskyC04}, a powerful framework for
modeling and reasoning about quantum physics. Another main appeal of
the ZX calculus is that ZX diagrams have an operational interpretation
using graph rewriting. There is a large body of work that studies
properties of these rewriting systems, and in particular completeness
(a set of rules is complete if it can prove all valid identities) and
minimality (a complete set of rules is minimal if removing a rule
leads to incompleteness). Two main proof techniques for completeness
are via termination, or via interpretation. In the first case, one
shows that a rewriting system has unique normal forms and that two
expressions are semantically equivalent iff they have the same normal
form---there exists many relaxations of this result---whereas in the
second case one shows completeness by exhibiting a well-behaved
translation to another system for which completeness holds. Both proof
techniques have been used to prove completeness for multiple settings,
including the Clifford fragment~\cite{backens2014zx}, the
Toffoli+Hadamard fragment~\cite{hadzi15zx}, the Clifford+T
fragment~\cite{jeandel2018complete} and the (qubit) universal
fragment~\cite{HNW18,JeandelPV18beyond}. Subsequent works generalize
completeness to qudits~\cite{Poor2023} for a calculus know as ZXW
which encompasses the ZX and ZW calculi~\cite{zwcalculus}, quantum
circuits~\cite{DBLP:conf/lics/ClementHMPV23} or optical quantum
circuits~\cite{clement_et_al:2022}. We refer the interested reader
to~\cite{vandewetering2020zx} for a historical and technical account
of completeness results up to 2020.

Because they are both grounded in rewriting techniques, there are many
similarities between prior work on the ZX calculus and our own work on
labelled Dirac notation. We highlight some key differences. First,
there exist a number of differences between our language and the
concrete fragments of the ZX-calculus for which completeness has been
studied. For instance, our language features big operators, and
operators for dual spaces. A second difference is that we do not claim
completeness: our main theorem shows a reduction from Labelled Dirac
notation to Dirac notation, for which completeness is only partially
known. From a practical point of view, many of the examples handled by
D-Hammer cannot be formalized in existing ZX-calculus tools. On the
contrary, D-Hammer is not competitive on the examples that can be
verified by ZX-calculus. \gb{give a few examples in terms of size and time}



\paragraph*{Comparison with quantum circuit equivalence checking tools}
A main application of the ZX-calculus is to check equivalence of
quantum circuits; for instance, PyZX~\cite{kissinger2020pyzx} is a
popular Python library for automating equational reasoning in the
ZX-calculus.

There are many other tools for equivalence checking of quantum
circuits. Amy~\cite{amy2018towards} uses the path-sums formalism
formalism to check circuits with thousands of T gates. A series of
works~\cite{AutoQ_pldi_2023,AutoQ2023,AutoQ_popl2025} explores
automata-based approaches to verify quantum circuits at scale. Their
main idea is to encode circuits as a compact, finite tree automata,
often with $\mathit{poly}(n)$ transitions (instead of $2^n$
transitions), where $n$ is the number of qubits of the state. By
providing the semantics of quantum gates over this representation,
they achieve practical verification. Their tool \textsc{autoq} is able
to verify circuits with over 100,00 gates, and was recently extended
to support parametrized verification.


\paragraph*{Canonical forms in multi-body quantum physics}
Canonical forms play a fundamental role in quantum physics. For
instance, \cite{perezgarcia2007,RevModPhys.93.045003,Acuaviva_2023}
discuss canonical forms of Matrix Product States (MPS) and Tensor
Networks respectively. An exciting direction for further work is to
further develop automated deduction techniques for quantum physics.

\paragraph*{Comparison with egraphs}
Our algorithm is based on term rewriting. However, it is a challenge
to device well-behaved and efficient term rewriting for (labelled)
Dirac notation. An alternative to term rewriting would be to use
equality saturation~\cite{DBLP:journals/pacmpl/WillseyNWFTP21}, a
powerful equational reasoning technique that does not require
existence of normal forms. Equality saturation may be particularly
useful when considering further extensions of labelled Dirac notation.

%% The first step is to formalize the language for Dirac notation. In DiracDec, the language has a concrete design, where the same syntax for different types corresponds to distinct symbols. Our goal is to transition from this concrete design to a more abstract one, which aligns more closely with the conventional Dirac notation we use and simplifies the term rewriting system.

%% %% Automated theorem proving has seen significant advancements in recent years, particularly through the development of satisfiability modulo theories (SMT) solvers,
%% %% % SMT solvers extend propositional satisfiability by integrating theories such as arithmetic, arrays, and bit-vectors, making them suitable for solving a wide range of verification problems in both hardware and software domains. 
%% %% including prominent tools like Z3. These solvers have become essential in various fields such as formal verification, synthesis, and model checking. 
%% %% Equational reasoning is another crucial area of research within automated theorem proving, which focuses on solving problems that involve equations between terms in an algebraic structure. 
%% %% Equational provers, such as Vampire and E, have played a pivotal role in addressing the challenge of proving equations in first-order logic, employing sophisticated algorithms like superposition and term rewriting.

%% %% Formal verification of quantum computation is receiving increasing attentions during these years. See~\cite{Lewis2023} for a comprehensive review. 
%% %% Verification frameworks in Coq include the foundational formalization CoqQ and quantum circuit language QWIRE. 
%% %% Verifications of quantum programs are also considered, such as the Hoare logic based methods [] and model checking based methods [].
%% %% The equational reasoning of Dirac notation is crucial for establishing property proofs in these works.
%% %% Verification theoreis and tools based on other languages are also proposed, such as PyZX for reasoning and simplification of ZX-calculus.




\section{Conclusion and Future Work}
We have designed and implemented D-Hammer, a dependently typed
higher-order language and proof system for labelled Dirac
notation. D-Hammer benefits from an optimized implementation in
\CC\ and a tight integration with Mathematica to reason about a broad
range of mathematical functions, including trigonometric and
exponential functions, that are commonly used in quantum physics.
There are two important directions for future work. 
% The first
% direction is to enhance D-Hammer with support for big tensor notation,
% i.e.\, to support expressions of the form $\bigotimes_{i\in I}
% e_i$. 
The first direction is to extend D-Hammer with a mechanism to
generate independently verifiable certificates.  There is a large body
of work on producing certificates for automated tools, in particular
SMT solvers; see e.g.~\cite{DBLP:journals/cacm/BarbosaBCDKLNNOPRTZ23}
for a recent overview. One potential option would be to integrate
D-Hammer with the Coq or Lean proof assistants; in the first case, one
would benefit from the formalization of labelled Dirac notation in
CoqQ~\cite{Zhou2023}, whereas in the second case, one would benefit
from powerful mechanisms to integrate rewriting procedures into the
Lean proof assistant. The second direction is to connect
D-Hammer with quantum program verifiers. Two potential applications
are automating equational proofs for tools that already use Dirac
notation, and to substitute numerical methods for tools that use
matrices instead of symbolic assertions.




%% Based on the first equational reasoning tool for Dirac notation called DiracDec, this work improves and extends the theory for practical applications, and provides the solver D-Hammer. Experiments show that the tool demonstrates advantages in decidability, efficiency and usability. 

%% We expect D-Hammer to have applications in areas like quantum program verification or proofs of post-quantum cryptography protocols in the future.
%% One promising following up is to connect D-Hammer with theorem provers like Coq. It involves transforming theorem prover expressions into D-Hammer, and verify the proof trace of D-Hammer in theorem provers. Besids, most quantum program verifiers nowadays depend on matrix calculations. D-Hammer can serve as the replacement for matrix methods to enable symbolic deductions.

