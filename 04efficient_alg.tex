


%%%%%%%%%%%%%%%%%%%%%%%%%%%%%%%%%%%%%%%%%%%%%%%%%%%%%%%%%%%%%%%%%%

\section{Efficient Algorithm for Equivalence Checking}
\label{sec: decide}
The equivalence of Dirac notations are decided by normalization, which transforms equivalent expressions under a set of axioms into the same syntax. For the entire axiom set $R \cup F$ of Dirac notation, the previous algorithm conducts partial normalization on $R$ by term rewriting modulo $F$, and axioms in $F$ are checked by iterating through all permutations. This direct solution has factorial compleixty on maximum number of AC symbol arguemnts, and is observed to be critical to the low efficiency. Here we propose an efficient algorithm to perform total nomralization on $R \cup F$, and their comparison are illustrated below.
\[
\text{(partial)}\qquad e_1 \mathop{\twoheadrightarrow}^R e_1' \mathop{=}^F e_2' \mathop{\twoheadleftarrow}^R e_2
\qquad
\text{(total)}\qquad e_1 \mathop{\twoheadrightarrow}^{R \cup F} e \equiv e \mathop{\twoheadleftarrow}^{R \cup F} e_2
\]

% The problem is that, the axiom set $F$ in~\Cref{def: axiom F} cannot be decided through term rewriting. A direct solution is to decide \( E \) by examining all possible permutations in the rewriting result, which has high complexity. 
% In this work, we introduce a more efficient approach by incorporating sorting algorithms into the normalization procedure. 

The normalization procedure consists of the following steps. Steps 1 - 3 are term rewriting steps for $R$, and steps 4 - 6 are our algorithm to further normalize axiom set $F$.
\begin{enumerate}
    \item \textbf{First Rewritings}: Expand definitions and simplify expressions.
    \item \textbf{Variable Expansion}: Convert to scalar expressions for better decidability.
    \item \textbf{Second Rewritings}: Normalize terms on \( R \) modulo \( F \).
    \item \textbf{Sorting Without Bound Variables}: Normalize AC-equivalence.
    \item \textbf{Swap Successive Summations}: Normalize SUM-SWAP equivalences.
    \item \textbf{De Bruijn Normalization}: Normalize \( \alpha \)-equivalence.
\end{enumerate}


% Term rewriting rules, represented as \( l \ \reduce\ r \), are applied recursively to normalize terms. In each step, subterms matching the left-hand side \( l \) of a rule are replaced with the corresponding right-hand side \( r \). The procedure terminates when no further rewritings can be made.
% A comprehensive list of rewriting rules can be found in~\Cref{sec: rewriting rules}. Below are some key examples to illustrate the design:

% One of our optimizations is using functions with indefinite arities. 
% Therefore, we use a flattening rule to handle associativity with AC symbols:
% \[
% a_1 + \cdots + (b_1 + \cdots + b_m) + \cdots + a_n \ \reduce\ a_1 + \cdots + b_1 + \cdots + b_m + \cdots + a_n.
% \]
% Commutativity is handled later in the sorting step. Many of the rewriting rules are directly derived from the equational axioms, such as:
% \begin{align*}
%     & \textrm{(R-DOT6)} && \bra{s} \cdot \ket{t} \ \reduce\  \delta_{s, t}, \\
%     & \textrm{(R-DELTA0)} && \delta_{s, s} \ \reduce\  1, \\
%     & \textrm{(R-MULK1)} && O : \OType(\sigma, \tau) \Rightarrow O \cdot \mathbf{0}_{\KType(\tau)} \ \reduce\  \mathbf{0}_{\KType(\sigma)}, \\
%     & \textrm{(R-MULK11)} && (O_1 \otimes O_2) \cdot (K_1 \otimes K_2) \ \reduce\  (O_1 \cdot K_1) \otimes (O_2 \cdot K_2).
% \end{align*}
% Some of these directions are obvious. For example, (R-DOT6) states that the inner product of two basis vectors is reduced to a delta expression, and (R-DELTA0) transforms a delta function on identical basis to a scalar \( 1 \). The rule (R-MULK1) reflects the axiom that multiplying a zero vector results in zero. This rule is conditional on typing, which is checked during rewriting. Some rules, like (R-MULK11), require a deeper understanding, such as the preference for tensor products over multiplication.

% As a reference, the term rewriting system in DiracDec has been proven complete for all axioms, except for the sum symbol. The completeness result is derived from checking the confluence and termination of the system. Our rewriting rules are translations from DiracDec into the typed and abstract language, ensuring that the corresponding symbols in our system are also complete.

% We also have additional rules that handle summations. For example:
% \begin{align*}
%     & \quad i \text{ free in } t \Rightarrow \sum_{i \in \mathbf{U}(\sigma)} \sum_{k_1 \in s_1} \cdots \sum_{k_n \in s_n} (\delta_{i, t}.A) \ \reduce\  \sum_{k_1 \in s_1} \cdots \sum_{k_n \in s_n} A\{i/t\}, \\
%     & \quad \left( \sum_{i \in M} B \right) \cdot K \ \reduce\  \sum_{i \in M} (B \cdot K).
% \end{align*}
% The first rule eliminates delta expressions in summations, while the second rule pushes summations outside of inner products. While there is no guarantee of completeness for these rules, they work effectively in practice.

% \subsubsection{Variable Expansion}
% One important technique, revealed in the DiracDec work, is the expansion of variables, which is critical for proofs involving summation. For example:
% \[
% \frac{E[\Gamma] \vdash K : \KType(\sigma)}{E[\Gamma] \vdash K \ \reduce\ \sum_{i \in \mathbf{U}(\sigma)}(\bra{i} \cdot K).\ket{i}} \quad \quad
% \frac{E[\Gamma] \vdash B : \BType(\sigma)}{E[\Gamma] \vdash B \ \reduce\ \sum_{i \in \mathbf{U}(\sigma)}(B \cdot \ket{i}).\bra{i}},
% \]
% \[
% \frac{E[\Gamma] \vdash O : \OType(\sigma, \tau)}{E[\Gamma] \vdash O \ \reduce\ \sum_{i \in \mathbf{U}(\sigma)} \sum_{j \in \mathbf{U}(\tau)}(\bra{i} \cdot O \cdot \ket{j}).(\ket{i} \cdot \bra{j})}.
% \]
% These rules transform variables into their symbolic summations based on their decomposition over the basis. 
% The rules are not terminating, therefore is applied recrusively once in the second step called \textit{variable expansion}.
% Nevertheless, we have found that applying the expansion only once for all variables is sufficient for normalization.
% \begin{lemma}
%     Let \( \textrm{expand}(e) \) denote the result of expanding all variables in \( e \) once. For all well-typed terms \( e \) in \( E[\Gamma] \), \( \textrm{expand}(\textrm{expand}(e)) \) and \( \textrm{expand}(e) \) have the same normal form.
% \end{lemma}
% \begin{proof}
%     Expanding a ket variable twice, for example, results in the following transformation:
%     \[
%     \sum_{i \in \mathbf{U}(\sigma)} (\bra{i} \cdot \sum_{j \in \mathbf{U}(\sigma)} (\bra{j} \cdot K) \cdot \ket{j}) \cdot \ket{i} \ \reduce\  \sum_{i \in \mathbf{U}(\sigma)} \sum_{j \in \mathbf{U}(\sigma)} (\bra{j} \cdot K \cdot \braket{i|j}) \cdot \ket{i},
%     \]
%     where the delta symbol elimination rule returns the term to its original form. The same holds for bra and operator terms.
% \end{proof}



Now we analyse the axioms in $F$ to understand the difficulty and solution for normalization. 
For $\alpha$-equivalence, we want to rule out the influence of bound variable names. Therefore we use de Bruijn notation~\cite{deBruijn1972lambda}, which replaces the name with the distance from the lambda abstraction to the variable. For instance, the nominal lambda abstraction \( \lambda x. x \) is transformed into \( \lambda . \$0 \), while \( \lambda x. \lambda y. (x\ (y\ x)) \) is transformed into \( \lambda.\lambda. (\$1\ (\$0\ \$1)) \).

The remaining axioms, such as AC-equivalence and SUM-SWAP, assert equivalence under permutations. A standard approach for deciding such equivalences is to normalize terms by sorting in a predefined order. For example, given the dictionary order \( a < b < c \), the term \( b + c + a \) (and any other AC-equivalent term) is normalized into \( a + b + c \). However, in our setting, two intertwined difficulties arise: how to assign an order to all terms in the language, and how to simultaneously sort for both axioms.

Consider the following two equivalent terms:
\[
\sum_{i \in s_1} \sum_{j \in s_2} \bra{i} A \ket{j} \times \bra{j} B \ket{i}
= 
\sum_{i \in s_2} \sum_{j \in s_1} \bra{i} B \ket{j} \times \bra{j} A \ket{i}
\]
While these two terms are equivalent, directly sorting the elements of scalar multiplication using lexical order does not yield the same form.

To address this issue, we propose an algorithm to sort in two steps. The key observation is that in a successive sum expression \( \sum_{i \in s_1} \cdots \sum_{j \in s_n} A \), the names and order of the bound variables \( i, \dots, j \) can be freely permuted. Therefore, a good idea is to normalize AC-equivalence first, where all bound variables are treated uniformly. Afterwards, the order of summation can then be determined based on the position of the bound variables.

% \begin{align*}
%     \sum_{i \in s_1} \sum_{j \in s_2} \bra{i} A \ket{j} \times \bra{j} B \ket{i} 
%     \qquad
%     \sum_{i \in s_2} \sum_{j \in s_1} \bra{i} B \ket{j} \times \bra{j} A \ket{i} \\
%     \sum_{i \in s_1} \sum_{j \in s_2} \bra{i} A \ket{j} \times \bra{j} B \ket{i} 
%     \qquad 
%     \sum_{i \in s_2} \sum_{j \in s_1} \bra{j} A \ket{i} \times \bra{i} B \ket{j} \\
%     \sum_{i \in s_1} \sum_{j \in s_2} \bra{i} A \ket{j} \times \bra{j} B \ket{i} \\
%     \sum_{s_1} \sum_{s_2} \bra{\$1} A \ket{\$0} \times \bra{\$0} B \ket{\$1}
% \end{align*}

In the example above, we first ignore the bound variables and sort the sum body into \( \bra{\bullet} A \ket{\bullet} \times \bra{\bullet} B \ket{\bullet} \). Then, we swap the summations such that the bound variable at the first \( \bullet \) position appears at the outermost position. The results will have the same de Bruijn normal form, namely \( \sum_{s_1} \sum_{s_2} \bra{\$1} A \ket{\$0} \times \bra{\$0} B \ket{\$1} \).


To describe the algorithm in the following, we introduce two key notations. For a term \( e = f(a_1, a_2, \dots, a_n) \), \( \textrm{head}(e) \) denotes the function symbol \( f \), while \( \textrm{arg}(e, i) \) refers to the \( i \)-th argument \( a_i \) of the term. In this context, variables and constants are treated as functions with zero arguments.
\begin{definition}[Order Without Bound Variables]
Let \( \mathcal{B} \) represent the set of bound variables, with the assumption that all bound variables are unique. We also assume that a total order exists over all symbols. The relation \( e_1 =_\mathcal{B} e_2 \) holds if:
\begin{itemize}
    \item \( \textrm{head}(e_1) = \textrm{head}(e_2) \), and for all \( i \), \( \textrm{arg}(e_1, i) =_\mathcal{B} \textrm{arg}(e_2, i) \), or
    \item \( e_1 \in \mathcal{B} \) and \(e_2 \in \mathcal{B}\).
\end{itemize}

The relation \( e_1 <_\mathcal{B} e_2 \) holds between two terms if:
\begin{itemize}
    \item $e_1 \notin \mathcal{B}$ and $e_2 \in \mathcal{B}$, or
    \item $head(e_1) < head(e_2)$, or
    \item $head(e_1) = head(e_2)$, and there exists $n$ with $arg(e_1, n) <_\mathcal{B} arg(e_2, n)$, where $arg(e_1, i) =_\mathcal{B} arg(e_2, i)$ for all $i < n$.
\end{itemize}
\end{definition}
It can be shown that \( e_1 =_\mathcal{B} e_2 \) if and only if neither \( e_1 <_\mathcal{B} e_2 \) nor \( e_2 <_\mathcal{B} e_1 \) holds. The purpose of this ordering is to compare function symbols in a top-down manner while ignoring bound variables. This order enables normalization of terms for AC equivalence.
\begin{definition}[Sort Transformation]
    For a term $e$ with bound variable set $\mathcal{B}$,
    The sort transformation is defined in~\Cref{alg: sort}.
\end{definition}

\begin{algorithm}
    \caption{Sort Transformation}
    \label{alg: sort}
    \begin{algorithmic}[1]
        \Procedure{Sort}{$e, \mathcal{B}$}
            \If{$e \equiv \lambda x : T . e'$}
                \State \Return $\lambda x : T . \textsc{Sort}(e')$
            \ElsIf{$e \equiv \mu x. e'$}
                \State \Return $\mu x . \textsc{Sort}(e')$
            \ElsIf{$e \equiv f(a_1, \cdots, a_n)$}
                \State $ls := \textsc{Sort}(a_1), \cdots, \textsc{Sort}(a_n)$
                \State $ls := ls$ sorted by $<_\mathcal{B}$
                \State \Return $f(ls)$
            \EndIf
        \EndProcedure
    \end{algorithmic}
\end{algorithm}

After sorting, the next step is the \textit{swap transformation}, which arranges successive summations based on the order of bound variables.
\begin{definition}[Swap Transformation]
For a term \( e \) with a sorting result \( \textsc{Sort}(e) \), the swap transformation proceeds by ordering all bound variables according to their first appearances, except in function definitions \( \lambda x \) and \( \mu x \). The swap transformation then reorders the successive summations accordingly.
\end{definition}

Take the term $\sum_{i \in s_2} \sum_{j \in s_1} \bra{i} B \ket{j} \times \bra{j} A \ket{i}$ as an example. Its bound variable set $\mathcal{B} = \{i, j\}$. Assume we have $A < B$, then the sorting result will be $\sum_{i \in s_2} \sum_{j \in s_1} \bra{j} A \ket{i} \times \bra{i} B \ket{j}$. Then we decide that the order for bound variables will be $j < i$ because $j$ appears first in the body. Using the swap transformation, the sorted result will be $\sum_{j \in s_1} \sum_{i \in s_2} \bra{j} A \ket{i} \times \bra{i} B \ket{j}$.

% The order depends on there occurances in the last sorting result. If no occurance, then the order will depend on the set (for sum) and the type (for lambda abstraction only).

%%%%%%%%%%%%%%%%%%%%%%%


% The idea is to assign an order to terms, which is independent on the bound variables. Because we can have terms with nested AC symbols.

Lastly, we can prove that the equivalence established by this normalization procedure is sound with respect to the semantics.

\begin{theorem}[Soundness]
    For any well-formed context \( E[\Gamma] \) and well-typed expressions \( e_1 \) and \( e_2 \), if  $e_1$ and $e_2$ have the same normal form, then \( \sem{e_1} = \sem{e_2} \).
\end{theorem}

\begin{proof}
    The soundness of the term rewriting procedure follows from the fact that each rewriting rule preserves equivalence. Furthermore, the operations in the sort and swap transformations respect the AC-equivalence and SUM-SWAP axioms. Finally, the de Bruijn normalization ensures soundness for \( \alpha \)-equivalence.
\end{proof}


